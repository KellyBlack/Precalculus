\documentclass[11pt]{article}
\usepackage{amsmath, amssymb, array}
\pagestyle{plain}

\textwidth=6.5in
\hoffset-1in
\textheight=9in
\voffset-1in

\usepackage{enumitem}
\usepackage{pgfplots}
\usepackage{graphicx}
\usepackage{lipsum}
\usepackage{stfloats}
\usepackage{multicol}
\setlength{\columnsep}{1cm}




\begin{document}


\noindent MATH 1113   \hfill 1.4 - Lines\\



\noindent \textbf{Topics:}  Equations of lines, parallel and perpendicular lines, linear models, average rate of change\\

\noindent \textbf{Student Learning Outcomes:}
\begin{enumerate}
\item Students will be able to graph linear equations.
\item Students will be able to determine the slope of a line and apply the slope-intercept form of a line.
\item Students will be able to compute the average rate of change of a function.
\end{enumerate}

\hrule 
\vspace{5mm}
\section{Slope of a Line}

\noindent 
\begin{tabular}{| l |} \hline
The \underline{slope of a line} is $m = \dfrac{y_2-y_1}{x_2-x_1}$.  \\ \hline
 \end{tabular} 
 



\begin{enumerate}
\item Sketch the line through the pair of points $P(4,2)$ and $Q(-3,5 )$, and find the slope of the line. \\
\scalebox{0.3}{\includegraphics{bigaxes}} 

\hspace{-.3in} \begin{tabular}{| l |}\hline
 \noindent The \underline{point-slope equation} for the line through the point $(x_1,y_1)$ with slope $m$ is \\

 $y-y_1 = m(x-x_1)$. \\ \hline
\end{tabular} 

\vspace{-.1in}
\item Determine a \emph{point-slope} equation for the line through $P(4,2)$ and $Q(-3,5)$.(See above.)\\[1in]






\hspace{-.3in} \begin{tabular}{| l |}\hline The \underline{slope-intercept equation} for the line with slope $m$ and $y$-intercept $b$ is \\

 $y = mx +b$. \\ \hline
\end{tabular} 



\vspace{-.1in}
\item Determine the \emph{slope-intercept} equation for the line through $(3, -2)$ which has slope $-4$. \\[1in]



\noindent \textbf{Horizontal lines} have 0 slope and \textbf{vertical lines} have undefined slopes.\\[1in]


\section{Parallel and Perpendicular Lines}
\hspace{-.3in} \begin{tabular}{| l |  }
\hline Parallel lines have the \emph{same slope.} \\ \hline
\end{tabular} %\\[.5in]

\vspace{-.1in}
\item Determine a point-slope equation for the line through $(-1,2)$ which is parallel to the line $2x + 3y - 5 = 0$ and graph the equation.\\
\scalebox{0.3}{\includegraphics{bigaxes}} 
\\[1in]





\hspace{-.3in}\begin{tabular}{| l |  }
\hline Perpendicular lines have slopes that are ``negative reciprocals" of each other. If $m_1$ is the \\ slope of one of the lines, then the slope of the other line must be $-1/m_1$. \\ \hline
\end{tabular} 

\item Determine an equation of the line through the point $(-3, 1)$ which is perpendicular to the line $2x+4y+7=0$ and graph the equation.\\
\scalebox{0.3}{\includegraphics{bigaxes}} 
\\


\section{Average Rate of Change}
\textbf{Average Rate of Change}\\
Given any function $y=f(x)$, we calculate the average rate of change of $y$ with respect to $x$ over the interval  $[x_1,x_2]$ by dividing the change in value of $y$, $\Delta y=f(x_2)-f(x_1)$, by the length  $\Delta x=x_2-x_1$ of the interval over which the change occurs.\\


 The \textbf{\emph{average rate of change}} of $y=f(x)$ with respect to $x$ over the interval $[x_1,x_2]$ is
 $$\frac{\Delta y}{\Delta x}=\frac{y_2-y_1}{x_2-x_1}=\frac{f(x_2)-f(x_1)}{x_2-x_1}$$\\ 
 
\noindent \textbf{Note:  }An average rate of change needs two points (or endpoints on an interval). 






\item Given the function defined $f(x)=x^2-1$, determine the average rate of change from $x_1=-2$ to $x_2=0.$\\[2in]


\end{enumerate}

\noindent \textbf{Student Learning Outcomes Check}

\begin{enumerate}
\item Can you graph a linear equation?
\item Can you determine the slope of a line and apply the slope-intercept form of a line?
\item Can you compute the average rate of change of a function?
\end{enumerate}

\noindent \textbf{If any of your answers were no, please ask about these topics in class.}













\end{document}