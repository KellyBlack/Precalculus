\actTitle{4.2 - Trigonometric Functions Defined on the Unit Circle}

\videoLink{Section 4.2 day 1}{https://www.youtube.com/playlist?list=PLYHZK3b8UFw2_C72_BEthTH-k3sKuNAEb}
%\videoLink{Section 4.2 day 2}{https://www.youtube.com/playlist?list=PLYHZK3b8UFw1pcxMutFvlZct58eGZRjlC}

\noindent \textbf{Topics:}  unit circle, trigonometric functions, trigonometric identities, periodic\\

\noindent \textbf{Student Learning Outcomes:}
\begin{enumerate}
\item Students will be able to evaluate trigonometric functions using the unit circle.
\item Students will be able to determine domains of trigonometric functions.
\item Students will be able to use trigonometric identities.
\end{enumerate}

\hrule 

\bigskip

\subsection{Pythagorean Identities} ~

$$\sin^2(t)+\cos^2(t)=1 \quad \quad \tan^2(t)+1=\sec^2(t) \quad \quad 1+\cot^2(t)=\csc^2(t)$$
\\
\begin{enumerate}
\item Given that $\tan(t)=\frac{12}{5}$ for $\pi < t < \frac{3\pi}{2}$, use an appropriate identity to find the value of $\sec(t)$.\\[2in]

\item Given that $\csc(t)=\frac{5}{4}$ for $\frac{\pi}{2} < t < \pi$, use an appropriate identity to find the value of $\cot(t)$.\\[2in]

\item Given a real number $t$, express $\sin(t)$ in terms of $\cos(t)$.\\[1.5in]

\newpage

\subsection{Even, Odd, and Periodic Properties} ~

   \noindent\colorbox{blue!10}{%
   \parbox{\dimexpr\linewidth}%
   {%
     \textbf{Even and odd trigonometric functions.}

     The cosine and secant functions are even:
     \begin{eqnarray*}
       \begin{array}{rcl@{\hspace{4em}}rcl}
         \cos(-t) & = & \cos(t), & \sec(-t) & = & \sec(t).
       \end{array}
     \end{eqnarray*}

     The sine, cosecant, tangent, and cotangent functions are odd:
     \begin{eqnarray*}
       \begin{array}{rcl@{\hspace{4em}}rcl}
         \sin(-t) & = & \sin(t), & \csc(-t) & = & \csc(t),\\ [10pt]
         \tan(-t) & = & \tan(t), & \cot(-t) & = & \cot(t).
       \end{array}
     \end{eqnarray*}

   }
 }


\item Use the unit circle to find the value of $\displaystyle \cos \Bigg(\frac{2\pi}{3}\Bigg)$ and odd trig functions to find the value of $\displaystyle \cos \Bigg(-\frac{2\pi}{3}\Bigg)$.

\vfill


   \noindent\colorbox{blue!10}{%
   \parbox{\dimexpr\linewidth}%
   {%
     \textbf{Periodic properties of the sine and cosine functions.}
     \begin{eqnarray*}
       \begin{array}{rcl@{~~\textrm{and}~~}rcl}
         \sin(t+2\pi) & = & \sin(t), & \cos(t+2\pi) & = & \cos(t).
       \end{array}
     \end{eqnarray*}
     The sine and cosine functions are both periodic with a period of $2\pi$.
   }
 }
 \noindent\colorbox{blue!10}{%
   \parbox{\dimexpr\linewidth}%
   {%
     \textbf{Periodic properties of the tangent and cotangent functions.}
     \begin{eqnarray*}
       \begin{array}{rcl@{~~\textrm{and}~~}rcl}
         \tan(t+\pi) & = & \tan(t), & \cot(t+\pi) & = & \cot(t).
       \end{array}
     \end{eqnarray*}
     The tangent and cotangent functions are both periodic functions with a period of $\pi$.

     \textbf{Repetitive behaviour of the sine, cosine, tangent and cotangent functions.}
     For any integer $n$ and any real number $t$
     \begin{eqnarray*}
       \begin{array}{rcl@{\hspace{4em}}rcl}
         \sin(t+2\pi n) & = & \sin(t), & \cos(t+2\pi n) & = & \cos(t), \\
         \tan(t+\pi n) & = & \tan(t), & \cot(t+\pi n) & = & \cot(t).
       \end{array}
     \end{eqnarray*}

   }
 }

\newpage
\item Given $\displaystyle \sin\Bigg(\frac{\pi}{12}\Bigg)=\frac{\sqrt{6}-\sqrt{2}}{4}$, determine the value of $\displaystyle \sin \Bigg( \frac{49\pi}{12} \Bigg)$.

\vfill
\item Use properties of trigonometric functions to simplify $\tan(-3t)-\tan(-3t+\pi)$.
\vfill

\subsection{Approximate Trigonometric Functions on a Calculator} ~

\item Use a calculator to approximate the function values.  Round to 4 decimal places.
\begin{enumerate}
\item $\displaystyle \cos \Bigg( \frac{2\pi}{7} \Bigg)$\\
\item $\csc(0.92)$\\[.2in]
\end{enumerate}



\end{enumerate}

\noindent \textbf{Student Learning Outcomes Check}

\begin{enumerate}
\item Can you evaluate trigonometric functions using the unit circle?
\item Can you determine domains of trigonometric functions?
\item Are you able to use trigonometric identities?

\end{enumerate}

\noindent \textbf{If any of your answers were no, please ask about these topics in class.}


