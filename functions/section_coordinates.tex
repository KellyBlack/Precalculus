%=========================================================================
% Start of
%=========================================================================
\preClass{Coordinate Systems}

\begin{problem}
\item Make a sketch of a number line with zero at the center.  Mark
  the locations of -2, -2.5, 1.1, and 2.3 on your number line.
  The relative distances between the points should be consistent.
  \sideNote{Label the number line by putting an ``$x$'' to the right
    and an arrow indicating the positive direction.}

  \vfill

\item Make a sketch of a number line with zero at the center.  Mark
  the locations of -2 and 2.15 on your number line. What is the
  distance between the two points?
  (The relative distances between the points should be consistent.)

  \vfill

\item Make a sketch of a number line with zero at the center.  Mark
  the locations of -1.54 and 2.07 on your number line. What is the
  distance between the two points?
  (The relative distances between the points should be consistent.)

  \vfill

\end{problem}


\actTitle{Coordinate Systems}
\begin{problem}
\item A set of points is given below as a table. Each point is given
  on a line. The left side of the table has the $x$-coordinate of each
  point, and the right side has the $y$ coordinate. Plot each point on
  the axes given below. Next to each point indicate which quadrant the
  point is located.

  \begin{tabular}{l|ll}
    $x$ & $y$ & Quadrant \\ \hline
     2.0 &  1.5 & \\ [10pt]
     2.2 & -4.5 & \\ [10pt]
    -3.4 &  2.8 & \\ [10pt]
    -2.8 & -1.3 & \\ [10pt]
    -4.5 &  2.5 & \\ [10pt]
  \end{tabular}

  \begin{tikzpicture}[y=1cm, x=1cm,font=\sffamily]
    % bounds
    \def\lowX{-5.5}
    \pgfmathtruncatemacro\startX{round(0.5+\lowX)}
    \pgfmathsetmacro\nextXValue{int(\startX+1)}
    \def\highX{5.5}
    \def\lowY{-5.5}
    \def\highY{5.5}
    \pgfmathsetmacro\nextYValue{int(\lowY+1)}
    % ticks
    \draw[step = 1, gray, very thin,dashed,opacity=0.85] (\lowX, \lowY) grid ( \highX,\highY);
 	% axis
	\draw[thick,->] (\lowX,0) -- coordinate (x axis mid) (\highX,0) node[anchor = north west] {$x$};
    \draw[thick,->] (0,\lowY) -- coordinate (y axis mid) (0,\highY) node[anchor = south east] {$y$};
    \foreach \y in {-5,-4,...,-1,1,2,...,\highY} {
      \draw (1pt, \y) -- (-1pt, \y) node[yshift=-6,xshift=-1,anchor=east] {$\y$};
    }
    \foreach \x in {-5,-4,...,-1,1,2,...,\highX} {
      \draw (\x,1pt) -- (\x,-1pt) node[yshift=-5,xshift=-1,anchor=east] {$\x$};
    }
  \end{tikzpicture}

  \vfill

  \clearpage

\item Mark the points $P_1(-2.1,-4.4)$ and $P_2(4.5,1.2)$ on the coordinate plane
  below. Determine the distance between the two points.  Include a
  sketch of a right triangle whose hypotenuse represents the distance
  between the two points.

  \begin{tikzpicture}[y=1cm, x=1cm,font=\sffamily]
    % bounds
    \def\lowX{-5.5}
    \pgfmathtruncatemacro\startX{round(0.5+\lowX)}
    \pgfmathsetmacro\nextXValue{int(\startX+1)}
    \def\highX{5.5}
    \def\lowY{-5.5}
    \def\highY{5.5}
    \pgfmathsetmacro\nextYValue{int(\lowY+1)}
    % ticks
    \draw[step = 1, gray, very thin,dashed,opacity=0.85] (\lowX, \lowY) grid ( \highX,\highY);
 	% axis
	\draw[thick,->] (\lowX,0) -- coordinate (x axis mid) (\highX,0) node[anchor = north west] {$x$};
    \draw[thick,->] (0,\lowY) -- coordinate (y axis mid) (0,\highY) node[anchor = south east] {$y$};
    \foreach \y in {-5,-4,...,-1,1,2,...,\highY} {
      \draw (1pt, \y) -- (-1pt, \y) node[yshift=-6,xshift=-1,anchor=east] {$\y$};
    }
    \foreach \x in {-5,-4,...,-1,1,2,...,\highX} {
      \draw (\x,1pt) -- (\x,-1pt) node[yshift=-5,xshift=-1,anchor=east] {$\x$};
    }
  \end{tikzpicture}

  \vfill

  \clearpage

\item Mark the point $P_3(1.3,-2.4)$ on the coordinate plane below. Determine the
  points on the $x$-axis that are a distance of 3 units from $P_3$.

  \begin{tikzpicture}[y=1cm, x=1cm,font=\sffamily]
    % bounds
    \def\lowX{-5.5}
    \pgfmathtruncatemacro\startX{round(0.5+\lowX)}
    \pgfmathsetmacro\nextXValue{int(\startX+1)}
    \def\highX{5.5}
    \def\lowY{-5.5}
    \def\highY{5.5}
    \pgfmathsetmacro\nextYValue{int(\lowY+1)}
    % ticks
    \draw[step = 1, gray, very thin,dashed,opacity=0.85] (\lowX, \lowY) grid ( \highX,\highY);
 	% axis
	\draw[thick,->] (\lowX,0) -- coordinate (x axis mid) (\highX,0) node[anchor = north west] {$x$};
    \draw[thick,->] (0,\lowY) -- coordinate (y axis mid) (0,\highY) node[anchor = south east] {$y$};
    \foreach \y in {-5,-4,...,-1,1,2,...,\highY} {
      \draw (1pt, \y) -- (-1pt, \y) node[yshift=-6,xshift=-1,anchor=east] {$\y$};
    }
    \foreach \x in {-5,-4,...,-1,1,2,...,\highX} {
      \draw (\x,1pt) -- (\x,-1pt) node[yshift=-5,xshift=-1,anchor=east] {$\x$};
    }
  \end{tikzpicture}

  \vfill

  \begin{enumerate}
  \item Place points on the graphs near to where you think the points
    \textbf{may} be located.
  \item Label one of the points, $(x,y)$. 
  \item Write out the general distance formula.
  \item What do you know about the point? Can you simplify the formula?
  \item Solve for the unknown variable.
  \end{enumerate}

  \vfill


\clearpage

\item Mark the point $P_4(1,-2)$ on the coordinate plane below. Mark \textbf{all}
  of the points that are a distance of 2 units from $P_4$.
  \label{act1:q:drawCircle}

  \begin{tikzpicture}[y=1cm, x=1cm,font=\sffamily]
      % bounds
      \def\lowX{-5.5}
      \pgfmathtruncatemacro\startX{round(0.5+\lowX)}
      \pgfmathsetmacro\nextXValue{int(\startX+1)}
      \def\highX{5.5}
      \def\lowY{-5.5}
      \def\highY{5.5}
      \pgfmathsetmacro\nextYValue{int(\lowY+1)}
      % ticks
      \draw[step = 1, gray, very thin,dashed,opacity=0.85] (\lowX, \lowY) grid ( \highX,\highY);
   	% axis
  	\draw[thick,->] (\lowX,0) -- coordinate (x axis mid) (\highX,0) node[anchor = north west] {$x$};
      \draw[thick,->] (0,\lowY) -- coordinate (y axis mid) (0,\highY) node[anchor = south east] {$y$};
      \foreach \y in {-5,-4,...,-1,1,2,...,\highY} {
        \draw (1pt, \y) -- (-1pt, \y) node[yshift=-6,xshift=-1,anchor=east] {$\y$};
      }
      \foreach \x in {-5,-4,...,-1,1,2,...,\highX} {
        \draw (\x,1pt) -- (\x,-1pt) node[yshift=-5,xshift=-1,anchor=east] {$\x$};
      }
    \end{tikzpicture}

  \vfill

  What kind of figure did you draw?

  \vfill

\clearpage

\item Suppose a point, $P(x,y)$ is a distance of 2 units from the
  point $P_4(1,-2)$.
  \begin{subproblem}
  \item Use the distance formula to express the distance relationship
    between $P$ and $P_4$.
    \vfill
  \item Square both sides of the previous equation.
    \vfill
  \end{subproblem}

\item Suppose a point, $P(x,y)$ is a distance of $R$ units from the
  point $P_4(1,-2)$.
  \begin{subproblem}
  \item Use the distance formula to express the distance relationship
    between $P$ and $P_4$.
    \vfill
  \item Square both sides of the previous equation.
    \vfill

  \item How does this formula relate to the figure you drew in
    question \ref{act1:q:drawCircle}.
  \end{subproblem}

\end{problem}


\postClass

\begin{problem}
\item Briefly state two ideas from today's class.
  \begin{itemize}
  \item
  \item
  \end{itemize}
\item For each equation below determine the values of $x$ that satisfy
  the equation. Determine the exact answer and also determine an
  approximations to the answer using two decimal places.
  \begin{subproblem}
    \item $2x^2 + 5x - 3 = 0$
    \item $5x-1=8x+7$
    \item $3x - 1 = 2x^2 + 2x + 6$
    \item $x^3 = 2$
  \end{subproblem}
\item Draw a coordinate axis, and properly label the axes. Use the
  axes to make a sketch of the graph of the relationship $y+x=2$.
\item Draw a coordinate axis, and properly label the axes. Use the
  axes to make a sketch of the graph of the relationship $y^2+x=2$.
\item Make a sketch of a number line with zero at the center.
  Indicate the set of numbers that satisfy $x^2>2$.
\item Make a sketch of a number line with zero at the center.
  Indicate the set of numbers that satisfy $x>2.2$ and $x<5.4$.
\item Make a sketch of a number line with zero at the center.
  Indicate the set of numbers that satisfy $|x|>1.5$.
\end{problem}
