

\actTitle{Worksheet 4.4}


\noindent \textbf{Instructions:}  Work together in groups of  3 or 4 to complete the following problems.\\
\noindent \textbf{NOTE:  Do not use your unit circle to answer the following questions.  Only use reference triangles.}

Student goals:
\begin{itemize}
\item Determine subsets of the domain where sine/cosine are increasing
  and where they are decreasing.
\item Determine whether a trigonometric function will increase or
  decrease given a specific angle.
\item Determine the reference angle of a given angle in any of the
  four quadrants.
\end{itemize}


\begin{enumerate}

\item Let $P(-7,\frac{\sqrt{3}}{4})$ be a point on the terminal side of $\theta$.  Find each of the six trig functions of $\theta$.\begin{enumerate}
\begin{multicols}{2}
\item $\sin(\theta)=$\\[1em]
\item $\cos(\theta)=$\\[1em]
\item $\tan(\theta)=$\\[1em]
\columnbreak
\item $\csc(\theta)=$\\[1em]
\item $\sec(\theta)=$\\[1em]
\item $\cot(\theta)=$\\[1em]
\end{multicols}
\end{enumerate}
\vfill

\item For each of the following questions $\theta$ is in standard
  position and  $\theta=2\pi/3$.
\begin{enumerate}
\item  What quadrant is $\theta$ in?
  \vfill
\item Determine the reference angle for $\theta$.
  \vfill
\item Determine exact values of each of the following and determine if
  the function is increasing, decreasing, or neither at $\theta$.
  % quadrant ii 
  \begin{enumerate}
  \item $\sin\left(\frac{2\pi}{3}\right)$
    \vfill
  \item $\cos\left(\frac{2\pi}{3}\right)$
    \vfill
  \item $\tan\left(\frac{2\pi}{3}\right)$
    \vfill
  \end{enumerate}

\end{enumerate}

\clearpage
\item For each of the following expressions, $t=\frac{7\pi}{2}$. For
  each expression determine the value and determine if the function is
  increasing, decreasing, or neither at $t$.
\begin{enumerate}
\item $\sin(t)$ \vfill
\item $\tan(t)$ \vfill
\item $\sec(t)$ \vfill
\end{enumerate}


\item In each question below  $\theta$ is an angle in standard
  position and  $\theta=\frac{16\pi}{3}$. 
  \begin{enumerate}
  \item What quadrant is $\theta$ in?  Determine the reference angle
    for $\theta$.
    \vfill
    \vfill

  \item Determine the exact values of $\sin(\theta)$, $\sec(\theta)$,
    and $\cot(\theta)$. For each value indicate if the function is
    increase, decreasing, or neither for this value of $\theta$.
    \vfill \vfill \vfill \vfill
\end{enumerate}
%\\[3in]

\clearpage
%QII 
\item \begin{enumerate}
  \item If $\theta$ is an angle in standard position, determine the
    quadrant corresponding to $\theta=-210^\circ$. Then determine the
    reference angle for $\theta=-210^\circ$.

    \vfill

  \item Determine the exact values of $\cos(\theta),$ $\csc(\theta)$,
    and $\tan(\theta)$ for $\theta=-210^\circ$.

    \vfill
    
\end{enumerate}








\item Suppose $\theta$ is an angle in the third quadrant with
  reference angle $\theta_R$ satisfying $\cos(\theta_R)=\frac{5}{13}$
  and $\sin(\theta_R)=\frac{12}{13}$.  Determine the exact values of
  $\cos(\theta)$ and $\csc(\theta)$.

  \vfill
  \vfill
  \vfill

\item An angle is in the second quadrant, and its reference angle is
  $\theta_R=\frac{2\pi}{9}$. Determine the radian measure of the
  angle.

  \vfill

\end{enumerate}



\hwTitle{Section 4.4}

\begin{enumerate}
\item  Determine exact values of the following.
  \begin{enumerate}
  \item $\sec\left(\frac{2\pi}{3}\right)$
  \item $\csc\left(\frac{2\pi}{3}\right)$
  \item $\cot\left(\frac{2\pi}{3}\right)$
  \end{enumerate}
\item Determine the following.
  \begin{enumerate}
  \item $\cos\left(\frac{7\pi}{2}\right)$
  \item $\csc\left(\frac{7\pi}{2}\right)$
  \item $\cot\left(\frac{7\pi}{2}\right)$
  \end{enumerate}
\item Determine the exact value of each of the following. %same ref
  \begin{enumerate}
  \item $\sin\left(\frac{7\pi}{6}\right)$ 
  \item $\sin\left(\frac{11\pi}{6}\right)$
  \item $\sin\left(\frac{5\pi}{6}\right)$ 
  \end{enumerate}
\item Find the value of each expression.
  \begin{enumerate}
  \item $\sin\left(30^\circ\right)\cdot \cos\left(150^\circ\right)\cdot \sec\left(60^\circ\right)\cdot \csc\left(120^\circ\right)$
  \item $\cos^2\left(\frac{5\pi}{4}\right)-\sin^2\left(\frac{2\pi}{3}\right)$
  \item $\sin^2\left(\frac{11\pi}{6}\right)+\cos^2\left(\frac{4\pi}{3}\right)$
  \item $\displaystyle \frac{2\tan\left(\frac{11\pi}{6}\right)}{1-\tan^2\left(\frac{11\pi}{6}\right)}$
  \end{enumerate}
\item An angle is in the third quadrant, and its reference angle is
  $\theta_R=\frac{5\pi}{12}$. Determine the radian measure of the
  angle.
\item Determine the reference angle for $\theta=5.7$ radians.
\item A function, $\textrm{Ref}(\theta)$, is defined so that it
  returns the reference angle of the given angle. Make a sketch of the
  function for $0\leq\theta < 2\pi$. Express the function as piecewise
  defined function. Is the function periodic? Is it even, odd, or
  neither?


\end{enumerate}
