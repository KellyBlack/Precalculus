%=========================================================================
% Start of activity on exponential equations.
%=========================================================================
\preClass{Exponential and Logarithmic Equations}

\begin{problem}
\item Solve for the variable $x$ in each expression below.
  \begin{subproblem}
  \item $\ln(x-1)=5$
    \vfill
  \item $e^{3x+1} = 5$
    \vfill
  \item $\log_4(x+1) = 2$
    \vfill
  \item $3^{-x+1} = 4$
    \vfill
  \end{subproblem}
\end{problem}


\actTitle{Exponential and Logarithmic Equations}
\begin{problem}
\item Determine the value of $x$ or $t$ as appropriate in each of the
  following expressions. (Justify the steps that you make.)
  \begin{subproblem}
  \item $10 = 20\left( 1 - e^{-3t} \right)$

    \vfill

  \item $\frac{1}{20} =  \frac{1}{3\sqrt{2\pi}} e^{-x^2/6}$
    \vfill
  \item $xe^{-x} + e^{-x} = 0$
    \vfill
    \clearpage
  \item $e^{t} + 2 - e^{-t} = 0$
    \vfill
  \item $e^{-x^2} = 3\cdot 4^{2x-1}$
    \vfill
  \end{subproblem}

  \clearpage

\item The goal is to determine the value of $x$ that satisfies the
  equation
  \begin{eqnarray*}
    \log_4(x) = 3 + \log_8(x).
  \end{eqnarray*}
  \vspace{-2em}
  \begin{subproblem}
  \item First focus on the left hand side of the equation.
    \begin{subsubproblem}
    \item Define a new variable, $u$, by setting $u$ equal to the left
      hand side of the equation.
      \vspace{1em}
    \item Exponentiate both sides of the equation using a clever
      choice for the base so that there will not be any logarithms in
      the equation. Simplify the result.
      \vspace{2em}
    \item Take the natural logarithm of both sides and solve for $u$.
      \vspace{2em}
    \end{subsubproblem}
  \item Now focus on the right hand side of the equation.
    \begin{subsubproblem}
    \item Define a new variable, $v$, by setting $v$ equal to the
      logarithm in the right hand side of the equation.  
      \vspace{1em}
    \item Exponentiate both sides of the equation using a clever
      choice for the base so that there will not be any logarithms in
      the equation. Simplify the result.
      \vspace{2em}
    \item Take the natural logarithm of both sides and solve for $v$.
      \vspace{2em}
    \end{subsubproblem}
  \item Substitute your value for $u$ into the left hand side and the
    value for $v$ into the right hand side.
    \vspace{1em}
  \item Solve the new equation for $x$.
    \vfill
  \end{subproblem}

\clearpage

\item A population of bacteria doubles every four hours. Determine a
  function that gives the number of individuals in the population for
  a given time.
  \begin{subproblem}
  \item Express the given information in function form. (Assume the
    population is given by $P(t)$.)
    \vfill
  \item The assumption is that the population is modeled as an
    exponential function. Determine the general form of the
    function. (Why would an exponential function be used in this
    instance?)  
    \vfill
  \item Using the function above express the given information as a
    mathematical equation.
    \vfill
  \item Which variable can you solve for? Identify the variable and
    solve for it.
    \vfill
  \item What is the general form for the function that models the
    population?
    \vspace{1em}
  \end{subproblem}

\clearpage

\item Carbon 14 has a half life of 5,730 years. How long will it take
  for a sample to decay to two thirds of its original value?
  \begin{subproblem}
  \item Express the given information in function form. (Assume the
    amount is given by $C(t)$.)
    \vfill
  \item The assumption is that the amount is modeled as an
    exponential function. Determine the general form of the function. (Why
    would an exponential function be used in this instance?)
    \vfill
  \item Using the function above express the given information as a
    mathematical equation.
    \vfill
  \item Which variable can you solve for? Identify the variable and
    solve for it.
    \vfill
  \item What is the general form for the function that models the
    amount of material?
    \vspace{1em}
  \end{subproblem}

\clearpage



\end{problem}

\postClass

\begin{problem}
\item Briefly state two ideas from today's class.
  \begin{itemize}
  \item
  \item
  \end{itemize}
\item Determine the value of $x$ in each expression below.
  \begin{subproblem}
    \item $3^x = 4^{x+1}$
    \item $3^x = 9\cdot 4^{x+1}$
    \item $2^x = 9^{5x-1}\cdot 4^{x+1}$
    \item $14^{3x+4} = 10^{9x/2}\cdot 20^{6x-1}$
    \item $3^{x^2} = 3 \cdot 8^x$
  \end{subproblem}
\end{problem}


%%% Local Variables:
%%% mode: latex
%%% TeX-master: "../labManual"
%%% End:
