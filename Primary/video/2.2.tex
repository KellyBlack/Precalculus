\documentclass[11pt]{article}
\usepackage{amsmath, amssymb, array}
\pagestyle{plain}

\textwidth=6.5in
\hoffset-1in
\textheight=9in
\voffset-1in

\usepackage{enumitem}
\usepackage{pgfplots}
\usepackage{graphicx}
\usepackage{lipsum}
\usepackage{stfloats}
\usepackage{multicol}
\setlength{\columnsep}{1cm}

\newcommand{\boxcolor}{gray!30}
\usepackage{mdframed}
\newenvironment{boxme}{\begin{mdframed}[backgroundcolor=\boxcolor,linewidth=0pt,nobreak=true]}{\end{mdframed}}
\newenvironment{boxthm}{\begin{mdframed}[backgroundcolor=\boxcolor,nobreak=true]}{\end{mdframed}}
\newenvironment{boxdef}{\begin{mdframed}[backgroundcolor=\boxcolor,linewidth=0pt,nobreak=true]}{\end{mdframed}}




\begin{document}


\noindent MATH 1113   \hfill 2.2 - Introduction to Polynomial Functions\\



\noindent \textbf{Topics:}  polynomials, leading term test, intermediate value theorem\\

\noindent \textbf{Student Learning Outcomes:}
\begin{enumerate}
\item Students will be able to determine the end behavior of a polynomial function.
\item Students will be able to identify zeros and multiplicities of zeros.
\item Students will be able apply the Intermediate Value Theorem.

\end{enumerate}

\hrule 
\vspace{5mm}

\section{End Behavior of Polynomial Functions}

\begin{boxthm}
{\bf Definition of a Polynomial Function}
Let $n$ be a positive, whole number and $a_n,a_{n-1}, a_{n-2}...,a_1,a_0$ be real numbers, where $a_n\neq 0$.  Then a function defined by $$f(x)=a_nx^n+a_{n-1}x^{n-1}+a_{n-2}x^{n-2}+...+a_1x+a_0$$ is called a \textbf{polynomial function of degree $n$}.

\end{boxthm}


Examples:
$$f(x)=5x^2+8x^4 \quad \quad \quad g(x)=\frac{x^2+3}{x^2} \quad \quad \quad  h(x)=4x^5-3x^4+2x^2 \quad \quad \quad k(x)=4\sqrt{x}-\frac{3}{x} +x^2$$

\vfill
We have already studied several special cases of polynomial functions:

$$f(x)=2 \quad \quad \quad g(x)=3x+1 \quad \quad \quad  h(x)=4x^2+7x-1$$


\vfill

Polynomial functions of degree 2 or higher have graphs that are smooth and continuous.\\[.5in]


\newpage

\includegraphics[scale=.95]{poly1}
\begin{enumerate}

\item Use the leading term test to determine the function's end behavior.


\begin{enumerate}
\item $f(x)=-7x^5+2x^3+7x+5$
\vfill

\item $f(x)=\frac{1}{4}x(2x-3)^3(x+4)^2$
\vfill
\vfill
\end{enumerate}

\section{Identify Zeros and Multiplicities of Zeros}
\begin{boxthm}
{\bf Multiplicities and $x$-Intercepts}
If $f$ is a polynomial function, then the values of $x$ for which $f(x)=0$ are called the \textbf{zeros (roots or solutions)} of $f(x)$.  Each real root of the polynomial equations appears as an $x$-intercept of the graph of the polynomial function.\\
\\
If $r$ is a zero of \textbf{even multiplicity}, then the graph \textbf{touches} the $x$-axis and \textbf{turns around} at $r$.  If $r$ is a zero of \textbf{odd multiplicity}, then the graph \textbf{crosses} the $x$-axis at $r$.  Regardless of whether the multiplicity of a zero is even or odd graphs tend to flatten out near zeros with multiplicity greater than one.
\end{boxthm}
\newpage



\item Find the zeros for each polynomial function and give the multiplicity for each.  State whether the graph crosses the $x$-axis, or touches the $x$-axis and turns around, at each zero.

\begin{enumerate}
\item $f(x)=4(x+3)(x-7)^2$
\vfill
\item $f(x)=x^3-6x^2+9x$
\vfill
\end{enumerate}


\newpage

\section{Apply the Intermediate Value Theorem}
\begin{boxthm}
{\bf Intermediate Value Theorem}
Let $f$ be a polynomial function.  For $a<b$, if $f(a)$ and $f(b)$ have opposite signs, then $f$ has at least one zero on the interval $[a,b]$.
\end{boxthm}

\item Use the Intermediate Value Theorem to determine if $f(x)=4x^4-8x^2+2$ has a real zero on the interval $[-1,0]$.
\vfill


\item Sketch a graph of $f(x)=x^3-9x$
\vfill
\vfill
\vfill



\end{enumerate}

\noindent \textbf{Student Learning Outcomes Check}

\begin{enumerate}
\item Are you able to determine the end behavior of a polynomial function?
\item Can you identify zeros and multiplicities of zeros?
\item Are you able apply the Intermediate Value Theorem?
\end{enumerate}

\noindent \textbf{If any of your answers were no, please ask about these topics in class.}













\end{document}