%=========================================================================
% Start of
%=========================================================================
\preClass{Parabolas}

\begin{problem}
\item A function is defined to be
  \begin{eqnarray*}
    f(x) & = & x^2.
  \end{eqnarray*}
  \begin{subproblem}
  \item Make a sketch of the function on the axes below.
  \item Make a sketch of the following new functions on the graph as
    well with clear annotations:
    \begin{eqnarray*}
      g(x) & = & f(3x), \\
      h(x) & = & f(x)+2, \\
      p(x) & = & f(x+2), \\
      q(x) & = & -f(x), \\
      r(x) & = & -f(x-2).
    \end{eqnarray*}

    \begin{tikzpicture}[y=1.1cm, x=1.1cm,font=\sffamily]
        % bounds
        \def\lowX{-5.5}
        \pgfmathtruncatemacro\startX{round(0.5+\lowX)}
        \pgfmathsetmacro\nextXValue{int(\startX+1)}
        \def\highX{5.5}
        \def\lowY{-5.5}
        \def\highY{5.5}
        \pgfmathsetmacro\nextYValue{int(\lowY+1)}
        % ticks
        \draw[step = 1, gray, very thin,dashed,opacity=0.85] (\lowX, \lowY) grid ( \highX,\highY);
      % axis
      \draw[thick,->] (\lowX,0) -- coordinate (x axis mid) (\highX,0) node[anchor = north west] {$x$};
        \draw[thick,->] (0,\lowY) -- coordinate (y axis mid) (0,\highY) node[anchor = north east] {$y$};
        \foreach \y in {-5,-4,...,-1,1,2,...,\highY} {
          \draw (1pt, \y) -- (-1pt, \y) node[yshift=-6,xshift=-1,anchor=east] {$\y$};
        }
        \foreach \x in {-5,-4,...,-1,1,2,...,\highX} {
          \draw (\x,1pt) -- (\x,-1pt) node[yshift=-5,xshift=-1,anchor=east] {$\x$};
        }
        \draw (0,5.5) node [anchor=south] {Comparing Shifted Functions};
      \end{tikzpicture}

  \end{subproblem}
\end{problem}


\actTitle{Quadratic Functions}
\begin{problem}

\item A traffic engineer observes the movement of traffic on a busy
  street. She estimates that when the traffic density is 0 cars/meter
  that the rate of flow is 0 cars/minute. She notices that when the
  density is high, roughly 0.1 cars/meter or more, there is gridlock,
  and the rate of flow is 0 cars/minute. Finally, she notices that
  when there is 0.05 cars per meter the rate of flow is at a maximum
  of 20 cars per minute.
  \begin{subproblem}
  \item Assuming that the rate of flow is 0 if the density is greater
    than 0.1 make a rough sketch of the rate of flow as a function of
    the density below.
    \vfill
  \item Assuming that the rate of flow is a quadratic function of the
    traffic density when the density is between 0 and 0.1 cars/meter
    inclusive, determine the formula for the traffic density.
    \vfill
  \end{subproblem}

  \clearpage

\item Maximize the product of two positive numbers whose sum is five.
\begin{subproblem}
  \item Make a rough sketch of the situation.
    \sideNote{The product of two numbers is the area of a rectangle!}
    \vfill
  \item Within the sketch above, identify and label the variables
    that you will use.
  \item Determine the formula for the function to be maximized.
    \sideNote{This is called the "Cost Function." It should have two
      variables.}  
    \vfill
  \item Determine any other relevant relationships between your
    variables that must be true in all circumstances.  \sideNote{This
      is called the "Constraint." It should have two variables.}
    \vfill
  \item How will you solve this problem?
    \vfill
  \item Determine the values of the two numbers.
    \sideNote{Check your work and make sure that you maximized the
      profit and did not minimize it.}
    \vfill
    \vfill
\end{subproblem}

  \clearpage

\item A developer has a square plot of land that is 100 meters by 100
  meters, and the land is next to a river. The land will be divided
  into two parts. One part will be formed by cutting out a rectangle
  in one corner of the large plot, and its width will be along the
  river. The height of the rectangle will be ten meters shorter than
  its width. 

  It is estimated that this new rectangular plot of land will sell for
  \$2.00 per square meter plus 100\$/meter times the width. The
  remaining land will be sold for \$4.00 per square meter.  Determine
  the dimensions of the rectangular plot that will result in the
  highest profit.

    \begin{subproblem}
      \item Make a sketch of the situation.
        \sideNote{Label important aspects of the sketch.}
        \vfill
      \item Within the sketch above, identify and label the variables
        that you will use.
      \item Determine the total selling price for the land.
        \sideNote{This is called the "Cost Function."}
        
        \vfill

        \clearpage

      \item Determine any other relevant relationships between your
        variables that must be true in all circumstances.
        \sideNote{This is called the "Constraint."}  \vfill
      \item How will you solve this problem?
        \vspace{2em}
      \item Determine the dimensions of the rectangular plot that will
        result in the highest profit.
        \sideNote{Check your work and make sure that you maximized the
          profit and did not minimize it.}
        \vfill
        \vfill
    \end{subproblem}

\end{problem}

\postClass

\begin{problem}
\item Briefly state two ideas from today's class.
  \begin{itemize}
  \item
  \item
  \end{itemize}

\item Determine the vertex of the following parabolas.
  \begin{subproblem}
  \item $y=-x^2+10x-5$
  \item $y=2x^2-10x+18$
  \item $y=3x^2+4+1$
  \item $y=5x^2+2$
  \item $y=5x^2+x+2$
  \end{subproblem}

\item Two super hero capes will be sewn as part of a
  demonstration. One cape will be in the shape of a square. The other
  cape will be in the shape of a triangle, and the height is the same
  length as its base. They will be displayed together, and the sum of
  the two lengths must be 6 feet. What are the dimensions of the capes
  that will minimize the total area of the capes?
\end{problem}


%%% Local Variables:
%%% mode: latex
%%% TeX-master: "../labManual"
%%% End:
