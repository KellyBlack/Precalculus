%=========================================================================
% Start of
%=========================================================================
\preClass{Inverse Trigonometric Functions}

\begin{problem}
\item Use the axes below to make a sketch of the graph of $\sin(x)$.

  \hspace*{-3.5em}
  \begin{tikzpicture}[y=1.2cm, x=1.2cm,font=\sffamily]
      % bounds
      \def\lowX{-6}
      \pgfmathtruncatemacro\startX{round(0.5+\lowX)}
      \pgfmathsetmacro\nextXValue{int(\startX+1)}
      \def\highX{6}
      \def\lowY{-2.1}
      \def\highY{2.1}
      \pgfmathsetmacro\nextYValue{int(\lowY+1)}
      % ticks
      \draw[step = 1, gray, very thin,dashed,opacity=0.85] (\lowX, \lowY) grid ( \highX,\highY);
      % axis
      \draw[thick,->] (\lowX,0) -- coordinate (x axis mid) (\highX,0) node[anchor = south west] {$x$};
      \draw[thick,->] (0,\lowY) -- coordinate (y axis mid) (0,\highY) node[xshift=-10,anchor = north east] {Sine};

      \foreach \y in {-2,-1,1,2} {
        \draw (-1pt,\y) -- (1pt,\y) node[anchor=east] {$\y$};
      }

      \draw ( 1,1pt) -- ( 1,-1pt) node[yshift=-1,xshift=0,anchor=north east] {$\frac{ \pi}{2}$};
      \draw (-1,1pt) -- (-1,-1pt) node[yshift=-1,xshift=0,anchor=north east] {$\frac{-\pi}{2}$};
      \foreach \x in {-5,-3,3,5} {
        \draw (\x,1pt) -- (\x,-1pt) node[yshift=-8,xshift=1,anchor=east] {$\frac{\x\pi}{2}$};
      }
      \draw ( 2,1pt) -- ( 2,-1pt) node[yshift=-1,xshift=-1,anchor=north east] {$\pi$};
      \draw (-2,1pt) -- (-2,-1pt) node[yshift=-1,xshift=-1,anchor=north east] {$-\pi$};
      \foreach \x in {-3,-2,2,3} {
        \draw (2*\x,1pt) -- (2*\x,-1pt) node[yshift=-7,xshift=3,anchor=east] {$\x\pi$};
      }
      \draw (0,2.2) node [anchor=south] {Sine Function};
    \end{tikzpicture}

  \begin{subproblem}
    \item Is the sine function one-to-one? (Justify your answer!)
    \vspace{3em}
    \item What is the domain for the sine function?
    \vspace{1em}
    \item Define a small part of the domain for which the sine
    function is one-to-one. (There is not a unique answer - just find
    one small part.)  Make a sketch of the sine function restricted to
    your domain and explain why it is now one-to-one.

    \vfill

  \end{subproblem}

\end{problem}


\actTitle{Inverse Trigonometric Functions}
\begin{problem}
\item Use your calculator to evaluate the following
  expressions. Explain why it gives your results.
  \begin{subproblem}
    \item $\arcsin\left(\sin\left(\frac{\pi}{4}\right)\right)$
      \vfill
    \item $\arcsin\left(\sin\left(\frac{3\pi}{4}\right)\right)$
      \vfill
  \end{subproblem}

  \clearpage

  \item For each equation below determine all possible values of $x$
    that satisfy the equations.
    \begin{subproblem}
      \item ${\displaystyle \sin(3x-1)=\frac{\sqrt{3}}{2}}$
        \vfill
      \item ${\displaystyle \ln(\cos(x+1)+2)=0.6}$
        \vfill
    \end{subproblem}

  \clearpage

\item The commander of a radar station just received word that a super secret spy plane
  will fly directly over it. The plane is expected to have an elevation of
  two miles, and the distance between the station and a point on the ground
  directly below the plane is roughly ninety miles away from the station's antenna.
  \begin{subproblem}
  \item What should the angle of elevation of the antenna be to point directly
    at the plane.
    \vfill
  \item Due to a mix up it is estimated that the orders are thirty minutes late.
    If the plane is traveling level flight at a constant speed of
    five-hundred miles per hour determine the angle of elevation.
    \vfill
  \item Determine the necessary angle of elevation if the orders were actually
    thirty-five minutes late.
    \vfill
  \end{subproblem}

  \vfill

  \clearpage

\item Determine the value of
  \begin{eqnarray*}
    \tan\left(\arcsin\left(\frac{1}{3}\right)\right)
  \end{eqnarray*}

  \begin{subproblem}
  \item Let $\theta$ be the angle that satisfies ${\displaystyle \sin\left(\theta\right)=\frac{1}{3}}$
    How does this definition relate to the expression above? Rewrite the expression in terms of the angle $\theta$.
    \vspace{3em}
  \item Make a sketch of a right triangle. Mark the appropriate angle
    and sides given that ${\displaystyle \sin\left(\theta\right)=\frac{1}{3}}$.
    \vfill
  \item Determine the value of the expression above based on your diagram.
    \vfill
    \vfill
  \end{subproblem}

  \clearpage

\item Determine the value of
  \begin{eqnarray*}
    \tan\left(\mathrm{arcsec}\left(\frac{4}{3}\right)\right)
  \end{eqnarray*}
  without using a calculator.

  \begin{subproblem}
  \item Make a sketch of a right triangle. Mark the appropriate angle
    and sides.
    \vfill
  \item Determine the value based on your diagram.
    \vfill
    \vfill
  \end{subproblem}

\end{problem}

\postClass

\begin{problem}
\item Briefly state two ideas from today's class.
  \begin{itemize}
  \item
  \item
  \end{itemize}
\item
  \begin{subproblem}
    \item
  \end{subproblem}
\end{problem}


%%% Local Variables:
%%% mode: latex
%%% TeX-master: "../labManual"
%%% End:
