%=========================================================================
% Start of 
%=========================================================================
\preClass{Introduction to Exponential Functions}

\begin{problem}
\item Carbon-15 has a half life of about 2.5 seconds. If an object has
  2 grams of carbon-15 in it now, then in 2.5 seconds it will only
  have 1 gram due to its decay. An object has $8.0\times 10^-{6}$
  grams of carbon-15, and it is placed in a sealed
  container. Determine how much carbon-15 is contained in the object
  at the following times:
  \begin{subproblem}
  \item After 2.5 seconds.
    \vfill
  \item After 5.0 seconds.
    \vfill
  \item After 7.5 seconds.
    \vfill
  \item After 10.0 seconds.
    \vfill
  \end{subproblem}
\end{problem}


\actTitle{Introduction to Exponential Functions}
\begin{problem}
\item A species of bacteria is able to divide every three hours. That
  is every three hours each bacteria splits into two new
  individuals. A colony starts with 10,000 individuals. Determine the
  number of bacteria at the following times''
  \begin{subproblem}
  \item After 3 hours.
    \vfill
  \item After 6 hours.
    \vfill
  \item After 9 hours.
    \vfill
  \item After $n\times 3$ hours where $n$ is an integer greater than zero.
    \vfill
  \item How many bacteria were in the colony 3 hours before the start
    of the experiment?
    \vfill
  \end{subproblem}

  \clearpage

\item A species of bacteria is able to divide every three hours. That
  is every three hours each bacteria splits into two new
  individuals. After each division, only 75\% of the remaining
  bacteria survive. A colony starts with 10,000 individuals. Determine
  the number of bacteria at the following times:
  \begin{subproblem}
  \item After 3 hours.
    \vfill
  \item After 6 hours.
    \vfill
  \item After 9 hours.
    \vfill
  \item After $n\times 3$ hours where $n$ is an integer greater than zero.
    \vfill
  \item How many bacteria were in the colony 3 hours before the start
    of the experiment?
    \vfill
  \end{subproblem}

  \clearpage

\item A bank offers a savings account in which the interest is
  compounded 1.5\% annually, and the interest is accrued each
  month. If a person places \$1,000 in an account how much money is in
  the account after $n$ months?
  \textit{Determine the amount of money in the account after the
    first, second, and third months. Do not simplify your results, and
  try to find the pattern.}

  \vfill

  \clearpage

\item Simplify each expression below.
  \begin{subproblem}
  \item $\frac{3^5\cdot 3^2}{3^4}$
    \vfill
  \item $\frac{2^8}{2^5}$
    \vfill
  \item $\frac{4^2\cdot 2^2}{4^3}$
    \vfill
  \item $5^9\cdot 5^2\cdot 5^{-7}$
    \vfill
  \item $\left(\frac{1}{2}\right)^5 \cdot 2^9 \cdot 2^{-3}$
    \vfill
  \end{subproblem}

\end{problem}

\postClass

\begin{problem}
\item Briefly state two ideas from today's class.
  \begin{itemize}
  \item 
  \item 
  \end{itemize}
\item A bank offers 1.5\% annual interest compounded weekly (assume 52
  weeks in a year). You will deposit \$5,000 into the account. How
  much money will be in the account at any time?
\item A bank offers 1.5\% annual interest compounded monthly. You will
  deposit some money into an account and wish to have \$25,000 after
  two years. How much money should you deposit?
\item A compound is created that decays over time. It takes four years
  until half of the compound decays in a sample. You wish to store the
  compound for 5 years. How much should you store if you wish to get 4
  kg after the storage period?
\end{problem}


\propertiesTitle{Properties of Exponentials}

Exponential functions satisfy the algebraic properties given below. In
each example it is assumed that $a$, $b$, and $c$ are constants, and
$a>0$. 
\begin{eqnarray}
  a^b \cdot a^c & = & a^{b+c}   \\ [10pt]
  \frac{a^b}{a^c} & = & a^{b-c} \\  [10pt]
  \left( a^b \right)^c & = & a^{b\cdot c}
\end{eqnarray}

Also, $e$ is a constant number, and we define the number $e$ to be 
\begin{eqnarray*}
  e & \approx & 2.718.
\end{eqnarray*}

It is common to use the number $e$ as the base for exponentials. The
number $e$ plays the same role as the constant $a$ in the equations
above:
\begin{eqnarray}
  e^b \cdot e^c & = & e^{b+c}   \\ [10pt]
  \frac{e^b}{e^c} & = & e^{b-c} \\  [10pt]
  \left( e^b \right)^c & = & e^{b\cdot c}
\end{eqnarray}


%%% Local Variables:
%%% mode: latex
%%% TeX-master: "../labManual"
%%% End:

