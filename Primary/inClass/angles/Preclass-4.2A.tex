\preClass{Coordinate Systems}


\videoLink{Section 4.2 day 1}{https://www.youtube.com/playlist?list=PLYHZK3b8UFw2_C72_BEthTH-k3sKuNAEb}

\begin{enumerate}
\item  Suppose that the real number $t$ corresponds to the point $\displaystyle P\Big(-\frac{\sqrt{13}}{4},\frac{\sqrt{3}}{4}\Big)$ on the unit circle.  Evaluate the six trigonometric functions at $t$.
\begin{enumerate}

\item $\sin(t)=$\vfill
\item $\cos(t)=$\vfill
\item $\tan(t)=$\vfill
\item $\csc(t)=$\vfill
\item $\sec(t)=$\vfill
\item $\cot(t)=$\vfill
\end{enumerate}

\item What is the radius of the unit circle?
  \vfill

\item Suppose a point $(x,y)$ is on the unit circle. Determine the
  distance from the point to the origin using the distance formula.
  \sideNote{You should have an equation that has both $x$ and $y$ in
    it.}
    \vfill


\clearpage


\item Use the coordinates on the unit circle to find the value of each trig function at the indicated real number.
\begin{enumerate}
\item $\displaystyle \sin\Big(\frac{4\pi}{3}\Big)=$\vfill
\item $\displaystyle \csc\Big(\frac{4\pi}{3}\Big)=$\vfill
\end{enumerate}

\item  Evaluate the trig functions at the indicated real number.
\begin{enumerate}
\item $\displaystyle \cos\Big(-\frac{\pi}{6}\Big)=$\vfill
\item $\displaystyle \tan\Big(-\frac{\pi}{6}\Big)=$\vfill
\end{enumerate}




\end{enumerate}



