%=========================================================================
% Start of
%=========================================================================
\preClass{Angle Measurement}

\begin{problem}
\item A circle has a radius of $r$ meters. Answer each of the
  following questions:
  \begin{subproblem}
  \item Determine the arc lengths described below.
    \begin{subsubproblem}
    \item What is the circumference of the whole circle?
      \vfill
    \item What is the arc length around half of the circle?
      \vfill
    \item What is the arc length around one-fourth of the circle?
      \vfill
    \end{subsubproblem}
  \item Determine the areas of the circle described below.
    \begin{subsubproblem}
    \item What is the area of the whole circle?
      \vfill
    \item What is the area of half of the circle?
      \vfill
    \item What is the area of one-fourth of the circle?
      \vfill
    \end{subsubproblem}
  \item How many degrees are there in a full circle? Where did this
    number come from?
    \vfill
  \item Make a sketch of a circle centered at the origin. Include the
    $x$ and $y$-axes. Add the ray from the origin that forms a 45
    degree angle with the positive $x$ axis.
    \sideNote{Label the axes and annotate your plot.}
    \vfill
    \vfill
  \end{subproblem}
\end{problem}


\actTitle{Angle Measurement}
\begin{problem}
\item For each of the angles below, print the angles in order from
  smallest to largest. In each picture the solid line is the initial
  side, and the dotted line is the terminal side.

    \vspace{1em}

  \begin{tikzpicture}[y=3cm, x=3cm,font=\sffamily]
    % Rays
    \draw[black,thick,->] (0, 0) -- (1,0);
    \draw[black,thick,dotted,->] (0, 0) -- (140:1);
    \draw[black,thin] (0.2,0) arc (0:140:0.2);
    \node at (70:0.32) {$\alpha$};

    \begin{scope}[shift={(0,-1.0)}]
      % Rays
      \draw[black,thick,->] (0, 0) -- (1,0);
      \draw[black,thick,dotted,->] (0, 0) -- (20:1);
      \draw[black,thin] (0.2,0) arc (0:20:0.2);
      \node at (8:0.32) {$\beta$};
    \end{scope}

    \begin{scope}[shift={(0,-2.2)}]
      % Rays
      \draw[black,thick,->] (0, 0) -- (1,0);
      \draw[black,thick,dotted,->] (0, 0) -- (70:1);
      \draw[black,thin] (0.2,0) arc (0:70:0.2);
      \node at (35:0.32) {$\gamma$};
    \end{scope}

    \begin{scope}[shift={(0,-3.0)}]
      % Rays
      \draw[black,thick,->] (0, 0) -- (1,0);
      \draw[black,thick,dotted,->] (0, 0) -- (-100:1);
      \draw[black,thin] (0.2,0) arc (0:-100:0.2);
      \node at (-50:0.32) {$\delta$};
    \end{scope}

    \draw[black,thick] (2,1) -- (2,-4);
    \node at (2.4,0.1)  {Smallest:};
    \node at (2.4,-1.45) {Largest:};
    \foreach \y in {0,0.5,1.0,1.5} { \draw (2.7,-\y) -- (3.7,-\y); }

  \end{tikzpicture}

  

  \clearpage
\item Use the diagram below to plot the points indicated in the
  descriptions below.

  \begin{center}
    \begin{tikzpicture}[y=3.2cm, x=3.2cm,font=\sffamily]
      % Rays
      \draw[black,thick,->] (-1.1, 0) -- (1.1, 0) node[anchor=west] {$x$};
      \draw[black,thick,->] (0, -1.1) -- (0, 1.1) node[anchor=south east] {$y$};
      \draw[black,thin] (1,0) arc (0:360:1.0);
    \end{tikzpicture}
  \end{center}
  
  \begin{subproblem}
  \item Mark and label a point, $P$, on the circle where the ray from the origin to
    the point form an angle of $\pi$ radians with the positive $x$-axis.
  \item Mark and label a point, $Q$, on the circle where the ray from the origin to
    the point form an angle of $2\pi$ radians with the positive $x$-axis.
  \item Mark and label a point, $R$, on the circle where the ray from the origin to
    the point form an angle of $\frac{\pi}{2}$ radians with the positive $x$-axis.
  \item Mark and label a point, $S$, on the circle where the ray from the origin to
    the point form an angle of $\frac{\pi}{4}$ radians with the positive $x$-axis.
  \item Mark and label a point, $T$, on the circle where the ray from the origin to
    the point form an angle of $0$ radians with the positive $x$-axis.
  \item Mark and label a point, $U$, on the circle where the ray from the origin to
    the point form an angle of $\frac{3\pi}{4}$ radians with the positive $x$-axis.
  \end{subproblem}

  \clearpage

\item Make a sketch of a circle centered at the origin and has a
  radius $r$, and mark the radius and circumference of the circle.
  \sideNote{Label your axes and annotate your plot.}
  \vfill

  \begin{subproblem}
  \item What is the general relationship between the radius and the
    circumference?
    \vspace{4em}

  \item Mark a sector on your circle above whose angle is one half of
    the angle needed to make one complete turn around the circle. What
    is the arc length of the sector of the circle?  \sideNote{This should
      be a function of $r$. ($p=0.5$)}


    \vspace{4em}

  \item Mark a sector on your circle above whose angle is one third of
    the angle needed to make one complete turn around the circle. What
    is the arc length of the sector?  \sideNote{This should be a function
      of $r$.  ($p=0.3\overline{3}$)}


    \vspace{4em}

  \item If the angle of a sector is a fraction, $p$, of one whole turn around
    the circle, what is the arc length of the sector? (If $p=0.5$ then it represents
    one half of a full turn around the circle.)
    \sideNote{This should be a function of $r$ and $p$.}
    \vspace{4em}

  \end{subproblem}

\clearpage

\item From the previous problem you should have a general formula that
  relates the arc length of the sector with radius $r$ given the
  fraction, $p$, that its angle is of one complete turn around a
  circle.

  \begin{subproblem}
  \item Rewrite your expression, and label the distance along the
    sector as $s$.
    \vfill
  \item Divide both sides of your formula by the radius, and you
    should have an expression for
    \begin{eqnarray*}
    \frac{s}{r} & = &
    \end{eqnarray*}

  \item The value on the right side of your expression is the
    definition of radian measure for an angle. In one sentence explain
    the meaning of the value on the ride side of the expression.

    \vfill

  \end{subproblem}

  \clearpage

\item A hare is placed on a track that is a circle with radius
  10m. The hare moves around the track in the counter-clockwise
  direction.
  \begin{subproblem}
  \item Make a sketch of the track below. Determine the circumference
    of the circle.
    \vfill

  \item Determine the angle that the hare moves around after moving
    1m.  \sideNote{All angle measures should be in radians.}
    \vfill

  \item Determine the angle that the hare moves around after moving 10m.
    \vfill

  \item Mark and annotate the sector formed after the hare moves
    10m. What is the angle of the sector, and what is the area of the
    sector?
    \vfill

  \item Mark and annotate the sector formed after the hare moves
    2m. What is the angle of the sector, and what is the area of the
    sector?
    \vfill
  \item Mark and annotate the sector formed after the hare moves
    15m. What is the angle of the sector, and what is the area of the
    sector?
    \vfill
  \end{subproblem}

\clearpage

\item A turtle and a hare are placed at the same start point, and they
  move around a circle of radius 10m. The hare moves around the circle
  counter-clockwise at 1m per minute. The turtle moves around the
  circle counter-clockwise at 0.1m per minute.
  \begin{subproblem}
  \item After one hour how far has the hare moved along the circle?
    \vfill

  \item After one hour what angle has the hare moved around the circle?
    \vfill

  \item After one hour how far has the turtle moved on the circle?
    \vfill

  \item After one hour what angle has the turtle moved around the circle?
    \vfill

  \item How long will it take until the hare and the turtle are at the
    start point at the same time?
    \vfill
  \end{subproblem}


\end{problem}

\postClass

\begin{problem}
\item Briefly state two ideas from today's class.
  \begin{itemize}
  \item
  \item
  \end{itemize}
\item The following questions refer to the measure of an angle in
  radians.
  \begin{subproblem}
  \item How many radians are there in one complete turn around a
    circle?
    \vfill
  \item How many radians are there in one half of one complete turn
    around a circle?
    \vfill
  \item How many radians are there in one fourth of one complete turn
    around a circle?
    \vfill
  \item How many radians are there in one third  of one complete turn
    around a circle?
    \vfill
  \item If an angle is measured as being 45 degrees, how many radians
    is it?
    \vfill
  \item If an angle is measured as being 120 degrees, how many radians
    is it?
    \vfill
  \end{subproblem}
\end{problem}


%%% Local Variables:
%%% mode: latex
%%% TeX-master: "../labManual"
%%% End:
