%=========================================================================
% Start of activity on linear models
%=========================================================================
\preClass{Linear Models}

\begin{problem}
\item A chemical reaction has a single reactant that breaks down, 
  and the resulting reaction produces two different
  products. It is estimated that for each gram of the reactant that
  $\frac{1}{3}$g of the first product is produced and $\frac{1}{6}$g
  of the original reactant remains. Everything else that remains is
  the second product.
  \begin{subproblem}
  \item Write down all of the relevant information that is given in the
    statement above.
    \vfill
  \item If you start with four grams of reactant how many grams of the
    reactant will you get in the end?
    \vfill
  \item If you start with four grams of reactant how much of each
    product do you get?  (Keep in mind that mass is
    conserved so you have to have the same total mass that you started
    with.)
    \vfill
  \item Determine the number of grams of the reactants will result
    when you start with $x$ grams of reactant.
    \vfill
  \item Determine the number of grams of the products will result when
    you start with $x$ grams of reactant.
    \vfill
  \end{subproblem}
\end{problem}


\actTitle{Modeling With Linear Functions}
\begin{problem}

\item The time required to bake a ceramic depends on the mass of the
  ceramic. It is estimated to be a linear relationship between the
  time and mass. \textbf{The goal is to determine the time required to
    bake a ceramic sample given its mass.}
  \begin{subproblem}
    \item Define your variables for the quantities described
      above. Then make a rough sketch of the relationships described
      above that will given an idea of the general relationship
      between the variables.
      \sideNote{Label your axes and properly annotate your sketch.}
      \vfill

    \item Should there be a positive or negative slope for the
      relationship? Briefly justify your answer.
      \vfill

    \item With respect to the cost for each ceramic, would you prefer
      a larger or smaller slope for the relationship? Briefly justify
      your answer.
      \vfill

    \item A set of samples will be baked, and it is estimated that the
      baking time for a ceramic whose mass is 2,000g is five hours. It
      is also estimated that the baking time for a ceramic whose mass is
      3,000g is five and a half hours. Determine the baking time given
      the mass.

      \vfill
      \vfill

    \item A sample ceramic will be tested, and its mass is
      4,500g. How long would you expect it to take to bake the
      ceramic?
      \vfill

  \end{subproblem}

  \clearpage

\item Alice was born on the same day as her father, Bob. This year
  Bob's age is three times Alice's age. In fifteen years, Bob will be
  twice his daughter's age. What are their ages this year?
  \begin{subproblem}
  \item Define the variables that you will be using for Alice's and
    Bob's ages.
    \vspace{2em}
  \item Write the algebraic expression that
    indicates that Bob's age is three times Alice's age.  
    \label{question:aliceBobCurrentAge}
    \vspace{2em}
  \item If Alice's age is now $A$ what will her age be in fifteen
    years?  
    \vspace{2em}
  \item If Bob's age is now $B$ what will his age be in fifteen years?
    \vspace{2em}
  \item Use the two previous expressions, and write out the algebraic
    expression that indicates that Bob's future age will be twice
    Alice's future age.
    \label{question:aliceBobFutureAge}
    \vspace{2em}
  \item Use the two expressions from parts
    \ref{question:aliceBobCurrentAge} and \ref{question:aliceBobFutureAge} 
    to draw a sketch of the two linear relationships. Assume that
    Bob's age is a function of Alice's age, and the horizontal axis
    will Alice's age. Will the system have a solution that makes
    sense? 
    \vfill

  \item Use the two expressions from parts
    \ref{question:aliceBobCurrentAge} and \ref{question:aliceBobFutureAge} 
    to determine Alice's and Bob's age.
    \vfill

  \item Next to each of the previous steps, write a note in the left
    margin. Your note should indicate the relationship of the step and
    what part of the original statement it corresponds to.
  \end{subproblem}

\clearpage

\item Trucks are unloaded at a warehouse, and during the summer it is
  estimated that it takes longer to unload a truck if the weather is
  warmer. When the temperature is 22C it is estimated that it takes
  sixty minutes to unload a truck. For each increase of one degree
  Celsius it is estimated to take four additional minutes to unload a
  truck.
  \begin{subproblem}
  \item Define the variables that you will use.
    \vspace{2em}
  \item How long will it take to unload a truck if the temperature
    is 23C?
    \vspace{2em}
  \item How long will it take to unload a truck if the temperature
    is 24C?
    \vspace{2em}
  \item How long will it take to unload a truck given a temperature
    of T degrees Celsius?
    \vfill
  \end{subproblem}

  \clearpage

\item In the previous problem it was assumed that the truck was being
  unloaded during the summer. In the winter the relationship between
  unloading time and the temperature is different. When the
  temperature is 5C it is estimated that it takes fifty minutes to
  unload a truck, and for each decrease of one degree Celsius it is
  estimated to take three additional minutes to unload the truck.
  \begin{subproblem}
  \item Define the variables that you will use.
    \vspace{2em}
  \item Determine how long it will take to unload the truck for any
    temperature less than 5C.
    \vfill
  \item At what temperature should you switch from using the
    formula on the previous problem to using the formula above?
    \vfill
    \vfill
  \item Write out the formula to determine the time required to
    unload the truck given \textbf{any} temperature in Celsius.
    \vspace{3em}
  \item What is the shortest time to unload a time and what is the
    best temperature to unload a truck?  (Explain your reasoning.)
    \vspace{2em}
  \end{subproblem}

\end{problem}

\postClass

\begin{problem}
\item Briefly state two ideas from today's class.
  \begin{itemize}
  \item 
  \item 
  \end{itemize}
\item 
  \begin{subproblem}
    \item
  \end{subproblem}
\end{problem}

