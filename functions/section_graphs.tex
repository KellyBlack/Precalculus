
%=========================================================================
% Start of 
%=========================================================================
\preClass{Graphs of Functions}

\begin{problem}
\item A tortoise and a hare move in a straight line, and the both
  start at $x=0$. The tortoise's position is given by
  \begin{eqnarray*}
    x_T & = & \frac{1}{2} t,
  \end{eqnarray*}
  where $t$ is in minutes and $x$ is in meters.  The hare's position
  is given by
  \begin{eqnarray*}
    x_H & = & 2 t,
  \end{eqnarray*}
  where $t$ is in minutes and $x$ is in meters.
  \begin{subproblem}
  \item Determine the positions of the tortoise at $t=0$, $t=1$, and
    $t=2$.
    \vfill
  \item Determine the positions of the hare at $t=0$, $t=1$, and
    $t=2$.  
    \vfill
  \item For each time, plot the coordinate of the relative positions
    on the set of axes below. Use the tortoise's position for the
    $x$-coordinate, and use the hare's position for the
    $y$-coordinate. For example, if the tortoise's position is 1m, and
    the hare's position is 4m, then the coordinate would be $P(1,4)$.

    \scalebox{0.95}{\input{functions/img/emptyAxesHareTortoise.pgf}}

  \end{subproblem}
\end{problem}


\actTitle{Graphs of Equations}
\begin{problem}
\item Sketch a graph of the relationship given by
  \begin{eqnarray*}
    x^2 + 2x + y^2 - 8y & = & 8.
  \end{eqnarray*}
  Determine the center and the radius of the circle.
  \vfill

  \clearpage

\item Windows are constructed, and their width is proportional to
  their height. One window is measured, and its width is 100cm, and its
  height is 200cm. Make a sketch of the relationship of the height of
  a window given its width.

  \vfill

\item The surface area of a sparrow's wing is proportional to the
  square of the length of the wing. A sparrow is measured, and it has
  a wing length of 9cm and an area of 45cm\textsuperscript{2}. Make a
  sketch of the relationship of the area of a sparrow's wing given the
  length. 

  \vfill


\end{problem}

\postClass

\begin{problem}
\item Briefly state two ideas from today's class.
  \begin{itemize}
  \item 
  \item 
  \end{itemize}
\item 
  \begin{subproblem}
    \item
  \end{subproblem}
\end{problem}



%%% Local Variables:
%%% mode: latex
%%% TeX-master: "functions"
%%% End:
