%=========================================================================
% Start of
%=========================================================================
\preClass{Coordinate Systems}

\begin{problem}
\item Make a sketch of a number line with zero at the center.  Mark
  the locations of -2, -2.5, 1.1, and 2.3 on your number line.
  The relative distances between the points should be consistent.
  \sideNote{Label the number line by putting an ``$x$'' to the right
    and an arrow indicating the positive direction.}

  \vfill

\item Make a sketch of a number line with zero at the center.  Mark
  the locations of -2 and 2.15 on your number line. What is the
  distance between the two points?
  (The relative distances between the points should be consistent.)

  \vfill

\item Make a sketch of a number line with zero at the center.  Mark
  the locations of -1.54 and 2.07 on your number line. What is the
  distance between the two points?
  (The relative distances between the points should be consistent.)

  \vfill

\end{problem}


\actTitle{Coordinate Systems}
\begin{problem}
\item A set of points is given below as a table. Each point is given
  on a row. The left side of the table has the $x$-coordinate of each
  point, and the right side has the $y$ coordinate. Plot each point on
  the axes given below. Next to each point indicate which quadrant the
  point is located.

  \begin{tabular}{l|ll}
    $x$ & $y$ & Quadrant \\ \hline
     2.0 &  1.5 & \\ [10pt]
     2.2 & -4.5 & \\ [10pt]
    -3.4 &  2.8 & \\ [10pt]
    -2.8 & -1.3 & \\ [10pt]
    -4.5 &  2.5 & \\ [10pt]
  \end{tabular}

  \begin{tikzpicture}[y=1cm, x=1cm,font=\sffamily]
    % bounds
    \def\lowX{-5.5}
    \pgfmathtruncatemacro\startX{round(0.5+\lowX)}
    \pgfmathsetmacro\nextXValue{int(\startX+1)}
    \def\highX{5.5}
    \def\lowY{-5.5}
    \def\highY{5.5}
    \pgfmathsetmacro\nextYValue{int(\lowY+1)}
    % ticks
    \draw[step = 1, gray, very thin,dashed,opacity=0.85] (\lowX, \lowY) grid ( \highX,\highY);
 	% axis
	\draw[thick,->] (\lowX,0) -- coordinate (x axis mid) (\highX,0) node[anchor = north west] {$x$};
    \draw[thick,->] (0,\lowY) -- coordinate (y axis mid) (0,\highY) node[anchor = south east] {$y$};
    \foreach \y in {-5,-4,...,-1,1,2,...,\highY} {
      \draw (1pt, \y) -- (-1pt, \y) node[yshift=-6,xshift=-1,anchor=east] {$\y$};
    }
    \foreach \x in {-5,-4,...,-1,1,2,...,\highX} {
      \draw (\x,1pt) -- (\x,-1pt) node[yshift=-5,xshift=-1,anchor=east] {$\x$};
    }
  \end{tikzpicture}

  \vfill

  \clearpage

\item Mark the points $P_1(-2.1,-4.4)$ and $P_2(4.5,1.2)$ on the coordinate plane
  below. Determine the distance between the two points.  Include a
  sketch of a right triangle whose hypotenuse represents the distance
  between the two points.

  \begin{tikzpicture}[y=1cm, x=1cm,font=\sffamily]
    % bounds
    \def\lowX{-5.5}
    \pgfmathtruncatemacro\startX{round(0.5+\lowX)}
    \pgfmathsetmacro\nextXValue{int(\startX+1)}
    \def\highX{5.5}
    \def\lowY{-5.5}
    \def\highY{5.5}
    \pgfmathsetmacro\nextYValue{int(\lowY+1)}
    % ticks
    \draw[step = 1, gray, very thin,dashed,opacity=0.85] (\lowX, \lowY) grid ( \highX,\highY);
 	% axis
	\draw[thick,->] (\lowX,0) -- coordinate (x axis mid) (\highX,0) node[anchor = north west] {$x$};
    \draw[thick,->] (0,\lowY) -- coordinate (y axis mid) (0,\highY) node[anchor = south east] {$y$};
    \foreach \y in {-5,-4,...,-1,1,2,...,\highY} {
      \draw (1pt, \y) -- (-1pt, \y) node[yshift=-6,xshift=-1,anchor=east] {$\y$};
    }
    \foreach \x in {-5,-4,...,-1,1,2,...,\highX} {
      \draw (\x,1pt) -- (\x,-1pt) node[yshift=-5,xshift=-1,anchor=east] {$\x$};
    }
  \end{tikzpicture}

  \vfill

  \clearpage

\item Mark the point $P_3(1.3,-2.4)$ on the coordinate plane below. Determine the
  points on the $x$-axis that are a distance of 3 units from $P_3$.

  \begin{tikzpicture}[y=1cm, x=1cm,font=\sffamily]
    % bounds
    \def\lowX{-5.5}
    \pgfmathtruncatemacro\startX{round(0.5+\lowX)}
    \pgfmathsetmacro\nextXValue{int(\startX+1)}
    \def\highX{5.5}
    \def\lowY{-5.5}
    \def\highY{5.5}
    \pgfmathsetmacro\nextYValue{int(\lowY+1)}
    % ticks
    \draw[step = 1, gray, very thin,dashed,opacity=0.85] (\lowX, \lowY) grid ( \highX,\highY);
 	% axis
	\draw[thick,->] (\lowX,0) -- coordinate (x axis mid) (\highX,0) node[anchor = north west] {$x$};
    \draw[thick,->] (0,\lowY) -- coordinate (y axis mid) (0,\highY) node[anchor = south east] {$y$};
    \foreach \y in {-5,-4,...,-1,1,2,...,\highY} {
      \draw (1pt, \y) -- (-1pt, \y) node[yshift=-6,xshift=-1,anchor=east] {$\y$};
    }
    \foreach \x in {-5,-4,...,-1,1,2,...,\highX} {
      \draw (\x,1pt) -- (\x,-1pt) node[yshift=-5,xshift=-1,anchor=east] {$\x$};
    }
  \end{tikzpicture}

  \vfill

  \begin{subproblem}
  \item Place points on the graphs near to where you think the points
    \textbf{may} be located.
  \item Label one of the points, $(x,y)$. 
  \item Write out the general distance formula.
  \item What do you know about the point? Can you simplify the formula?
  \item Solve for the unknown variable.
  \end{subproblem}

  \vfill


\clearpage

\item Mark the point $P_4(1,-2)$ on the coordinate plane below. Mark \textbf{all}
  of the points that are a distance of 2 units from $P_4$.
  \label{act1:q:drawCircle}

  \begin{tikzpicture}[y=1cm, x=1cm,font=\sffamily]
      % bounds
      \def\lowX{-5.5}
      \pgfmathtruncatemacro\startX{round(0.5+\lowX)}
      \pgfmathsetmacro\nextXValue{int(\startX+1)}
      \def\highX{5.5}
      \def\lowY{-5.5}
      \def\highY{5.5}
      \pgfmathsetmacro\nextYValue{int(\lowY+1)}
      % ticks
      \draw[step = 1, gray, very thin,dashed,opacity=0.85] (\lowX, \lowY) grid ( \highX,\highY);
   	% axis
  	\draw[thick,->] (\lowX,0) -- coordinate (x axis mid) (\highX,0) node[anchor = north west] {$x$};
      \draw[thick,->] (0,\lowY) -- coordinate (y axis mid) (0,\highY) node[anchor = south east] {$y$};
      \foreach \y in {-5,-4,...,-1,1,2,...,\highY} {
        \draw (1pt, \y) -- (-1pt, \y) node[yshift=-6,xshift=-1,anchor=east] {$\y$};
      }
      \foreach \x in {-5,-4,...,-1,1,2,...,\highX} {
        \draw (\x,1pt) -- (\x,-1pt) node[yshift=-5,xshift=-1,anchor=east] {$\x$};
      }
    \end{tikzpicture}

  \vfill

  What kind of figure did you draw?

  \vfill

\clearpage

\item Suppose a point, $P(x,y)$ is a distance of 2 units from the
  point $P_4(1,-2)$.
  \begin{subproblem}
  \item Use the distance formula to express the distance relationship
    between $P$ and $P_4$.
    \vfill
  \item Square both sides of the previous equation.
    \vfill
  \end{subproblem}

\item Suppose a point, $P(x,y)$ is a distance of $R$ units from the
  point $P_4(1,-2)$.
  \begin{subproblem}
  \item Use the distance formula to express the distance relationship
    between $P$ and $P_4$.
    \vfill
  \item Square both sides of the previous equation.
    \vfill

  \item How does this formula relate to the figure you drew in
    question \ref{act1:q:drawCircle}.
  \end{subproblem}

\end{problem}


\postClass

\begin{problem}
\item Briefly state two ideas from today's class.
  \begin{itemize}
  \item
  \item
  \end{itemize}
\item For each equation below determine the values of $x$ that satisfy
  the equation. Determine the exact answer and also determine an
  approximations to the answer using two decimal places.
  \begin{subproblem}
    \item $2x^2 + 5x - 3 = 0$
    \item $5x-1=8x+7$
    \item $3x - 1 = 2x^2 + 2x + 6$
    \item $x^3 = 2$
  \end{subproblem}
\item Draw a coordinate axis, and properly label the axes. Use the
  axes to make a sketch of the graph of the relationship $y+x=2$.
\item Draw a coordinate axis, and properly label the axes. Use the
  axes to make a sketch of the graph of the relationship $y^2+x=2$.
\item Make a sketch of a number line with zero at the center.
  Indicate the set of numbers that satisfy $x^2>2$.
\item Make a sketch of a number line with zero at the center.
  Indicate the set of numbers that satisfy $x>2.2$ and $x<5.4$.
\item Make a sketch of a number line with zero at the center.
  Indicate the set of numbers that satisfy $|x|>1.5$.

\item 
  In mathematics the idea of proportionality has a specific
  definition. The idea is that when two things are proportional then
  any changes in one yield a similar change in the other. For example,
  suppose we have two quantities. The first we call $x$, and the other
  we call $y$. If $y$ is proportional to $x$, then if we double $x$
  then $y$ will double. Likewise, if we triple $x$ then $y$ will
  triple.

  We express this mathematically by noting that if $y$ is proportional
  to $x$ then the ratio of $y$ to $x$ must be a constant,
  \begin{eqnarray*}
    \frac{y}{x} & = & \mathrm{constant}.
  \end{eqnarray*}
  If we multiply both sides by $x$ then
  \begin{eqnarray*}
    y & = & x \cdot \mathrm{constant}.
  \end{eqnarray*}

  As an example, it is estimated that the length of a person's femur
  is proportional to the person's total height. This implies that
  \begin{eqnarray*}
    \frac{\mathrm{height}}{\mathrm{femur ~ length}} & = & \mathrm{constant}.
  \end{eqnarray*}
  In a paper by Obialor \textit{et al}\footnote{Ambrose Obialor,
    Churchill Ihentuge and Frank Akpuaka, \textbf{Determination of
      Height Using Femur Length in Adult Population of Oguta Local
      Government Area of Imo State Nigeria}, The FASEB Journal, April
    2015, vol. 29 no. 1 Supplement LB19}, it is estimated that in a
  specific area in Nigeria the mean height of women is 161.90 cm and
  the mean femur length of women is 40.82 cm. If a woman's femur has a
  length of 42.00 cm, what is her expected height?

  First, we have to estimate the value of the constant. Assuming that
  the means are consistent then 
  \begin{eqnarray*}
    \frac{161.90}{40.82} & = & \mathrm{constant.}
  \end{eqnarray*}
  Now we look at the expression for the unidentified woman,
  \begin{eqnarray*}
    \frac{\mathrm{height}}{42.00} & = & \frac{161.90}{40.82}.
  \end{eqnarray*}
  Solving for the height we get
  \begin{eqnarray*}
    \mathrm{height} & = & 42.00\cdot\frac{161.90}{40.82} \mathrm{cm}.
  \end{eqnarray*}
  \begin{subproblem}
  \item It is estimated that in a specific area in India the mean
    height of women is 160.60 cm and the mean femur length of women is
    40.65 cm. If a woman's femur has a length of 38.50 cm, what is her
    expected height?
  \item It is estimated that in a specific area in Canada the mean
    height of women is 162.50 cm and the mean femur length of women is
    41.16 cm. If a woman's height is 163.40 cm, what is the expected
    length of her femur?
  \end{subproblem}

\end{problem}
