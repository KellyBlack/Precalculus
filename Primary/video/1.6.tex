\actTitle{1.6 - Transfomations of Graphs}

\videoLink{Section 1.6 Day 1}{https://www.youtube.com/playlist?list=PLYHZK3b8UFw3srUWdN3pnV1WRNsEcqx-T}
% \videoLink{Section 1.6 Day 2}{https://www.youtube.com/playlist?list=PLYHZK3b8UFw3PC3qUoQPSdpFhzCqjRG8m}

\noindent \textbf{Topics:}  basic functions, vertical translations and scaling, horizontal translations and scaling, horizontal and vertical reflections, graphing transformations\\

\noindent \textbf{Student Learning Outcomes:}
\begin{enumerate}
\item (In class) Students will be able to recognize basic functions.
\item Students will be able to transform graphs of functions.
\item Students will be able to graph a function based on transformations.
\end{enumerate}

\hrule 

\bigskip


\subsection{Vertical and Horizontal Shifts} ~

\hspace{-.4in} \begin{tabular}{| l |} \hline \underline{Vertical shifts.} Assume that $c$ is a positive number. \\
The graph of $y = f(x)+c$ is obtained from the graph of $y=f(x)$ by shifting it $c$ units upward. \\
The graph of $y = f(x)-c$ is obtained from the graph of $y=f(x)$ by shifting it $c$ units downward.
 \\ \hline
\end{tabular}

\begin{enumerate}

\item For this problem, let $f(x)=x^2$. Sketch the graph of the function $y=f(x) - 3$. Compare the domains and ranges of $y=f(x)$ and $y=f(x)-3$.



\scalebox{0.4}{\includegraphics{bigaxes}} 


\newpage
\hspace{-.4in} \begin{tabular}{| l |} \hline \underline{Horizontal shifts.} Assume that $c$ is a positive number. \\
The graph of $y = f(x-c)$ is obtained from the graph of $y=f(x)$ by shifting it $c$ units to the right. \\
The graph of $y = f(x+c)$ is obtained from the graph of $y=f(x)$ by shifting it $c$ units to the left.
 \\ \hline
\end{tabular} 

\item Convince yourself that the rule above is correct by using the example $f(x)=x^2.$ Sketch the graph of $y = f(x -2)$ on the axes below.  \\
\scalebox{0.4}{\includegraphics{bigaxes}} \\[1in]

\item Compare the domains and ranges of $y = \sqrt{x}$ and $f(x)=\sqrt{x+4}$. Think about why this makes sense and is consistent with the shifting.\\[2in]



\item If the point $(3, -4)$ is on the graph of $y = f(x)$, find the corresponding point on the graph of $y = f(x - 5)+3$. \\

\newpage

\subsection{Reflection, Compression, and Stretching} ~

\hspace{-.4in} \begin{tabular}{| l |} \hline \underline{Reflection through the $x$-axis.} The graph of $y=-f(x)$ is obtained by reflecting the graph of $y=f(x)$ \\ through the $x$-axis.
 \\ \hline
\end{tabular} 


\vspace{-.1in}
\item For the graph of $y=f(x)$ shown below, sketch the graph of $y=-f(x)$.\\
\scalebox{0.5}{\includegraphics{pwshift1}} 




\hspace{-.3in} \begin{tabular}{| l |} \hline \underline{Vertical Compression/Stretching} Assume that $c$ is a positive number. \\
If $c>1$, the graph of $y=cf(x)$ is obtained by stretching the graph of $y=f(x)$ vertically \\ by a factor of $c$. \\
If $c$ is between $0$ and $1$, the graph of $y=cf(x)$ is obtained by compressing the graph vertically\\ by a factor of $1/c$.
 \\ \hline
\end{tabular} 


\vspace{-.1in}
\item If the point $P(3,-1)$ is on the graph of $y=f(x)$, find the corresponding point on the graph of (a) $y = 7f(x)$ and (b) $y = \dfrac{1}{4}f(x)$. \\[.4in]



\newpage
\item For the graph of $y=f(x)$ shown below, sketch the graph of $y=2f(x+3)-1$. \\
%\noindent Hint: work in stages, one modification at a time. Use order of operations to make your stages.\\
\scalebox{0.25}{\includegraphics{pwshift}} 
%\end{enumerate}
%
%
%
%
%\noindent MATH 1113 - Alli  \hfill Section 2.5 - Graphs of Functions\\
%\noindent Part 2 \\
%\noindent Topics: reflections of graphs, stretching/compressing of graphs   \\



\begin{tabular}{| l |} \hline \underline{Reflection through the $y$-axis.} The graph of $y=f(-x)$ is obtained by reflecting the graph of $y=f(x)$ \\ through the $y$-axis.
 \\ \hline
\end{tabular} 

%\begin{enumerate}
\vspace{-.1in}
\item For the graph of $y=f(x)$ shown below, sketch the graph of $y=f(-x)$.\\
\scalebox{0.5}{\includegraphics{pwshift4}} 






\hspace{-.4in} \begin{tabular}{| l |} \hline \underline{Horizontal Compression/Stretching} Assume that $c$ is a positive number. \\
If $c>1$, the graph of $y=f(cx)$ is obtained by compressing the graph horizontally by a factor $c$. \\
If $c$ is between $0$ and $1$, the graph of $y=f(cx)$ is obtained by stretching the graph of $y=f(x)$ \\ horizontally by a factor of $1/c$.
 \\ \hline
\end{tabular} 

\vspace{-.1in}
\item For the graph of $y=f(x)$ shown below, sketch the graph of $y=f(2x)$.\\
\scalebox{0.3}{\includegraphics{pwshift2}} 

\newpage
\item For the graph of $y=f(x)$ shown below, sketch the graph of $y = |f(x)|$.\\
\scalebox{0.3}{\includegraphics{sineflip}} \\[2in]





\end{enumerate}

\noindent \textbf{Student Learning Outcomes Check}

\begin{enumerate}
\item Can you transform graphs of functions?
\item Do you understand the difference between a shift and a reflection?  Or vertical and horizontal transformations?
\end{enumerate}

\noindent \textbf{If any of your answers were no, please ask about these topics in class.}



