%=========================================================================
% Start of 
%=========================================================================
\preClass{The Natural Logarithm}

\begin{problem}
\item Determine an approximation for the numerical value of each
  number below. Also determine the natural logarithm of each of the
  following values. Express the numbers to two decimal places.
  \begin{subproblem}
  \item $e$
    \vfill
  \item $e^2$
    \vfill
  \item $e^3\cdot e^{-4}$
    \vfill
  \item $e^{-2}\cdot e^{-3}$
    \vfill
  \end{subproblem}
\end{problem}


\actTitle{The Natural Logarithm}
\begin{problem}
\item Determine the natural logarithm of each value below, and use the
  properties of logarithms to express the value as a sum or
  differences of logarithms.
  \begin{subproblem}
  \item $a\cdot b \cdot c$
    \vfill
  \item $a\cdot b^2 \cdot c \cdot d^3$
    \vfill
  \item $\frac{(x-1)\cdot (x+3)}{(x-2)}$
    \vfill
  \item $\frac{(x-4)\cdot (x+2)^3 \cdot (x+4)}{(x-9)^2}$
    \vfill
  \end{subproblem}

\clearpage

\item A computer virus is constructed that will infect two new
  computer systems each day and then wipe the disk drive of its
  current computer clean.  The virus is installed on one computer on
  day 0 (zero).
  \begin{subproblem}
    \item Determine the formula that gives the number of computers
      infected on the number of days since the first computer is
      infected.
      \begin{eqnarray*}
        \mathrm{Number(t)} & = & 
      \end{eqnarray*}
    \item Make a rough sketch of the number of new computers infected
      as a function of time.
      \vfill
    \item Solve the equation above for $t$. That is determine a formula for
      $t$ given the number of newly infected computers.
      \vfill
    \item It is determined that 524288 new computers were infected on
      a given day. How long ago was the first virus installed?
      \vspace{3em}
    \item It is determined that 67108864 new computers were infected
      on a given day. How long ago was the first virus installed?
      \vspace{3em}
  \end{subproblem}

\clearpage

\item Radon 222 has a half life of 3.8 days. It is estimated that
  there is 0.2g of radon 222 in a basement. Assume that no more
  radon enters the basement after a treatment is applied, and the area
  is not ventilated.
  \begin{subproblem}
    \item Determine a formula for the amount of radon in the basement
      at a given time, in days, from when the treatment is applied.
      \begin{eqnarray*}
        \mathrm{Amount(t)} & = & 
      \end{eqnarray*}
    \item Make a rough sketch of the amount of Radon 222 in the
      basement as a function of time.  
      \vfill
    \item Solve the equation for $t$. That is determine a formula for
      $t$ given the amount of radon in the basement.
      \vfill
    \item How long will it take before the amount of radon is down to
      one tenth the original amount?
      \vfill
  \end{subproblem}

\clearpage

\item A patient has a growth, and a treatment is applied that is
  estimated to reduce the size of the growth by ten percent each
  week. 
  \begin{subproblem}
  \item Determine a formula for the size of the growth at a given
    time, in days, from when the treatment is applied.
      \begin{eqnarray*}
        \mathrm{Size(t)} & = & 
      \end{eqnarray*}
    \item Make a rough sketch of the size of the growth as a function of time.  
      \vfill
    \item Solve the equation for $t$. That is determine a formula for
      $t$ given the size of the growth.
      \vfill
    \item How long will it take before the growth is down to
      one tenth the original size?
      \vfill
  \end{subproblem}


\end{problem}


\postClass

\begin{problem}
\item Briefly state two ideas from today's class.
  \begin{itemize}
  \item 
  \item 
  \end{itemize}
\item 
  \begin{subproblem}
    \item
  \end{subproblem}
\end{problem}



\propertiesTitle{Properties of logarithms}

Logarithmic functions satisfy the algebraic properties given below. In
each example it is assumed that $b$, and $c$ are constants. 
\begin{eqnarray}
  \ln(a\cdot b) & = & \ln(a) + \ln(b) \\ [10pt]
  \ln\left(\frac{a}{b}\right) & = & \ln(a) - \ln(b) \\  [10pt]
  \ln\left(a^r\right) & = & r \ln(a) \\ [30pt]
  \log_{10}(a\cdot b) & = & \log_{10}(a) + \log_{10}(b) \\ [10pt]
  \log_{10}\left(\frac{a}{b}\right) & = & \log_{10}(a) - \log_{10}(b) \\  [10pt]
  \log_{10}\left(a^r\right) & = & r \log_{10}(a)
\end{eqnarray}


%%% Local Variables:
%%% mode: latex
%%% TeX-master: "../labManual"
%%% End:

