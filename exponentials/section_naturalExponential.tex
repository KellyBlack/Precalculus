%=========================================================================
% Start of 
%=========================================================================
\preClass{Algebra With Exponentials}

\begin{problem}
\item Determine an approximation for the value of each expression
  below. Your approximation should be to the nearest 0.01.
  \begin{subproblem}
  \item $2^{3}$
    \vfill
  \item $2.5^{2}$
    \vfill
  \item $2.7^{1.5}$
    \vfill
  \item $2.718^{2.1}$
    \vfill
  \item $2.7183^{2.44}$
    \vfill
  \end{subproblem}
\end{problem}


\actTitle{The Natural Exponential}
\begin{problem}
\item A bank manager is considering the impact of different terms for
  an account that offers compounded interest. She assumes that the
  interest rate is a constant annual one percent rate and then checks
  to see what happens for different lengths of time between
  compounding. Assume that one dollar is initially deposited.
  \begin{subproblem}
  \item Determine the amount of money in the account after one hundred
    years, if the interest is compounded yearly.  \vfill
  \item Determine the amount of money in the account after 100 years,
    if the interest is compounded once every six months.  \vfill
  \item Determine the amount of money in the account after 100 years,
    if the interest is compounded once a month.
    \vfill
  \item Determine the amount of money in the account after 100 years,
    if the interest is compounded once a day.
    \vfill
  \item Determine the amount of money in the account after 100 years,
    if the interest is compounded twice  a day.
    \vfill    
  \item What is happening to the balance as the number of terms
    increases?
    \vfill
  \end{subproblem}
  \clearpage

\item Generalize the value found on the previous page.
  \begin{subproblem}
  \item Determine a formula for the balance for 1\% annual interest
    after 100 years if the interest is compounded $n$ times per year.
    \vfill
  \item Make a substitution, $u=100n$, in the previous
    expression. Write out the expression for the balance as a function
    of $u$.
    \vfill
  \item Determine the values of the balance for the following values
    of $u$.

    \begin{tabular}{l|l}
      $u$ & balance \\ \hline
      1    &  \\ \hline
      2    &  \\ \hline
      12   &  \\ \hline
      365  &  \\ \hline
      1000 &  \\ \hline
    \end{tabular}

  \item What is the value approaching as $u$ gets large? This is a
    number that occurs in many situations, and we do not want to write
    it out every time we use it, so we use the symbol $e$ as a form of
    short hand notation.
    \begin{eqnarray*}
      e & = & 
    \end{eqnarray*}
    
  \end{subproblem}

  \clearpage

\item When interest is compounded continuously, the balance is
  determined using the function
  \begin{eqnarray*}
    \mathrm{Balance}(t) & = & P e^{rt},
  \end{eqnarray*}
  where $P$ is the initial balance, $r$ is the annual interest rate,
  and $t$ is the time in years.

  Sketch a plot of the balance over time if 1\$ is deposited with a
  rate of 1\%.

  \vfill

\clearpage

\item As radioactive isotopes decay, the amount of isotope in a sample
  decreases. If the decay rate of an isotope is $r$ then the amount of
  an isotope in a sample is expressed using the function
  \begin{eqnarray*}
    \mathrm{Amount}(t) & = & A e^{-rt},
  \end{eqnarray*}
  where $t$ is measured in years.

  \begin{subproblem}
  \item What is the physical interpretation of the constant $A$?
    \vspace{3em}

  \item If a sample of a radioactive substance initially contains 3
    grams, and the radioactive decay is 0.00004, sketch a plot of the
    amount of the substance in the sample over time.
  \end{subproblem}

\end{problem}

\postClass

\begin{problem}
\item Briefly state two ideas from today's class.
  \begin{itemize}
  \item 
  \item 
  \end{itemize}
\item 
  \begin{subproblem}
    \item
  \end{subproblem}
\end{problem}


%%% Local Variables:
%%% mode: latex
%%% TeX-master: "../labManual"
%%% End:

