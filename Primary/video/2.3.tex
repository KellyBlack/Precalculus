\documentclass[11pt]{article}
\usepackage{amsmath, amssymb, array}
\pagestyle{plain}

\textwidth=6.5in
\hoffset-1in
\textheight=9in
\voffset-1in

\usepackage{enumitem}
\usepackage{pgfplots}
\usepackage{graphicx}
\usepackage{lipsum}
\usepackage{stfloats}
\usepackage{multicol}
\setlength{\columnsep}{1cm}

\newcommand{\boxcolor}{gray!30}
\usepackage{mdframed}
\newenvironment{boxme}{\begin{mdframed}[backgroundcolor=\boxcolor,linewidth=0pt,nobreak=true]}{\end{mdframed}}
\newenvironment{boxthm}{\begin{mdframed}[backgroundcolor=\boxcolor,nobreak=true]}{\end{mdframed}}
\newenvironment{boxdef}{\begin{mdframed}[backgroundcolor=\boxcolor,linewidth=0pt,nobreak=true]}{\end{mdframed}}




\begin{document}


\noindent MATH 1113   \hfill 2.3 - Division of Polynomials and the Remainder and Factor Theorems\\



\noindent \textbf{Topics:}  long division, synthetic division, remainder theorem, factor theorem\\

\noindent \textbf{Student Learning Outcomes:}
\begin{enumerate}
\item Students will be able to divide polynomials using long division.
\item Students will be able to divide polynomials using synthetic division.
\item Students will be able apply the Remainder and Factor Theorems.

\end{enumerate}

\hrule 
\vspace{5mm}

\section{Long Division}

\begin{enumerate}

\item Use long division to divide $(-5+x+4x^2+2x^3+3x^4) \div (x^2+2)$.
\vfill
\vfill

\item Use long division to divide $\displaystyle \frac{2x^2+3x-14}{x-2}$.
\vfill

\newpage




\section{Synthetic Division}
\item Use synthetic division to divide $(-10x^2+2x^3-5) \div (x-4)$.
\vfill


\item Use synthetic division to divide $\displaystyle \frac{x^4+4x^3-2x+18}{x+2}$.
\vfill

\newpage
\section{Remainder and Factor Theorems}
\begin{boxthm}
{\bf Remainder Theorem}
If a polynomial $f(x)$ is divided by $x-c$, then the remainder is $f(c)$.
\end{boxthm}

\textbf{Note:  }The remainder theorem tell us that the value of $f(c)$ is the same as the remainder we get from dividing $f(x)$ by $x-c$.

\item Given $f(x)=x^4+6x^3-12x^2-30x+35$, use the remainder theorem to evaluate $f(2)$.
\vfill

\item Use the remainder theorem to determine if $c=\sqrt{3}$ is a zero of $f(x)=x^3+x^2-3x-3$.
\vfill

\newpage

\begin{boxthm}
{\bf Factor Theorem}
Let $f(x)$ be a polynomial.
\begin{enumerate}
\item If $f(c)=0$, then $(x-c)$ is a factor of $f(x)$.
\item If $(x-c)$ is a factor of $f(x)$, then $f(c)=0$.
\end{enumerate}
\end{boxthm}

\item Use the factor theorem to determine if $x-3$ is a factor of $f(x)=x^4-x^3-11x^2+11x+12$.
\vfill

\item Factor $f(x)=3x^3+25x^2+42x-40$, given that -5 is a zero of $f(x)$.  Then solve the equation $3x^3+25x^2+42x-40=0$.
\vfill
\vfill



\end{enumerate}

\noindent \textbf{Student Learning Outcomes Check}

\begin{enumerate}
\item Can you divide polynomials using long division?
\item Can you divide polynomials using synthetic division?
\item Are you able apply the Remainder and Factor Theorems?

\end{enumerate}

\noindent \textbf{If any of your answers were no, please ask about these topics in class.}













\end{document}