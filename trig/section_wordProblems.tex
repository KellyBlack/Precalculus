%=========================================================================
% Start of
%=========================================================================
\preClass{Word Problems}

\begin{problem}
\item A triangle, ABC, has lengths $a=5$, $b=4$, and the angle
  directly across from $c$ is $\gamma=\frac{\pi}{6}$.
  (The triangle is \textbf{not} a right triangle.)
  \begin{subproblem}
  \item Draw a picture of the triangle.
    \vfill
  \item Label the sides and the angle in your picture.
  \item Determine the area of the triangle. (Hint: draw a vertical
    line that represents the height in your triangle and use the
    appropriate trigonometric functions to determine the height.)
    \vfill
  \end{subproblem}
\end{problem}


\actTitle{Word Problems}
\begin{problem}
\item A surveyor sets up a transit 2m above the surface of the
  ground. The transit is 80m away from the base of a building.
  The transit is pointing at the top of the building,
  and its angle of elevation is 35 degrees. How tall is the building?
  \begin{subproblem}
  \item Make a sketch of the situation. (It may take a couple tries!)
    \vfill
    \vfill
  \item Indicate and label all of the information that is given and
    indicate any variables that are not known in your diagram above.
  \item Identify the relationships between the variables.
    \vfill
    \vfill
  \item How do you plan on solving the problem?
    \vfill
  \item Determine the height of the building.
    \vfill
    \vfill
  \end{subproblem}

\clearpage

\item Chris Hadfield is in the International Space Station and is
  360km above the surface of the earth. He looks down toward the
  center of the earth and then to the horizon of the earth. He
  measures an angle of 71 degrees between the two directions. What is
  the radius of the earth?
  \begin{subproblem}
  \item Make a sketch of the situation. (It may take a couple tries!)
    \vfill
  \item Indicate and label all of the information that is given and
    indicate any variables that are not known in your diagram above.
  \item Identify the relationships between the variables.
    \vfill
    \vfill
  \item How do you plan on solving the problem?
    \vfill
  \item Determine the radius of the earth.
    \vfill
    \vfill
  \end{subproblem}

\end{problem}

\postClass

\begin{problem}
\item Briefly state two ideas from today's class.
  \begin{itemize}
  \item
  \item
  \end{itemize}
\item Captain Horatio McCallister is standing on the bridge of his
  ship. He spies a buoy through his telescope. He estimates that the
  straight line distance to the buoy is 180m, and the angle of
  depression is 4 degrees. How far must he sail to get to the buoy?
\item A regular polygon has seven equal sides. The distance from each
  vertex to the center of the polygon is 0.5m. What is the area of the
  polygon?
\item A regular polygon with 8 sides is inscribed within a circle of
  radius 2m. What is the area between the circle and the polygon?
\item A circle of radius 2m is inscribed within a regular polygon with
  six sides. What is the area between the polygon and the circle?
\item The base of a rectangular box has dimensions 20cm by 15cm, and
  the height is $h$ cm. The angle between the bottom of the box and
  the diagonal is 21 degrees. What is the height of the box?
\end{problem}


%%% Local Variables:
%%% mode: latex
%%% TeX-master: "../labManual"
%%% End:
