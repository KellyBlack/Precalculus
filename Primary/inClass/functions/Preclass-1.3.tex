\preClass{Coordinate Systems}

\videoLink{Section 1.3}{https://www.youtube.com/playlist?list=PLYHZK3b8UFw1_K7VRAlYGdUZOaqjJTOZ\_}

\begin{table}[h]
  \center%
  \begin{tabular}{|l|l|}
    \hline
    \textbf{Actor $x$} & \textbf{Number of Oscar Nominations $y$} \\ \hline
    Tom Hanks          & 5                                        \\ \hline
    Jack Nicholson     & 12                                       \\ \hline
    Sean Penn          & 5                                        \\ \hline
    Dustin Hoffman     & 7                                        \\ \hline
  \end{tabular}
  \label{table:oscarNominations}
  \caption{Number of Oscar nominations for different actors.}
\end{table}



\begin{enumerate}
\item Use the relation given in the table above to answer the following.


\begin{enumerate}
\item Write a set of ordered pairs $(x,y)$ that defines the relation.\vfill
\item Write the domain of the relation.\vfill
\item Write the range of the relation.\vfill
\item Determine if the relation defines $y$ as a function of $x$.
\end{enumerate}
\vfill

\newpage
\item Given $f(x)=x^2+3x$ and $\displaystyle g(x)=\frac{1}{x}$, evaluate the function at the given value of $x$.
\begin{enumerate}
\item $f\left(-2\right)=$
  \vfill
\item $\displaystyle g\left(-\frac{1}{2}\right)=$
  \vfill
\end{enumerate}




\item Express the domain of the functions below three different ways,
  using interval notation, graphically on a line segment, as well as
  using inequalities.
  \begin{enumerate}
  \item $\displaystyle h(x)=\frac{x-3}{x-4}$
    \vfill
  \item $k(x)=\sqrt{x+9}$
    \vfill
  \item $\displaystyle m(x)=\frac{3}{\sqrt{x+9}}$
    \vfill
  \end{enumerate}





\end{enumerate}



