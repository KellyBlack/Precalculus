\actTitle{Exponential and Logarithmic Equations}



\noindent \textbf{Topics:}  solving exponential and logarithmic equations, exponential and logarithmic functions, defining relationships from written descriptions\\

\noindent \textbf{Student Learning Outcomes:}
\begin{enumerate}
\item Students will be able to solve exponential equations.
\item Students will be able to solve logarithmic equations.
\item Students will be able to use exponential and logarithmic equations in applications.
\end{enumerate}

\hrule 

\bigskip

\subsection{Exponential Equations} ~


\noindent \textbf{Equivalence Property of Exponential Expressions:} If $b, x,$ and $y$ are real numbers where $b>0, b \neq 1.$  Then $$b^x=b^y \text{ implies that } x=y.$$   



\begin{enumerate}
\item Solve the following exponential equations.
\begin{enumerate}
\item $\displaystyle 3^{2x-6}=81$\vfill
\item $\displaystyle 25^{4-t}=\Bigg(\frac{1}{5}\Bigg)^{3t+1}$\vfill
\end{enumerate}

\newpage

\noindent \textbf{Steps to Solve Exponential Equations by Using Logarithms}\\
1. Isolate the exponential expression on one side of the equation.\\
2. Take a logarithm of the same base on both sides.\\
3. Use he power property of logarithms to "bring down" the exponent.\\
4. Solve the resulting equation.\\

\item Solve the exponential equation using logarithms.
\begin{enumerate}
\item $\displaystyle 10^{5+2x}+820=49,600$\vfill
\item $\displaystyle 2000=18,000e^{-0.4t}$\vfill
\end{enumerate}



\item Solve the exponential equation.
\begin{enumerate}
\item $\displaystyle 4^{2x-7}=5^{3x+1}$\vfill
\item $\displaystyle e^{2x}+5e^x-36=0$\vfill
\end{enumerate}

\newpage

\subsection{Solve Logarithmic Equations} ~

\noindent \textbf{Equivalence Property of Logarithmic Expressions:} If $b, x,$ and $y$ are positive real numbers with $$\log_b x = \log_b y \text{ implies that } x=y.$$   


\item Solve the Logarithmic Equation.
\begin{enumerate}
\item $\displaystyle \log_2(3x-4)=\log_2(x+2)$\vfill
\item $\displaystyle \ln(x-4)=\ln(x+6)-\ln(x)$\vfill
\vfill
\end{enumerate}

\newpage

\noindent \textbf{Steps to Solve Logarithmic Equations by Using Exponential Form}\\
1. Given a logarithmic equation, isolate the logarithms on one side of the equation.\\
2. Use the properties of logarithms to write the equation in the form $\log_b x=k$, where $k$ is a  constant.\\
3. Write the equation in exponential form.\\
4. Solve the equation from step 3.\\
5. Check the potential solution(s) in the original equation.\\



\item Solve the logarithmic equation.
\begin{enumerate}
\item $4\log_3 (2t-7)=8$\vfill
\item $\log(w+47)=2.6$\vfill
\end{enumerate}


\newpage

\subsection{Exponential and Logarithmic Equations in Applications} 


\item A couple invests \$8000 in a bond fund. The expected yield is 4.5\% and the earnings are reinvested monthly.
\begin{enumerate}
\item Use $\displaystyle A=P \Bigg(1+\frac{r}{n}\Bigg)^{nt}$ to write a model representing the amount $A$ (in \$) in the account after $t$ years.  The value $r$ is the interest rate and $n$ is the number of times interest is compounded per year.\\[1in]
\item Determine how long it will take the initial investment to double.  Round to one decimal place.\vfill
\end{enumerate}
\newpage

\item Suppose that the sound at a rock concert measures 124 dB (decibels).
\begin{enumerate}
\item Use the formula $\displaystyle L=10\log \Bigg(\frac{I}{I_0} \Bigg)$ to find the intensity of sound $I$ (in W/m$^2$).  The variable $L$ represents the loudness of sound (in dB) and $I_0=10^{-12}$W/m$^2$.\vfill
\item If the threshold at which the sound becomes painful is 1 W/m$^2$, will the music at this concert be physically painful? \\[1.5in]
\end{enumerate}
\end{enumerate}

\noindent \textbf{Student Learning Outcomes Check}

\begin{enumerate}
\item Can you solve exponential equations?
\item Can you solve logarithmic equations?
\item Can you use exponential and logarithmic equations in applications?
\end{enumerate}

\noindent \textbf{If any of your answers were no, please ask about these topics in class.}


