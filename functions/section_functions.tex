%=========================================================================
% Start of 
%=========================================================================
\preClass{Introduction to Functions}

\begin{problem}
\item A biologist grows four different colonies of bacteria. The
  number of bacteria in the colonies is estimated to be 10,000,
  20,000, 30,000, and 40,000. The mass for each colony is measured and
  is estimated to be $3.33\times 10^{-6}$, $7.00\times 10^{-6}$,
  $1.70\times 10^{-6}$ and $2.78\times 10^{-6}$ grams respectively.

  Organize the information above into a table so that the mass can be
  more easily determined given the number of bacteria in the
  colony. Also, graph each point as a coordinate where the number of
  bacteria is on the horizontal axis, and the mass is on the vertical
  axis. 

  \vfill

\item Make rough estimate for a relationship that will provide a
  prediction for the mass of a colony given the number of bacteria
  within it.

  \vfill

\item Each time the number of bacteria increase by 10,000 what is the
  change in the mass?

  \vspace{5em}


\end{problem}


\actTitle{Functions}
\begin{problem}
\item A balloon has a tether to the ground, and it can be extended or
  retracted as the balloon is raised or lowered. One end of the tether
  is attached to the ground 20m away from a point directly below the
  balloon. If the balloon is $x$ meters high in the air what is the
  length of the tether?
  \begin{subproblem}
    \item Sketch a diagram of the situation. Label the known and
      unknown quantities.
      \sideNote{Assume that the balloon only moves up and down with no
        lateral motion.}
      \vfill
      \vfill
    \item Determine the important relationships between the known and
      unknown quantities.
      \vfill
    \item Determine the length of the tether given the height.
      \vfill
    \item Determine the domain and range of the function.
      \vfill
  \end{subproblem}

  \clearpage

\item A park has two distinct areas separated by a river, and each
  area has its own population of mice.  The population East of the
  river is estimated to have 10,000 individuals at the beginning of
  the year, and each week it grows by a constant 200 individuals. The
  population West of the river is estimated to have 8,000 individuals
  at the beginning of the year, and each week it grows by a constant
  250 individuals.

  \begin{subproblem}
  \item Make a rough sketch of the number of mice in the two
    populations on the same graph. The horizontal axis should be the
    time from the beginning of the year in weeks.
    \vfill

  \item Describe what is happening to the two populations. Is there a
    time when the two populations are equal? If so when is it?
    \vfill

  \item Determine a formula for the total number of mice in the park
    at any week after the beginning of the year.
    \vfill
  \end{subproblem}

  
\end{problem}

\postClass

\begin{problem}
\item Briefly state two ideas from today's class.
  \begin{itemize}
  \item 
  \item 
  \end{itemize}
\item 
  \begin{subproblem}
    \item
  \end{subproblem}
\end{problem}


%%% Local Variables:
%%% mode: latex
%%% TeX-master: "../labManual"
%%% End:

