%=========================================================================
% Start of 
%=========================================================================
\preClass{Introduction to Logarithms}

\begin{problem}
\item A computer virus is constructed that will infect two new
  computer systems each day and then wipe the disk drive of its
  current computer clean.  The virus is installed on one computer on
  1 January.
  \begin{subproblem}
  \item How many new computers will it infect on 2 January?
    \vfill
  \item How many new computers will it infect on 3 January?
    \vfill
  \item How many new computers will it infect on 4 January?
    \vfill
  \end{subproblem}
\item A computer virus is constructed that will infect two new
  computer systems each day and then wipe the disk drive of its
  current computer clean.  The virus is installed on a computer but it
  is not clear when it was first installed.
  \begin{subproblem}
  \item It is estimated that 128 computers were infected on a given
    day. How many days beforehand was it installed on the first computer?
    \vfill
  \item It is estimated that 1024 computers were infected on a given
    day. How many days beforehand was it installed on the first computer?
    \vfill
  \item It is estimated that 32768 computers were infected on a given
    day. How many days beforehand was it installed on the first computer?
    \vfill
  \end{subproblem}
\end{problem}


\actTitle{Introduction to Logarithms}
\begin{problem}
\item The approximate distances in kilometers from the earth to
  various destinations is given in the table below. Use this
  information to answer each of the questions below.

  \begin{tabular}{ll}
    Destination & Distance \\ \hline
    The moon & $3.84 \times 10^5$ km \\
    Mars &  $5.46 \times 10^7$ km \\
    Saturn & $1.28 \times 10^9$ km \\
    Proxima Centauri (Closest star) & $4.01 \times 10^{13}$ km \\
    Center of the Milky Way & $2.83 \times 10^{17}$ km
  \end{tabular}

  \begin{subproblem}
    \item Make a sketch of a number line with the distances for each
      destination marked on the number line.
      \vfill
    \item For each distance in the table above what is the approximate
      power of ten for the distance? For example, if a distance is
      $1\times 10^4$ km it is a power of 4. Sketch a number line and
      indicate the powers of ten on the number line.
      \vfill
  \end{subproblem}

\clearpage

The approximate average lengths of various items are given in the table
below. Use this information to answer each of the questions below.

  \begin{tabular}{ll}
    Item & Size \\ \hline
    Person & 1.75 m \\
    Finger & $9.2 \times 10^{-2}$ m\\
    DNA & $ 5.0 \times 10^{-2}$ m \\
    Hair (Width) & $ 9.0\times 10^{-5}$ m \\
    Bacteria & $ 0.8 \times 10^{-7}$ m\\
  \end{tabular}

  \begin{subproblem}
    \item Make a sketch of a number line with the lengths for each
      item  marked on the number line.
      \vfill
    \item For each item in the table above what is the approximate
      power of ten for the length? For example, if a length is
      $1\times 10^{-4}$ m it is a power of -4. Sketch a number line
      and indicate the powers of ten on the number line.
      \vfill
  \end{subproblem}

\clearpage

\item In one cycle a type of mayfly lays eggs and on average three
  females survive to lay more eggs. One of the mayfly are introduced
  to a stream for the first time and lays eggs. 
  \begin{subproblem}
  \item Draw a tree diagram of the individuals from each cycle that
    survive starting with the original mother. Include three cycles in
    your diagram.
    \vfill
  \item On the top of the tree diagram above indicate the total number
    of mayflies that survive each cycle.
  \item On the bottom of the tree diagram indicate the cycle with the
    first cycle labeled as cycle 0 (zero).
  \item It is estimated that there are 2187 mayfly in the river. How
    many cycles have there been?
    \vspace{3em}
  \item It is estimated that there are 19683 mayfly in the river. How
    many cycles have there been?
    \vspace{3em}
  \end{subproblem}

\clearpage

\item Mosquitoes lay between 50 to 200 eggs each cycle. Assume that
  for a given species in a particular area roughly 75 eggs hatch and
  survive to be adults and lay eggs.
  \begin{subproblem}
  \item Assume that in the spring there is one surviving female
    mosquito starting with cycle 0. Make a table to indicate how many
    mosquitoes there will be from cycle 0 to cycle 4.
    \vfill
  \item Using your table, if it is estimated that there are 5625
    mosquitoes what cycle in the season is it?
    \vfill
  \item Using your table, if it is estimated that there are 421875
    mosquitoes what cycle in the season is it?
    \vfill
  \end{subproblem}

\end{problem}

\postClass

\begin{problem}
\item Briefly state two ideas from today's class.
  \begin{itemize}
  \item 
  \item 
  \end{itemize}
\item Determine the value of $x$ in each expression below. Any
  approximations should be to at least two decimal places.
  \begin{subproblem}
    \item $\log_5(2x+1) = 9$
    \item $\log_{7}(x-1) + \log_{7}(x+1) = 2$
    \item $3^{\log_3(x+1)} = 4$
    \item $\log_8\left(8^{5-2x}\right) = 9$
    \item $\log_{10}(2x+1) = \log_3(4.1)$
  \end{subproblem}
\item Use the natural logarithm to determine the value of $x$ in each
  expression below. Any approximations should be to at least two
  decimal places.
  \begin{subproblem}
  \item $3.5^{x+1}=2^{x-1}$
  \item $10=e^{-2x}$
  \item $1=2^{3x}\cdot 4^{8x-1}$
  \item $5=e^{3x-1}$
  \item $e^{9x} = 2^{9x}$
  \end{subproblem}
\end{problem}


%%% Local Variables:
%%% mode: latex
%%% TeX-master: "../labManual"
%%% End:

