%=========================================================================
% Start of activity on angle measure
%=========================================================================
\preClass{Properties of Circles}

\begin{problem}
\item A circle has a radius of $r$ meters. Answer each of the
  following questions:
  \begin{subproblem}
  \item Determine the lengths of the circle described below?
    \begin{subproblem}
    \item What is the circumference of the whole circle?
      \vfill
    \item What is the length around half of the circle?
      \vfill
    \item What is the length around one-fourth of the circle?
      \vfill
    \end{subproblem}
  \item Determine the areas of the circle described below.
    \begin{subproblem}
    \item What is the area of the whole circle?
      \vfill
    \item What is the area of half of the circle?
      \vfill
    \item What is the area of one-fourth of the circle?
      \vfill
    \end{subproblem}
  \item How many degrees are there in a full circle? Where did this
    number come from?
    \vfill
  \end{subproblem}
\end{problem}


\actTitle{Angle Measurement}
\begin{problem}
\item For each of the angles below, print the angles in order from
  smallest to largest. In each picture the dotted line is the initial
  side, and the solid line is the terminal side.

    \vspace{1em}

  \begin{tikzpicture}[y=3cm, x=3cm,font=\sffamily]
    % Rays
    \draw[black,thick,->] (0, 0) -- (1,0);
    \draw[black,thick,dotted,->] (0, 0) -- (140:1);
    \draw[black,thin] (0.2,0) arc (0:140:0.2);
    \node at (70:0.32) {$\alpha$};

    \begin{scope}[shift={(1.2,0)}]
      % Rays
      \draw[black,thick,->] (0, 0) -- (1,0); 
      \draw[black,thick,dotted,->] (0, 0) -- (20:1); 
      \draw[black,thin] (0.2,0) arc (0:20:0.2);
      \node at (8:0.32) {$\beta$};
    \end{scope}

    \begin{scope}[shift={(0,-1.2)}]
      % Rays
      \draw[black,thick,->] (0, 0) -- (1,0);
      \draw[black,thick,dotted,->] (0, 0) -- (70:1);
      \draw[black,thin] (0.2,0) arc (0:70:0.2);
      \node at (35:0.32) {$\gamma$};
    \end{scope}

    \begin{scope}[shift={(1.5,-0.2)}]
      % Rays
      \draw[black,thick,->] (0, 0) -- (1,0);
      \draw[black,thick,dotted,->] (0, 0) -- (-100:1);
      \draw[black,thin] (0.2,0) arc (0:-100:0.2);
      \node at (-50:0.32) {$\delta$};
    \end{scope}

  \end{tikzpicture}

\item Use the diagram below to make the points indicated in the
  descriptions below.

  \begin{tikzpicture}[y=2.7cm, x=2.7cm,font=\sffamily]
    % Rays
    \draw[black,thick,->] (-1.1, 0) -- (1.1, 0);
    \draw[black,thick,->] (0, -1.1) -- (0, 1.1);
    \draw[black,thin] (1,0) arc (0:360:1.0);
  \end{tikzpicture}

  \begin{subproblem}
  \item Mark and label a point, $P$, where the ray from the origin to
    the point form an angle of $\pi$ radians.
  \item Mark and label a point, $Q$, where the ray from the origin to
    the point form an angle of $2\pi$ radians.
  \item Mark and label a point, $R$, where the ray from the origin to
    the point form an angle of $\frac{\pi}{2}$ radians.
  \item Mark and label a point, $S$, where the ray from the origin to
    the point form an angle of $\frac{\pi}{4}$ radians.
  \item Mark and label a point, $T$, where the ray from the origin to
    the point form an angle of $0$ radians.
  \item Mark and label a point, $U$, where the ray from the origin to
    the point form an angle of $\frac{3\pi}{4}$ radians.
  \end{subproblem}

  \clearpage

\item A hare is placed on a track that is a circle with radius
  10m. The hare moves around the track.
  \begin{subproblem}
  \item Make a sketch of the situation below.
    \vfill

  \item Determine the angle that the hare moves around after moving
    1m.  \sideNote{All angle measures should be in radians.} 
    \vfill

  \item Determine the angle that the hare moves around after moving 10m.
    \vfill

  \item Mark and annotate the sector formed after the hare moves
    10m. What is the angle of the sector, and what is the area of the
    sector?
    \vfill

  \item Mark and annotate the sector formed after the hare moves
    2m. What is the angle of the sector, and what is the area of the
    sector?
    \vfill
  \item Mark and annotate the sector formed after the hare moves
    15m. What is the angle of the sector, and what is the area of the
    sector?
    \vfill
  \end{subproblem}

\clearpage

\item A turtle and a hare are placed at the same start point, and they
  move around a circle of radius 10m. The hare moves around the circle
  counter-clockwise at 1m per minute. The turtle moves around the
  circle counter-clockwise at 0.1m per minute.
  \begin{subproblem}
  \item After one hour where is the hare on the circle?
    \vfill

  \item After one hour what angle has the hare moved around the circle?
    \vfill

  \item After one hour where is the turtle on the circle?
    \vfill

  \item After one hour what angle has the turtle moved around the circle?
    \vfill

  \item How long will it take until the hare and the turtle are at the
    start point at the same time?
    \vfill
  \end{subproblem}

\clearpage


\end{problem}

\postClass

\begin{problem}
\item Briefly state two ideas from today's class.
  \begin{itemize}
  \item
  \item
  \end{itemize}
\item
  \begin{subproblem}
    \item
  \end{subproblem}
\end{problem}


%%% Local Variables:
%%% mode: latex
%%% TeX-master: "../labManual"
%%% End:
