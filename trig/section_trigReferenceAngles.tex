%=========================================================================
% Start of activity on reference angles
%=========================================================================
\preClass{Reference Angles}

\begin{problem}
\item Make a sketch of the unit circle. 
  \sideNote{Label your axes and annotate your plot.}
  \vfill
  \vfill
  \vfill
\item Using your sketch above, draw the ray through the origin whose
  angle to the positive $x$-axis is $\frac{3\pi}{4}$ radians. (The ray
  should be in the second quadrant.)
\item Label the angle between the ray and the negative $x$-axis as $\beta$.
\item What is the value of $\beta$ in radians? Briefly explain how
  your determined your result.
  \vfill
\end{problem}


\actTitle{Reference Angles}
\begin{problem}
\item Make a sketch of the unit circle. Indicate the point whose angle
  coincides with the angle $\theta=\frac{11\pi}{6}$.
  \begin{subproblem}
    \item Determine the reference angle for $\theta$.
      \vfill
    \item According to some website,
    $\displaystyle{\sin\left(\frac{\pi}{6}\right)=\frac{1}{2}}$.
    Assuming this is correct determine the sine of the original angle,
    $\theta$.
      \vfill
    \item Determine the cosine of the original angle, $\theta$.
      \sideNote{Do not just use your caclulator! Use reference angles
        and justify your answer.}
      \vfill
  \end{subproblem}

  \clearpage

  \item Make a sketch of the unit circle. Indicate the point whose angle
    coincides with the angle $\theta=\frac{13\pi}{12}$.
    \begin{subproblem}
      \item Determine the reference angle for $\theta$.
        \vfill
      \item According to some website,
      $\displaystyle{\cos\left(\frac{\pi}{12}\right)=\frac{1}{4}\left( \sqrt{6} + \sqrt{2}\right)}$.
      Assuming this is correct determine the cosine of the original
      angle, $\theta$.
        \vfill
      \item Determine the sine of the original angle, $\theta$.
        \sideNote{Do not just use your caclulator! Use reference angles
          and justify your answer.}
        \vfill
    \end{subproblem}

    \clearpage

  \item The angle $\theta$ is in the second quadrant.
  \begin{subproblem}
    \item How do you calculate the reference angle given $\theta$?
      \sideNote{Make a sketch of the unit circle and label the angles.}
      \vfill
    \item What happens to the sine of the angle if its reference angle is
      increased and the angle remains in the second quadrant?
      \vfill
    \item What happens to the cosine of the angle if its reference angle is
      increased and the angle remains in the second quadrant?
      \vfill
  \end{subproblem}

  \clearpage

  \item The angle $\theta$ is in the third quadrant.
  \begin{subproblem}
    \item How do you calculate the reference angle given $\theta$?
      \sideNote{Make a sketch of the unit circle and label the angles.}
      \vfill
    \item What happens to the sine of the angle if its reference angle is
      increased and the angle remains in the third quadrant? (Does the
      sine increase or decrease?)
      \vfill
    \item What happens to the cosine of the angle if its reference angle is
      increased and the angle remains in the third quadrant? (Does the
      cosine increase or decrease?)
      \vfill
  \end{subproblem}

\end{problem}

\postClass

\begin{problem}
\item Briefly state two ideas from today's class.
  \begin{itemize}
  \item
  \item
  \end{itemize}
\item
  \begin{subproblem}
    \item
  \end{subproblem}
\end{problem}


%%% Local Variables:
%%% mode: latex
%%% TeX-master: "../labManual"
%%% End:
