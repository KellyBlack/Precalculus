\actTitle{5.2 - Sum and Difference Formulas}

\noindent \textbf{Topics:}  sum and difference formulas\\

\noindent \textbf{Student Learning Outcomes:}
\begin{enumerate}
\item Students will be able apply the sum and difference formulas for sine and cosine.
\item Students will be able to apply the sum and difference formulas for tangent.
\item Students will be able to use sum and difference formulas to verify identities.

\end{enumerate}

\hrule 

\bigskip

\subsection{Apply the Sum and Difference Formulas} ~

\begin{boxthm}
{\bf Sum and Difference Identities}

$$\sin(u+v) = \sin(u) \cos(v) + \cos(u) \sin(v) \hspace{3cm}\sin(u-v) = \sin(u) \cos(v) - \cos(u) \sin(v)$$

$$\cos(u+v) = \cos(u)\cos(v) - \sin(u)\sin(v) \hspace{3cm}\cos(u-v) = \cos(u) \cos(v) + \sin(u) \sin(v)$$

$$\tan(u+v) = \frac{\tan(u) +\tan(v)}{1-\tan(u) \tan(v)} \hspace{3cm}\tan(u-v) =  \frac{\tan(u) -\tan(v)}{1-\tan(u) \tan(v)}$$

\end{boxthm}






\begin{enumerate}
\vspace{-.1in}
\item Evaluate each of the following using the above identities and the unit circle.


\begin{enumerate}
\item $\cos\left(15^{\circ}\right)$
\vfill

\item $\sin\left(\frac{5\pi}{12}\right)$
\vfill

\newpage

\item $\sin\left(25^{\circ}\right)\cos\left(35^{\circ}\right) +\cos\left(25^{\circ}\right)\sin\left(35^{\circ}\right)$
\vfill

\item $\tan 75^{\circ}$
\vfill

\item $\sin \frac{\pi}{12}$
\vfill
 
\item $\cos 99^{\circ}\cos 36^{\circ}-\sin 99^{\circ}\sin 36^{\circ}$
\vfill

\end{enumerate}

\newpage


%\item Find the exact value of $\cos (\alpha - \beta)$ given that $\sin \alpha = -\frac{4}{5}$ and $\cos \beta = -\frac{5}{8}$ for $\alpha$ in Quadrant III and $\beta$ in Quadrant II.
%\vfill

\item Find the exact value of $\sin (\alpha + \beta)$ given that $\sin \alpha = \frac{5}{13}$ and $\cos \beta = \frac{5}{6}$ for $\alpha$ in Quadrant II and $\beta$ in Quadrant IV.
\vfill



\item Find the exact value of each of the following.

$$\sin \left( \arctan \left(-\frac{9}{4}\right)+\arccos \left(\frac{8}{17}\right)\right)$$
\vfill

%\item $\cos \left( \arcsin \left(-\frac{12}{37}\right)+\arctan \left(\frac{5}{12}\right)\right)$
%\vfill


\newpage

\subsection{Use Sum and Difference Formulas to Verify Identitities} ~

\item Prove that each of the following is an identity.

\begin{enumerate}
\item $\sin \left(\frac{\pi}{2}-x\right)=\cos(x)$
\vfill

\item $\sin(x+y)-\sin(x-y) = 2 \cos x \sin(y)$
\vfill

\item $\tan(\pi-x)=-\tan(x)$
\vfill


\end{enumerate}



\end{enumerate}

\noindent \textbf{Student Learning Outcomes Check}

\begin{enumerate}
\item Can you apply the sum and difference formulas for sine and cosine?
\item Can you apply the sum and difference formulas for tangent?
\item Are you able to use sum and difference formulas to verify identities?
\end{enumerate}

\noindent \textbf{If any of your answers were no, please ask about these topics in class.}

