%=========================================================================
% Start of activity on reference angles
%=========================================================================
\preClass{Reference Angles}

\begin{problem}
\item Determine the sine and cosine of the angle $\frac{\pi}{4}$.
  \vfill
\item Determine the sine and cosine of the angle $\frac{3\pi}{4}$.
  \vfill
\item Determine the sine and cosine of the angle $\frac{5\pi}{4}$.
  \vfill
\item Determine the sine and cosine of the angle $\frac{7\pi}{4}$.
  \vfill
\item Make a sketch of the unit circle and include the points on the
  unit circle whose angle coincides with the angles above. What is the
  relationship between the $x$ and $y$ values of the points in relationship
  to one another?
  \vfill
\end{problem}


\actTitle{Reference Angles}
\begin{problem}
\item Make a sketch of the unit circle. Indicate the point whose angle
  coincides with the angle $\theta=\frac{13\pi}{12}$.
  \begin{subproblem}
    \item Determine the reference angle for $\theta$.
      \vfill
    \item According to some website,
    $\displaystyle{\cos\left(\frac{\pi}{12}\right)=\frac{1}{4}\left( \sqrt{6} + \sqrt{2}\right)}$.
    Assuming this is correct determine the cosine of the angle.
      \vfill
    \item Determine the sine of the angle.
      \vfill
  \end{subproblem}

  \clearpage

  
\end{problem}

\postClass

\begin{problem}
\item Briefly state two ideas from today's class.
  \begin{itemize}
  \item
  \item
  \end{itemize}
\item
  \begin{subproblem}
    \item
  \end{subproblem}
\end{problem}


%%% Local Variables:
%%% mode: latex
%%% TeX-master: "../labManual"
%%% End:
