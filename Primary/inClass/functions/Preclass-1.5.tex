\preClass{Coordinate Systems}

\videoLink{Section 1.5}{https://www.youtube.com/playlist?list=PLYHZK3b8UFw0ZgjxBc6v6QlaHhSZheL-B}



\begin{enumerate}
\item A sales person makes a base salary of \$350 per week plus 15\% commission on sales.
\begin{enumerate}
\item Write a linear function to model the sales person's weekly salary $S(x)$ for $x$ dollars in sales.\vfill
\item Evaluate $S(700)$ and interpret the meaning in the context of this problem.
\end{enumerate}

\vfill
\item A town's population has been growing linearly. On January 1,
  2004 the population was 6,200. By July 1, 2009 the population had
  grown to 8,100. Assume this trend continues.

\begin{enumerate}
\item The town's population is a function of time, and a coordinate on
  the graph of the function can be written in the form $(t,P)$, where
  $t$ is the number of years since January 1, 2004 and $P$ is the
  population at the given time. From the statement above determine two
  points that will be on the graph of the function.
  
  \vspace{2em}

\item Use the points to determine a linear model for this data.
  
  \vfill

\item Interpret the meaning of the slope in this context.
  
\end{enumerate}

\vfill

\textit{(Continued on next page)}

\clearpage

\item A small animal moves along a wall and starts at a position 3.5
  meters from a corner.  The animal travels 0.3 meters away from the
  corner each minute. Determine the function that provides the
  distance the animal is from the corner given the number of minutes
  from the initial time. What are the units of the slope?

  \vfill

\item A manufacturing facility has 3.5 litres of a material available
  at the start of a given days. The facility produces 0.3 litres of
  material each hour. Determine the function that provides the amount
  of material available given the number of hours from the initial
  time. What are the units of the slope?

  \vfill




\end{enumerate}



