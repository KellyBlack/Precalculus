%=========================================================================
% Start of
%=========================================================================
\preClass{Graphs of Functions}

\begin{problem}
\item Two populations of different species of bacteria interact. The
  number of bacteria (in millions) in the first population is given by
  \begin{eqnarray*}
    B(t) & = & 10 + t^2,
  \end{eqnarray*}
  where $t$ is the time in days since the beginning of the year.  The
  number of bacteria (in millions) in the second population is given by
  \begin{eqnarray*}
    C(t) & = & 10+(t+2)^2,
  \end{eqnarray*}
  where $t$ is the time in days since the beginning of the year.
  \begin{subproblem}
  \item Make a sketch of the two functions below.
  \sideNote{Label your axes and properly annotate your plot.}
    \vfill
  \item For what values of $t$ does it make sense to use these functions?
    \vfill
  \item A researcher decides to alter the situation and adds 5 million
    bacteria to the first population given by $B(t)$. Determine a
    formula for the altered population. Make a sketch of the original
    and altered populations below.
    \vfill
  \end{subproblem}

\end{problem}


\actTitle{Graphs of Functions}
\begin{problem}
\item The height, in meters, of a certain tree changes by the
  relationship
  \begin{eqnarray*}
    h(t) & = & \sqrt{\frac{t}{3}},
  \end{eqnarray*}
  where $t$ is the time in years from when the seed was germinated.
  \begin{subproblem}
  \item Make a sketch of the height of a tree as a function of time.
    \sideNote{Label your axes and properly annotate your plot.}
    \vfill
  \item Two seeds are planted, and the first seed germinates
    immediately. The second seed germinates one year after the first
    is germinated, and then begins to grow. Determine the formulas for
    the height of the two trees with respect to the time that they
    were planted. Make a sketch of the two functions on
    the same graph.
    \vfill

  \item A new strain of the tree is developed that grows to the same height
    in half the time. Determine the formula that will
    give the height of the new strain. Make a sketch comparing the
    height of the original and the new strains.
    \vfill
  \end{subproblem}

  \clearpage

\item A function is defined to be
  \begin{eqnarray*}
    f(x) & = & |x|.
  \end{eqnarray*}
  \begin{subproblem}
  \item Make a sketch of the function on the axes below.
  \item Make a sketch of the following new functions on the graph as
    well with clear annotations:
    \begin{eqnarray*}
      g(x) & = & f(3x), \\
      h(x) & = & f(x)+2, \\
      p(x) & = & f(x+2), \\
      q(x) & = & 3f(x), \\
      r(x) & = & -f(x)-2.
    \end{eqnarray*}

    \begin{tikzpicture}[y=1.1cm, x=1.1cm,font=\sffamily]
        % bounds
        \def\lowX{-5.5}
        \pgfmathtruncatemacro\startX{round(0.5+\lowX)}
        \pgfmathsetmacro\nextXValue{int(\startX+1)}
        \def\highX{5.5}
        \def\lowY{-5.5}
        \def\highY{5.5}
        \pgfmathsetmacro\nextYValue{int(\lowY+1)}
        % ticks
        \draw[step = 1, gray, very thin,dashed,opacity=0.85] (\lowX, \lowY) grid ( \highX,\highY);
      % axis
      \draw[thick,->] (\lowX,0) -- coordinate (x axis mid) (\highX,0) node[anchor = north west] {$x$};
        \draw[thick,->] (0,\lowY) -- coordinate (y axis mid) (0,\highY) node[anchor = north east] {$y$};
        \foreach \y in {-5,-4,...,-1,1,2,...,\highY} {
          \draw (1pt, \y) -- (-1pt, \y) node[yshift=-6,xshift=-1,anchor=east] {$\y$};
        }
        \foreach \x in {-5,-4,...,-1,1,2,...,\highX} {
          \draw (\x,1pt) -- (\x,-1pt) node[yshift=-5,xshift=-1,anchor=east] {$\x$};
        }
        \draw (0,5.5) node [anchor=south] {Comparing Shifted Functions};
      \end{tikzpicture}

  \end{subproblem}
\end{problem}

\postClass

\begin{problem}
\item Briefly state two ideas from today's class.
  \begin{itemize}
  \item
  \item
  \end{itemize}
\item An enzyme in a solution decays, and the concentration (mg/liter)
  as a function of time in hours is
  \begin{eqnarray*}
    C(t) & = & \frac{3.5}{5.0+t}.
  \end{eqnarray*}
  \begin{subproblem}
  \item Make a sketch of the concentration as a function of
    time. Assume that the time is positive. Annotate your plot and
    label your axes.
    \item How long will it take for the enzyme to be reduced to half
      its original concentration?
    \item Another enzyme is present, and its concentration is linked
      to the first. Its concentration is half the first enzyme's
      concentration 30 minutes in the past. Determine the formula for
      the second enzyme's concentration. (This is referred to as a
      delay relationship.)
    \item How long will it take for the second enzyme to be reduced to
      half its original concentration?
    \item Make a sketch of the concentration of both enzymes as a
      function of time. Assume that the time is positive. Annotate
      your plot and label your axes.
  \end{subproblem}
\end{problem}


%%% Local Variables:
%%% mode: latex
%%% TeX-master: "../labManual"
%%% End:
