%=========================================================================
% Start of section on modeling with exponential and log equations.
%=========================================================================
\preClass{More Exponential and Logarithmic Equations}

\begin{problem}
\item Solve for the variable $x$ in each expression below.
  \begin{subproblem}
  \item $e^{4x-1}  =  8$
    \vfill
  \item $\ln(2x+1)  =  -4$
    \vfill
  \item $3^{2x+1}  =  e^{4x}$
    \vfill
  \item $8^{3x+2}  =  6e^{4x}$
    \vfill
  \end{subproblem}
\end{problem}


\actTitle{Exponential and Logarithmic Models}
\begin{problem}
\item A population of bacteria doubles every four hours. How long does
  it take for the population to triple?
  \vfill
\item Carbon 14 has a half life of 5,730 years. How long will it take
  for a sample to decay to two thirds of its original value?
  \vfill
\clearpage

\item The number of animals in a population of a small mammals follows
  a logistic function,
  \begin{eqnarray*}
    P(t) & = & \frac{10,000}{1+e^{-\frac{1}{2}t}}.
  \end{eqnarray*}
  \begin{subproblem}
  \item What is the initial population?
    \vspace{4em}
  \item How many animals will the population approach after a very
    long time?
    \vspace{4em}
  \item How long will it take for the number of animals to reach 75\%
    of its long term value?
    \vfill
  \item How long will it take for the number of animals to reach 80\%
    of its long term value?
    \vfill
  \end{subproblem}

\clearpage

\item A spill occurs and a chemical is introduced into a lake. It is
  estimated that the amount of the chemical in the lake decays like
  \begin{eqnarray*}
    \mathrm{Amount}(t) & = & A e^{rt}.
  \end{eqnarray*}
  \begin{subproblem}
  \item Should the value of $r$ be positive or negative? (Briefly
    explain how you arrive at your conclusion.)
    \vspace{5em}

  \item At some time after the spill occurs it is estimated that there
    are 4,000 kg of the chemical in the lake. A month later it is
    estimated that there is 3,500 kg in the lake. What is the half
    life of the chemical?  \vfill

  \item How long will it take for the amount of the chemical to be
    reduced to 1\% of the original amount that was spilled?
    \vfill
  \end{subproblem}

\end{problem}

\postClass

\begin{problem}
\item Briefly state two ideas from today's class.
  \begin{itemize}
  \item
  \item
  \end{itemize}
\item The number of animals in a population is approximated using a
  logistic function,
  \begin{eqnarray*}
    P(t) & = & \frac{80,000}{1+10e^{-\frac{1}{10}t}}.
  \end{eqnarray*}
  \begin{subproblem}
  \item Make a sketch of the function for $t$ going from 0 to 50.
  \item How long will it take for the population to double with
    respect to its original population?
  \item How many animals does the population approach in the long run?
    (i.e. what happens to the population after a very long time?)
  \end{subproblem}
\end{problem}


%%% Local Variables:
%%% mode: latex
%%% TeX-master: "../labManual"
%%% End:
