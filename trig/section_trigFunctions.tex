%=========================================================================
% Start of
%=========================================================================
\preClass{Trigonometric Functions}

\begin{problem}
\item An ant starts at the coordinate $P(1,0)$, and it moves
  counter-clockwise around a circle of radius one centered at the
  origin. It moves at a constant 1 meter per minute.
  \begin{subproblem}
  \item Sketch a plot of the ant's path.
    \sideNote{Label your axes and annotate your plot.}
    \vfill
  \item Sketch a plot of the ant's $x$-coordinate as a function of
    time.
    \sideNote{Label your axes and annotate your plot.}
    \vfill
  \item Sketch a plot of the ant's $y$-coordinate as a function of
    time.
    \sideNote{Label your axes and annotate your plot.}
    \vfill
  \end{subproblem}
\end{problem}


\actTitle{Trigonometric Functions}
\begin{problem}
\item
  \begin{subproblem}
    \item Make a sketch of the sine function.

      \hspace*{-3em}
      \begin{tikzpicture}[y=2cm, x=1.2cm,font=\sffamily]
          % bounds
          \def\lowX{0.0}
          \pgfmathtruncatemacro\startX{round(0.5+\lowX)}
          \pgfmathsetmacro\nextXValue{int(\startX+1)}
          \def\highX{12}
          \def\lowY{-1.25}
          \def\highY{1.25}
          \pgfmathsetmacro\nextYValue{int(\lowY+1)}
          % ticks
          \draw[step = 1, gray, very thin,dashed,opacity=0.85] (0, \lowY) grid ( \highX,\highY);
          % axis
          \draw[thick,->] (0,0) -- coordinate (x axis mid) (\highX,0) node[anchor = south west] {$\theta$};
          \draw[thick,->] (0,\lowY) -- coordinate (y axis mid) (0,\highY) node[anchor = north east] {Sine};

          \draw (1pt, 1) -- (-1pt, 1) node[yshift=-6,xshift=-1,anchor=east] { 1};
          \draw (1pt,-1) -- (-1pt,-1) node[yshift=-6,xshift=-1,anchor=east] {-1};

          \draw (1,1pt) -- (1,-1pt) node[yshift=-1,xshift=0,anchor=north east] {$\frac{ \pi}{2}$};
          \foreach \x in {3,5,...,\highX} {
            \draw (\x,1pt) -- (\x,-1pt) node[yshift=-8,xshift=1,anchor=east] {$\frac{\x\pi}{2}$};
          }
          \draw (2,1pt) -- (2,-1pt) node[yshift=-1,xshift=-1,anchor=north east] {$\pi$};
          \foreach \x in {2,3,...,6} {
            \draw (2*\x,1pt) -- (2*\x,-1pt) node[yshift=-7,xshift=3,anchor=east] {$\x\pi$};
          }
          \draw (6,1.3) node [anchor=south] {The Sine Function};
        \end{tikzpicture}

    \item After what values of $\theta$ does the sine function start to repeat
      itself?
      \vfill

    \item For what value of $a$ is
      \begin{eqnarray*}
        \sin(\theta) & = & \sin(\theta-a)?
      \end{eqnarray*}
      \vspace{1em}

    \item Is the sine function an invertible function?
      \vspace{1em}

      \clearpage
    \item Make a sketch of the cosine function.

      \hspace*{-5em}
      \begin{tikzpicture}[y=2cm, x=1.2cm,font=\sffamily]
          % bounds
          \def\lowX{0.0}
          \pgfmathtruncatemacro\startX{round(0.5+\lowX)}
          \pgfmathsetmacro\nextXValue{int(\startX+1)}
          \def\highX{12}
          \def\lowY{-1.25}
          \def\highY{1.25}
          \pgfmathsetmacro\nextYValue{int(\lowY+1)}
          % ticks
          \draw[step = 1, gray, very thin,dashed,opacity=0.85] (0, \lowY) grid ( \highX,\highY);
          % axis
          \draw[thick,->] (0,0) -- coordinate (x axis mid) (\highX,0) node[anchor = south west] {$\theta$};
          \draw[thick,->] (0,\lowY) -- coordinate (y axis mid) (0,\highY) node[anchor = north east] {Cosine};

          \draw (1pt, 1) -- (-1pt, 1) node[yshift=-6,xshift=-1,anchor=east] { 1};
          \draw (1pt,-1) -- (-1pt,-1) node[yshift=-6,xshift=-1,anchor=east] {-1};

          \draw (1,1pt) -- (1,-1pt) node[yshift=-1,xshift=0,anchor=north east] {$\frac{ \pi}{2}$};
          \foreach \x in {3,5,...,\highX} {
            \draw (\x,1pt) -- (\x,-1pt) node[yshift=-8,xshift=1,anchor=east] {$\frac{\x\pi}{2}$};
          }
          \draw (2,1pt) -- (2,-1pt) node[yshift=-1,xshift=-1,anchor=north east] {$\pi$};
          \foreach \x in {2,3,...,6} {
            \draw (2*\x,1pt) -- (2*\x,-1pt) node[yshift=-7,xshift=3,anchor=east] {$\x\pi$};
          }
          \draw (6,1.3) node [anchor=south] {The Cosine Function};
        \end{tikzpicture}


    \item For what values of $\theta$ does the cosine function start to repeat
      itself?
      \vfill

    \item For what value of $a$ is
      \begin{eqnarray*}
        \cos(\theta) & = & \cos(\theta-a)?
      \end{eqnarray*}
      \vspace{1em}

    \item Is the cosine function an invertible function?
    \vspace{1em}

  \end{subproblem}

  \clearpage

\item What is the relationship between the sine and cosine functions?
  For each relationship below determine the value and use the unit
  circle to explain why these relationships should be expected.
  \begin{subproblem}
  \item Determine a value of $a$ where
    \begin{eqnarray*}
      \cos(\theta) & = & \sin(\theta+a).
    \end{eqnarray*}
    \vfill
  \item Determine a value of $a$ where
    \begin{eqnarray*}
      \cos(\theta) & = & -\sin(\theta+a).
    \end{eqnarray*}
    \vfill

    \clearpage

  \item Determine values of $a$, and $b$ so that the function
    \begin{eqnarray*}
      f(\theta) & = & a\sin(\theta)+b
    \end{eqnarray*}
    oscillates between 2 and 6.
    \vfill

  \item Determine values of $a$, and $b$ so that the function
    \begin{eqnarray*}
      f(\theta) & = & a\sin(\theta)+b
    \end{eqnarray*}
    oscillates between -5 and -3.
    \vfill


  \item Determine values of $a$, and $b$ so that the function
    \begin{eqnarray*}
      f(\theta) & = & a\sin(\theta)+b
    \end{eqnarray*}
    oscillates between -1 and 4.
    \vfill
  \end{subproblem}

\end{problem}

\postClass

\begin{problem}
\item Briefly state two ideas from today's class.
  \begin{itemize}
  \item
  \item
  \end{itemize}
\item
  \begin{subproblem}
    \item
  \end{subproblem}
\end{problem}


%%% Local Variables:
%%% mode: latex
%%% TeX-master: "../labManual"
%%% End:
