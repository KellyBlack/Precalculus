%=========================================================================
% Start of
%=========================================================================
\preClass{Linear Functions}

\begin{problem}
\item A tortoise and a hare move in a straight line, and the both
  start at $x=0$. The tortoise's position is given by
  \begin{eqnarray*}
    x_T & = & \frac{1}{2} t,
  \end{eqnarray*}
  where $t$ is in minutes and $x$ is in meters.  The hare's position
  is given by
  \begin{eqnarray*}
    x_H & = & 2 t,
  \end{eqnarray*}
  where $t$ is in minutes and $x$ is in meters.

  Determine the relationship between the hare's and the tortoise's
  position. That is, given the hare's position determine the
  tortoise's position. Make a sketch of the graph of the relationship using the axes below.

  \begin{tikzpicture}[y=1.1cm, x=1.1cm,font=\sffamily]
      % bounds
      \def\lowX{-5.5}
      \pgfmathtruncatemacro\startX{round(0.5+\lowX)}
      \pgfmathsetmacro\nextXValue{int(\startX+1)}
      \def\highX{5.5}
      \def\lowY{-5.5}
      \def\highY{5.5}
      \pgfmathsetmacro\nextYValue{int(\lowY+1)}
      % ticks
      \draw[step = 1, gray, very thin,dashed,opacity=0.85] (\lowX, \lowY) grid ( \highX,\highY);
    % axis
    \draw[thick,->] (\lowX,0) -- coordinate (x axis mid) (\highX,0) node[anchor = north west] {Hare};
      \draw[thick,->] (0,\lowY) -- coordinate (y axis mid) (0,\highY) node[anchor = north east] {Tortoise};
      \foreach \y in {-5,-4,...,-1,1,2,...,\highY} {
        \draw (1pt, \y) -- (-1pt, \y) node[yshift=-6,xshift=-1,anchor=east] {$\y$};
      }
      \foreach \x in {-5,-4,...,-1,1,2,...,\highX} {
        \draw (\x,1pt) -- (\x,-1pt) node[yshift=-5,xshift=-1,anchor=east] {$\x$};
      }
      \draw (0,5.5) node [anchor=south] {Hare vs. Tortoise};
    \end{tikzpicture}


    What is the tortoise's position when the hare's position is 15
    meters? (Mark the associated coordinate on the plot above.)


\end{problem}


\actTitle{Linear Equations}
\begin{problem}
\item In each case below determine the formulas for the lines that
  satisfy the given requirements. In each case make a rough sketch of
  the line.

  \begin{subproblem}
  \item Goes through the point $P(-2,5)$ and has a slope of -3.
    \vfill
  \item Goes through the points $P_1(-3,-4)$ and $P_2(4,1)$.
    \vfill
  \end{subproblem}

  \clearpage

\item Birds near a park are studied by a group of researchers. The
  birds tend to use cigarette butts in their nests, and it is believed
  to help reduce the number of parasitic insects. It is estimated that
  the number of cigarette butts used for nesting materials varies
  linearly with the distance from the nest to a nearby open air
  theater. A nest that is a distance of 30 meters appears to have 10
  cigarette butts, and a nest that is a distance of 40 meters appears
  to have 8 cigarette butts.
  \begin{subproblem}
  \item Determine the relationship that will predict the number of
    cigarette butts in a nest given its distance from the theater.
    Use it to predict the number of cigarette butts in a nest 50
    meters from the theater. Also, make a sketch of the relationship.
    \sideNote{Be sure to label your axes and annotate your plot.}

    \vfill
    \vfill
    \vfill

  \item What is the domain for the relationship?
    \vfill
  \item A nest is found that has 4 cigarette butts. What is the
    prediction for the distance the nest is from the theater.
    \vfill
  \item If the conjecture for the reason why birds use cigarette butts
    in their nests is true what would you expect is the general
    relationship between the fledgling success rate for birds and the
    location of their nests?
  \end{subproblem}


\end{problem}

\postClass

\begin{problem}
\item Briefly state two ideas from today's class.
  \begin{itemize}
  \item
  \item
  \end{itemize}
\item The growth rate for a population is the change in the number of individuals
  in a unit time. The per-capita growth rate is the growth rate divided by
  the total number of individuals.
  The per-capita growth rate for a species is approximated as a
  linear function. It is estimated that when the population is near
  zero the per-capita growth rate is highest due to a lack of
  competition and approaches 0.5 (the time units are hours). When the
  population approaches 1,000 the per-capita growth rate is estimated
  to be zero.
  \begin{subproblem}
    \item What are the units for the per capita growth rate?
    \item Determine the relationship that gives the per-capita growth
      rate as a function of the population, $p$.
    \item Make a sketch of the graph of the per-capita growth
      rate. (Make sure to annotate your graph and label your axes.)
    \item What happens to the per-capita growth rate as the population
      increases? Why might this happen?
    \item Determine the values where the per-capita growth rate is
      negative. Why would the per-capita growth rate be negative?
  \end{subproblem}
\end{problem}


%%% Local Variables:
%%% mode: latex
%%% TeX-master: "../labManual"
%%% End:
