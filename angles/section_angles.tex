%=========================================================================
% Start of 
%=========================================================================
\preClass{Angle Measurement}

\begin{problem}
\item Make a sketch of a horizontal and vertical axes with labels for
  the $x$ and $y$ axes. Make a sketch of a circle centered on the
  origin of your axes. Label the radius and the
  circumference. Finally, answer each of the questions below.  

  \vfill

  \begin{subproblem}
  \item If the radius is 1m what is the circumference?
    \vspace{4em}
  \item If the radius is 2m what is the circumference?
    \vspace{4em}
  \item What is the general formula that relates the radius and the circumference?
    \vspace{4em}
  \item How many degrees are there in the angle around one complete
    circle?
    \vspace{4em}
  \item On your circle above draw a ray from the origin along the line
    that forms a 45 degree angle with the positive horizontal
    axis. Indicate and label the angle.
  \end{subproblem}
\end{problem}


\actTitle{Angle Measurement}
\begin{problem}
\item Make a sketch of a horizontal and vertical axes with labels for
  the $x$ and $y$ axes. Make a sketch of a circle of centered on the
  origin of your axes. Label the radius and the
  circumference. Finally, answer each of the questions below.

  \vfill

  \begin{subproblem}
    \item What is the length of the top half of the circle assuming
      its radius is 1?
      \vspace{3em}
    \item What is the length of the top half of the circle assuming
      its radius is $r$?
      \vspace{3em}
    \item What is the length of the part of the circle in the first
      quadrant assuming its radius is $r$?  
      \vspace{3em}

    \item What is the length of the sector that forms part of a circle
      with an angle of 180 degrees assuming its radius is 1?  
      \vspace{3em}
    \item What is the length of the sector that forms part of a circle
      with an angle of 180 degrees assuming its radius is $r$?  
      \vspace{3em}
    \item What is the length of the sector that forms part of a circle
      with an angle of 90 degrees assuming its radius is $r$?  
      \vspace{3em}

  \end{subproblem}

  \clearpage

\item Make a sketch of a circle of radius $r$, and mark the radius and
  circumference of the circle.
  \vfill

  \begin{subproblem}
  \item What is the general relationship between the radius and the
    circumference? 
    \vspace{4em}

  \item Mark a sector on your circle above whose angle is one half of
    the angle of one complete turn around the circle. What is the
    length of the sector that is along the outside of the circle?
    \sideNote{This should be a function of $r$.}


    \vspace{4em}

  \item Mark a sector on your circle above whose angle is in the
    second quadrant and the angle is one third of one complete
    turn around the circle. What is the length of the sector that is
    along the outside of the circle?
    \sideNote{This should be a function of $r$.}


    \vspace{4em}

  \item If the sector has an angle is $p$\% of one whole turn around
    the circle, what is the length of the sector along the outside of
    the circle?
    \sideNote{This should be a function of $r$ and $p$.}
    \vspace{4em}

  \end{subproblem}

\clearpage

\item From the previous problem you should have a general formula that
  relates the length of the part of a sector that is along the outside
  of a circle with radius $r$ given the percentage, $p$\%, that an
  angle is of one complete turn around a circle.

  \begin{subproblem}
  \item Rewrite your expression, and label the distance along the
    sector as $s$.
    \vfill
  \item Divide both sides of your formula by the radius, and you
    should have an expression for 
    \begin{eqnarray*}
    \frac{s}{r} & = & 
    \end{eqnarray*}

  \item The value on the right side of your expression is the
    definition of radian measure for an angle. In one sentence explain
    the meaning of the ratio on the left side of the expression.
    
    \vfill

  \end{subproblem}

  \clearpage

\item The following questions refer to the measure of angles in
  radians.
  \begin{subproblem}
  \item How many radians are there in one complete turn around a
    circle?
    \vfill
  \item How many radians are there in one half of one complete turn
    around a circle?  
    \vfill
  \item How many radians are there in one fourth of one complete turn
    around a circle?  
    \vfill
  \item How many radians are there in one third  of one complete turn
    around a circle?  
    \vfill
  \item If an angle is measured as being 45 degrees, how many radians
    is it?
    \vfill
  \item If an angle is measured as being 120 degrees, how many radians
    is it?
    \vfill
  \end{subproblem}

\end{problem}

\postClass

\begin{problem}
\item Briefly state two ideas from today's class.
  \begin{itemize}
  \item 
  \item 
  \end{itemize}
\item 
  \begin{subproblem}
    \item
  \end{subproblem}
\end{problem}


%%% Local Variables:
%%% mode: latex
%%% TeX-master: "../labManual"
%%% End:

