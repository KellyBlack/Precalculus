\actTitle{5.1 - Fundamental Trigonometric Identities}

\videoLink{Section 5.1}{https://www.youtube.com/playlist?list=PLYHZK3b8UFw0Afc951bZjITtow-fxztxp}

\noindent \textbf{Topics:}  inverse trig functions\\

\noindent \textbf{Student Learning Outcomes:}
\begin{enumerate}
\item Students will be able to simplify trigonometric expressions.
\item Students will be able to verify trigonometric identities.

\end{enumerate}

\hrule 

\bigskip

\subsection{Simplifying Trigonometric Expressions} ~

\begin{boxthm}
{\bf Fundamental Trigonometric Identities}

Reciprocal Identities

$$\sin(x) =\frac{1}{\csc(x)} \hspace{2cm}\cos(x) = \frac{1}{\sec(x)}\hspace{2cm}\tan(x) = \frac{1}{\cot(x)}$$

$$\csc(x) =\frac{1}{\sin(x)} \hspace{2cm}\sec(x) = \frac{1}{\cos(x)}\hspace{2cm}\cot(x) = \frac{1}{\tan(x)}$$

Quotient Identities

$$\tan(x) = \frac{\sin(x)}{\cos(x)} \hspace{3cm} \cot(x) = \frac{\cos(x)}{\sin(x)}$$

Pythagorean Identities

$$\sin^2(x) + \cos^2(x)=1 \hspace{2cm}\tan^2(x) +1 = \sec^2(x) \hspace{2cm}1+\cot^2(x) = \csc^2(x)$$

Even and Odd Identities

$$\sin( -x) =-\sin(x)\hspace{2cm}\cos( -x) = \cos(x)\hspace{2cm}\tan(- x) = -\tan(x)$$

$$\csc( -x) =-\csc(x)\hspace{2cm}\sec( -x) = \sec(x)\hspace{2cm}\cot(- x) = -\cot(x)$$

\end{boxthm}

{\bf Note:} It is often helpful to notice alternative forms of
Pythagorean Identities, such as $\sin^2(x) = 1-\cos^2(x)$ or
$\cos^2(x) = 1-\sin^2(x)$.




\begin{enumerate}
\vspace{-.1in}
\item Simplify each of the following. Write the final form with no fractions or products.

\begin{enumerate}
\item $\tan(x) \cos^2(x) \sec(x)$
\vfill


\newpage

\item $\displaystyle \frac{\cos (\theta)}{1+\sin (\theta)}+\tan (\theta)$
\vfill

%\item $\displaystyle \frac{\tan^2 (t) - 1}{\tan (t) \sin( t) + \sin (t)}$
%\vfill

\end{enumerate}

\subsection{Verify Trigonometric Identities} ~

\begin{boxthm}
{\bf Trigonometric Identities}

An {\bf identity} is an equation which it is true for all values of x for which the expressions on the left and right are defined.

\vspace{0.5cm}

{\bf Guidelines for proving Trigonometric Identities:}
\begin{enumerate}
\item Work with one side of the equation (usually the more complicated side) and keep the other side in mind as your final goal.

\item Look for opportunities to apply the fundamental identities.

\begin{itemize}

\item If the expression is a product or quotient of factors, consider the reciprocal and quotient identities.

\item If squared terms are present, look to see if the terms can be grouped in one of the forms of a Pythagorean identity.

\item If an expression involves a negative argument, consider using the even or odd function identities.

\end{itemize}

\item Apply basic algebraic techniques such as factoring, multiplying terms, combining like terms, and writing fractions with a common denominator.

\item Consider writing expressions explicitly in terms of sine and cosine.

\end{enumerate}

\end{boxthm}

\newpage

\item Prove that each of the following equations is an identity. (Note: The {\bf entire} proof is your answer.)
\begin{enumerate}
\item $\displaystyle \frac{\sin(-x)\cot(-x)}{\cos(x)} = 1$
\vfill

\item $\displaystyle \frac{1}{1-\cos (x)}-\frac{1}{1+\cos (x)} = 2\cot (x)\csc (x)$
\vfill
\vfill
\newpage

\item $\displaystyle 1 - \frac{\sin^2(t)}{1+\cos(t)} = \cos(t)$
\vfill

\item $\displaystyle \frac{\cot(x)}{\csc(x)}-\frac{\csc(x)}{\cot(x)} = -\sin(x) \tan(x)$
\vfill

\newpage

%\item $\displaystyle \frac{1}{1+\sin(x)}+\frac{1}{1-\sin(x)} = 2\sec^2(x)$
%\vfill

\item $\displaystyle \frac{\sin(x)}{\csc(x) - \cot(x)} = 1+\cos(x)$
\vfill


\end{enumerate}


\end{enumerate}
\vfill
\noindent \textbf{Student Learning Outcomes Check}

\begin{enumerate}
\item Are you able to simplify trigonometric expressions?
\item Are you able to verify trigonometric identities?

\end{enumerate}

\noindent \textbf{If any of your answers were no, please ask about these topics in class.}

