\actTitle{3.4 - Properties of Logarithms}

\videoLink{Section 3.4}{https://www.youtube.com/playlist?list=PLYHZK3b8UFw3229NnJqFoe05V02x1QMb8}

\noindent \textbf{Topics:}  properties of logarithms, change of base\\

\noindent \textbf{Student Learning Outcomes:}
\begin{enumerate}
\item Students will be able to apply the product, quotient, and power properties of logarithms.
\item Students will be able to write a logarithm in expanded form.
\item Students will be able to write a logarithm as a single logarithm.
\item Students will be able to apply the change of base formula.
\end{enumerate}

\hrule 

\bigskip

\subsection{Properties of Logarithms} ~


\noindent
\fbox{
  \parbox{\dimexpr\linewidth}%
  {    
    \noindent \textbf{Product Property of Logarithms:} Let $b, x,$ and
    $y$ be positive real numbers where $b \neq 1.$ Then
    $$\log_b (xy)=\log_b (x) + \log_b(y).$$
  }
}


\begin{enumerate}
\item Write the logarithm as a sum and simplify if possible.  Assume $x$ and $y$ represent positive real numbers. \\
\begin{enumerate}
\item $\log_2 (8x)$
  \vfill
\item $\ln (5xy)$
  \vfill
\end{enumerate}

\clearpage

\noindent
\fbox{
  \parbox{\dimexpr\linewidth}%
  {
    \noindent \textbf{Quotient Property of Logarithms:} Let $b, x,$
    and $y$ be positive real numbers where $b \neq 1.$ Then
    $$\log_b \left(\frac{x}{y}\right)=\log_b (x) - \log_b(y).$$
  }
}



\item Write the logarithm as a difference of logarithms and simplify
  if possible.  Assume $x$ and $y$ represent positive real numbers.
\begin{enumerate}
\item $\displaystyle \log_3 \left(\frac{c}{d}\right)$
  \vfill
\item $\displaystyle \log \left(\frac{x}{100}\right)$
  \vfill
\end{enumerate}


\noindent
\fbox{
  \parbox{\dimexpr\linewidth}%
  {
    \noindent \textbf{Power Property of Logarithms:} Let $b$ and $x$
    be positive real numbers where $b \neq 1.$ Let $p$ be any real
    number. Then $$\log_b (x^p)=p \log_b (x).$$
  }
}


\item Apply the power property of logarithms.
\begin{enumerate}
\item $\displaystyle\ln\left(\sqrt[5]{x^2}\right)$
  \vfill
\item $\displaystyle \log\left(x^2\right)$
  \vfill
\end{enumerate}



\subsection{Writing a Logarithmic Expression in Expanded Form}

\item Write the expression as a sum and differences of
  logarithms. (Reduce the expression as far as possible.)
\begin{enumerate}
\item $\displaystyle \log_2 \left(\frac{z^3}{xy^5}\right)$
  \vfill
\item $\displaystyle \log \left(\sqrt[3]{\frac{(x+y)^2}{10}}\right)$
  \vfill
\end{enumerate}

\clearpage

\subsection{Writing a Logarithmic Expression as a Single Logarithm}

\item Write the expression as a single logarithm and simplify the result if possible.
\begin{enumerate}
\item $\log_2\left(560\right) - \log_2\left(7\right)- \log_2\left(5\right)$
  \vfill
\item $\frac{1}{2} \ln\left(x\right) + \ln \left( x^2-1 \right)-
  \ln(x+1)$
  \vfill
\end{enumerate}



\subsection{Change of Base Formula} ~

\noindent
\fbox{
  \parbox{\dimexpr\linewidth}%
  {
    \noindent \textbf{Change-of-Base Formula: } Let $a$ and $b$ be
    positive real numbers such that $a \neq 1$ and $b \neq 1.$ Then
    for any positive real number $x$,
    $$\log_b\left(x\right)=\frac{\log_a\left(x\right)}{\log_a\left(b\right)}
       = \frac{\ln\left(x\right)}{\ln\left(b\right)}$$
   }
}

\item Use the change of base formula to approximate
  $\log_4\left(153\right)$ by using the base $e$ logarithm.
  \vfill

\end{enumerate}

\noindent \textbf{Student Learning Outcomes Check}

\begin{enumerate}
\item Can you apply the product, quotient, and power properties of logarithms?
\item Can you write a logarithm in expanded form?
\item Can you write a logarithm as a single logarithm?
\item Are you able to apply the change of base formula?
\end{enumerate}

\noindent \textbf{If any of your answers were no, please ask about these topics in class.}


