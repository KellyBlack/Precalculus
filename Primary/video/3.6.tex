\actTitle{3.6 - Modeling with Exponential and Logarithmic Functions}


\noindent \textbf{Topics:}  solving exponential and logarithmic equations, applications of exponential and logarithmic functions\\

\noindent \textbf{Student Learning Outcomes:}
\begin{enumerate}
\item Students will be able to use exponential and logarithmic equations in applications.
\item Students will be able to construct equations from written descriptions.
\item Students will be able to identify growth vs decay.
\end{enumerate}

\hrule 

\bigskip

\subsection{Solve Equations for a Specified Variable}



\begin{enumerate}
\item One hundred emu, each 1 year old, are to be introduced into an emu sanctuary.  The number $N(t)$ alive $t$ years later is predicted to be $N(t)=100(0.8)^t$.
\begin{enumerate}
\item Estimate the number alive after 5 years.\\[.3in]
\item When will there be 50 emu remaining? (Round to the nearest year.)\\[1in]
\end{enumerate}


\subsection{Creating a Model for Growth and Decay}

\item Suppose that \$15,000 is invested and at the end of 3 years, the value of the account is \$19,356.92.  Use the model $A=Pe^{rt}$ to determine the average rate of return $r$ under \\continuous compounding. (Round your answer to the nearest tenth of a percent.)

\newpage

\noindent \begin{tabular}{| l |} \hline
If you are not given a growth/decay function for your bacteria, radioactive substance, etc., assume\\ that your function has the form $P=P_0e^{kt}$ where the initial population is $P_0$, the rate of growth \\or decay is $k$, and $t$ is time (in consistent units for the whole problem).\\ \hline
\end{tabular} \\

\item 75\% of a radioactive material remains after 13 days. 
\begin{enumerate}
\item Find the decay constant. (Do not round your answer.)\\[1.5in]
\item Find the time (in days) after the initial measurement when 15\% of the original amount of radioactive material remains. (Round to the nearest whole number.) \\[2.5in]
\end{enumerate}


\item If a certain bacteria population doubles in 3 hours, determine the time $t$ (in hours) that it takes the population to triple. (Do not round your answer.)\\[2in]

\newpage

\subsection{Logistic Growth Models} ~

\noindent \begin{tabular}{| l |} \hline
\textbf{Logistic Growth Model:  } A logistic growth model is a function written in the form \\
$\displaystyle y=\frac{c}{1+ae^{-bt}}$ where $a$, $b$, and $c$ are positive constants.
\\ \hline
\end{tabular} \\

\item The population of California $P(t)$ (in millions) can be approximated by the logistic growth function $\displaystyle P(t)=\frac{95.2}{1+1.8e^{-0.018t}}$, where $t$ is the number of years since the year 2000. (Round to the nearest tenth of a year.)

\begin{enumerate}
\item Determine the population in the year 2000.\\[.5in]
\item Use this function to determine the time required for the population of California to double from its value in 2000.\vfill
\item What is the limiting value of the population of California under this model.\\[1in]
\end{enumerate}

\end{enumerate}

\noindent \textbf{Student Learning Outcomes Check}

\begin{enumerate}
\item Can you use exponential and logarithmic equations in applications?
\item Are you able to construct equations from written descriptions?
\item Can you identify growth vs decay?
\end{enumerate}

\noindent \textbf{If any of your answers were no, please ask about these topics in class.}



