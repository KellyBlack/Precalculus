

\actTitle{Worksheet 3.5A}

\noindent \textbf{Instructions:}  Work together in groups of  3 or 4
to complete the following problems.\\

Student goals:
  \begin{itemize}
  \item Solve equations with multiple exponential terms.
  \item Determine inverses of complicated functions that have either
    exponential or logarithmic terms.
  \item Manipulate equations with exponentials and transform them into
    other forms. For example, transform an expression with exponential
    terms into a quadratic equation.
  \item Define relationships given written descriptions that include
    wither exponential or logarithmic terms.
  \item Convert and write any exponential function using base $e$.
  \item Determine the value of any or all parameters in a compound
    interest problem given a written description of the situation.
  \item Determine the value of any or all parameters in an exponential
    growth/decay problem given a written description of the
    situation. Be able to identify if a situation results in either
    growth or decay.
  \item Determine the value of any or all parameters given a written
    description of a logistic growth function.
  \end{itemize}


  \noindent
\fbox{
  \parbox{\dimexpr\linewidth}%
  {
    
    We have learned 3 techniques to solve exponential and logarithmic
    equations.
    \begin{description}
    \item[(1) Equivalence Property of Exponential Expressions:] If $b, x,$
      and $y$ are real numbers where $b>0, b \neq 1.$ Then
      $$b^x=b^y \text{ implies that } x=y.$$

    \item[(2) Equivalence Property of Logarithmic Expressions:] If $b, x,$
      and $y$ are positive real numbers with
      $$\log_b (x) = \log_b (y) \text{ implies that } x=y.$$

    \item[(3) Solving Exponential Equations by Using Logarithms:] Isolate
      the exponential expression on one side of the equation and take a
      logarithm of the same base on both sides.
    \end{description}

  }
}

\clearpage

\begin{enumerate}

\item  For each equation, determine the best
  technique to use to solve the equation.  \textbf{Do not solve the
    equations.}
\begin{enumerate}
\begin{multicols}{2}
\item $3^x=81$\\[.2in]
\item $\log(x^2+6x)=\log(7)$\\[.2in]
\item $\displaystyle 11^{3x+1}=\Big(\frac{1}{11}\Big)^{x-5}$\\[.2in]
\item $6^x=87$\\[.2in]
\columnbreak
\item $10^{3+4x}-8100=120,000$ \\[.2in]

\item $\log_4(3x+11)=\log_4(3-x)$\\[.2in]

\item $1024=19^x+4$\\[.2in]
\item $5^{2x+2}=625$\\[.2in]
\end{multicols}
\end{enumerate}


\item Solve the following equations using the appropriate techniques.
  Then check your answers.
\begin{enumerate}

\item $2^t=32$
  \vfill

\item $100^{3m-5}=1000^{3-m}$
\vfill

\clearpage
\item $2^x=70$
\vfill
\item $80=320e^{-0.5t}+3$
\vfill

\item $\log_3(12-x)=\log_3(x+6)$
\vfill

\item $\ln(w^2+7w)=\ln(18)+1$
\vfill


\end{enumerate}


\clearpage

\item Sometimes, we need to use other techniques and properties to
  solve exponential equations.  For example, when there are multiple
  exponential functions, it can be helpful to take $\ln$ or $\log$ of
  both sides.  Solve the equations.  Then check your answers.
\begin{enumerate}
\item $3^{6x+5}=5^{2x}$
\vfill

\item $2^{1-6x}=7^{3x+4}$
\vfill

\end{enumerate}
\clearpage

\item You may also notice that an exponential equation has the same
  form as a quadratic equation.  So you may need to substitute to
  solve the problem like a quadratic equation.  Solve the equations.
  Then check your answers.
\begin{enumerate}
\item $e^{2x}-9e^x-22=0$\vfill
\item $e^{2x}-6e^x-16=0$\vfill
\end{enumerate}

\item Show that the function
  \begin{eqnarray*}
    k(x) & = & 3 \ln(x) + 1
  \end{eqnarray*}
  is one-to-one.
  \vfill

  \clearpage

\item The body temperature of a reptile is $38^\circ$ Celsius. The
  reptile moves into a cooler, sheltered area, and the reptile's body
  temperature at a time, $t$ minutes, after moving is
  \begin{eqnarray*}
    BT(t) & = & 30+C\cdot e^{-0.15t}.
  \end{eqnarray*}

  \begin{enumerate}
  \item Given the reptile's initial temperature, determine the value
    of $C$.
    \vfill
  \item How long will it take for the reptile's body temperature to
    reach $34^\circ$ Celsius?
    \vfill
  \item What happens to the reptile's body temperature in the long
    term? What does this imply about the ambient temperature of the
    sheltered area?
    \vfill
  \end{enumerate}


\end{enumerate}


\noindent \textbf{Once you have finished this worksheet, go back and
  solve the equations from \#1.}


\hwTitle{Section 3.5A}

\begin{enumerate}
\item Solve the following equations using the appropriate techniques.
  Then check your answers.
  \begin{enumerate}
  \item $\sqrt[3]{5}=5^x$
  \item $\displaystyle 7^{2p-3}=\Big(\frac{1}{49}\Big)^{p+1}$
  \item $801=23^y+6$
  \item $\ln(2x+1) = \ln(x)$
  \item $\log_9(1-2x) = \log_9(x) - 3$
  \end{enumerate}
\item For each function below determine algebraically whether or not
  the function is one-to-one. Determine the inverse of a function if
  it is one-to-one.
  \begin{enumerate}
  \item ${\displaystyle c(x) = e^{-3x+1}}$
  \item ${\displaystyle d(x) = 4\cdot 6^{x+1} + 1}$
  \item ${\displaystyle g(x) = 4 + 9\ln\left(x+1\right)}$
  \item ${\displaystyle h(x) = \ln\left(x^2\right)}$
  \end{enumerate}
    
\item The eggs of a bird are placed in an incubator, and the
  temperature of the eggs are the same as that of the inside of the
  incubator. When the incubator is on its internal temperature is
  $40^\circ$ Celsius. If the power is cut off to the incubator then
  its internal temperature as a function of the number of minutes
  since the power was cut, $t$, is given by
  \begin{eqnarray*}
    IT(t) & = & 26+C\cdot e^{-0.05t}.
  \end{eqnarray*}

  \begin{enumerate}
  \item If the incubator is at its initial temperature of $40^\circ$
    Celsius and power is cut off, determine the value of $C$.
  \item Describe the long term trend of the internal temperature when
    the power is cut to the incubator.
  \item The eggs of a particular bird should be periodically cooled to
    $35^\circ$ Celsius and then reheated back to their normal
    temperature. How long should the incubator be turned off for this
    to occur?
  \end{enumerate}


\end{enumerate}
