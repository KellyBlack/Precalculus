\actTitle{1.5 - Modeling with Linear Equations}

\videoLink{Section 1.5}{https://www.youtube.com/playlist?list=PLYHZK3b8UFw0ZgjxBc6v6QlaHhSZheL-B}

\noindent \textbf{Topics:}  linear relationship, interpretation of the point-slope formula, and constructing a linear equation\\

\noindent \textbf{Student Learning Outcomes:}
\begin{enumerate}
\item Students will be able to apply the point-slope formula.
\item Students will be able to construct a linear equation from a written description.

\end{enumerate}

\hrule 

\bigskip

\subsection{Applying the Point-Slope Formula} ~

\begin{tabular}{| l |}\hline
The \underline{point-slope equation} for the line through the point $(x_1,y_1)$ with slope $m$ is \\

 $y-y_1 = m(x-x_1)$. \\ \hline
\end{tabular} 

\vspace{-.1in}
\begin{enumerate}
\item Use the point-slope formula to find an equation of the line passing through the point $(2,-3)$ and having slope -4.  Write your answer in slope-intercept form.\\[1.5in]









\item Use the point-slope formula to write an equation of the line passing through the points $(4,-6)$ and $(-1,2)$.  Write your answer in slope-intercept form.\\[3in]



\newpage


\subsection{Create Linear Functions to Model Data}
\noindent In many day-to-day applications, two variables are related linearly.  This means that any given any change in an independent variable, $x$, will always produce a corresponding change in the dependent variable, $y$.  And when you plot this relationship on a graph, it traces a straight line.\\

\item A family plan for a cell phone has a monthly base price of \$99 plus \$12.99 for each additional family member added beyond the primary account holder.
\begin{enumerate}
\item Write a linear function to model the monthly cost $C(x)$, in dollars, of a family plan for $x$ additional family members added.\\[1in]
\item Evaluate $C(4)$ and interpret the meaning in the context of this problem.\\[1in]
\end{enumerate}




\item The data given in the table represent the age and systolic blood pressure for a sample of 12 randomly selected healthy adults.\\
\begin{table}[h]
\centering

\begin{tabular}{|l|l|}
\hline
Age in years & Pressure in mmHG \\ \hline
17           & 110              \\ \hline
21           & 118              \\ \hline
26           & 120              \\ \hline
32           & 121              \\ \hline
35           & 115              \\ \hline
37           & 124              \\ \hline
43           & 126              \\ \hline
51           & 130              \\ \hline
58           & 132              \\ \hline
59           & 139              \\ \hline
65           & 137              \\ \hline
68           & 141              \\ \hline
\end{tabular}
\end{table}

\newpage
\begin{enumerate}
\item Suppose $x$ represents the age of an adult, in years, and $y$ represents the systolic blood pressure, in mmHG.  Use the points $(21, 118)$ and $(51, 130)$ to write a linear model relating $y$ as a function of $x$.\\[2.5in]
\item Interpret the meaning of the slope in the context of this problem.\\[.5in]
\item Use the model to estimate the systolic blood pressure for a 55 year old.  Round to the nearest whole unit.\\[2in]
\end{enumerate}



\end{enumerate}

\noindent \textbf{Student Learning Outcomes Check}

\begin{enumerate}
\item Can you interpret the point-slope formula?
\item Are you able to construct a linear equation from a written description?
\end{enumerate}

\noindent \textbf{If any of your answers were no, please ask about these topics in class.}

