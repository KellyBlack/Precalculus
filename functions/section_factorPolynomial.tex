%=========================================================================
% Start of activity on factoring polynomials
%=========================================================================
\preClass{Graphs of Polynomials}

\begin{problem}
\item Given the graph of the polynomial below determine one possible
  definition of the polynomial.

  \vfill
\end{problem}


\actTitle{Factoring Polynomials}
\begin{problem}
\item For each function below express the result as a polynomial plus
  a remainder.
  \begin{subproblem}
  \item ${\displaystyle x }$.
    \vfill
  \item ${\displaystyle x }$.
    \vfill
  \item ${\displaystyle x }$.
    \vfill
  \end{subproblem}

  \clearpage
  
\item Make a rough sketch of the polynomial
  \begin{eqnarray*}
    h(x) & = & 
  \end{eqnarray*}

  \begin{tikzpicture}[y=1.1cm, x=1.1cm,font=\sffamily]
    % bounds
    \def\lowX{-5.5}
    \pgfmathtruncatemacro\startX{round(0.5+\lowX)}
    \pgfmathsetmacro\nextXValue{int(\startX+1)}
    \def\highX{5.5}
    \def\lowY{-5.5}
    \def\highY{5.5}
    \pgfmathsetmacro\nextYValue{int(\lowY+1)}
    % ticks
    \draw[step = 1, gray, very thin,dashed,opacity=0.85] (\lowX, \lowY) grid ( \highX,\highY);
    % axis
    \draw[thick,->] (\lowX,0) -- coordinate (x axis mid) (\highX,0) node[anchor = north west] {$x$};
    \draw[thick,->] (0,\lowY) -- coordinate (y axis mid) (0,\highY) node[anchor = south east] {$y$};
    \foreach \y in {-5,-4,...,-1,1,2,...,\highY} {
      \draw (1pt, \y) -- (-1pt, \y) node[yshift=-6,xshift=-1,anchor=east] {$\y$};
    }
    \foreach \x in {-5,-4,...,-1,1,2,...,\highX} {
      \draw (\x,1pt) -- (\x,-1pt) node[yshift=-5,xshift=-1,anchor=east] {$\x$};
    }
    \draw (0,6.0) node [anchor=south] {Comparing Shifted Functions};
  \end{tikzpicture}

  \clearpage

\item Given the graph of a polynomial below, determine the smallest
  possible degree of the polynomial.


  \clearpage

\item 
  \begin{subproblem}
  \item 
    \vfill
  \item 
    \vfill
  \end{subproblem}

\end{problem}

\postClass

\begin{problem}
\item Briefly state two ideas from today's class.
  \begin{itemize}
  \item 
  \item 
  \end{itemize}
\item 
  \begin{subproblem}
    \item
  \end{subproblem}
\end{problem}


%%% Local Variables:
%%% mode: latex
%%% TeX-master: "../labManual"
%%% End:

