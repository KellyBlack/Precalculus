%=========================================================================
% Start of 
%=========================================================================
\preClass{Algebra With Exponentials}

\begin{problem}
\item Determine an approximation for the value of each expression
  below. Your approximation should be to the nearest 0.01.
  \begin{subproblem}
  \item $2^{3}$
    \vfill
  \item $2.5^{2}$
    \vfill
  \item $2.7^{1.5}$
    \vfill
  \item $2.718^{2.1}$
    \vfill
  \item $2.7183^{2.44}$
    \vfill
  \end{subproblem}
\end{problem}


\actTitle{The Natural Exponential}
\begin{problem}
\item A bank manager is considering the impact of different terms for
  an account that offers compounded interest. She assumes that the
  interest rate is a constant annual one percent rate and then checks
  to see what happens for different lengths of time between
  compounding. Assume that one dollar is initially deposited.
  \begin{subproblem}
  \item Determine the amount of money in the account after one hundred
    years, if the interest is compounded yearly.  \vfill
  \item Determine the amount of money in the account after 100 years,
    if the interest is compounded once every six months.  \vfill
  \item Determine the amount of money in the account after 100 years,
    if the interest is compounded once a month.
    \vfill
  \item Determine the amount of money in the account after 100 years,
    if the interest is compounded once a day.
    \vfill
  \item Determine the amount of money in the account after 100 years,
    if the interest is compounded twice  a day.
    \vfill    
  \item What is happening to the balance as the number of terms
    increases?
    \vfill
  \end{subproblem}
  \clearpage

\item Generalize the value found on the previous page.
  \begin{subproblem}
  \item Determine a formula for the balance for 1\% annual interest
    after 100 years if the interest is compounded $n$ times per year.
    \vfill
  \item Make a substitution, $u=100n$, in the previous
    expression. Write out the expression for the balance as a function
    of $u$.
    \vfill
  \item Determine the values of the balance for the following values
    of $u$.

    \begin{tabular}{l|l}
      $u$ & balance \\ \hline
      1    &  \\ \hline
      2    &  \\ \hline
      12   &  \\ \hline
      365  &  \\ \hline
      1000 &  \\ \hline
    \end{tabular}

  \item What is the value approaching as $u$ gets large? This is a
    number that occurs in many situations, and we do not want to write
    it out every time we use it, so we use the symbol $e$ as a form of
    short hand notation.
    \begin{eqnarray*}
      e & \approx & 
    \end{eqnarray*}
    
  \end{subproblem}

  \clearpage

\item When interest is compounded continuously, the balance is
  determined using the function
  \begin{eqnarray*}
    \mathrm{Balance}(t) & = & P e^{rt},
  \end{eqnarray*}
  where $P$ is the initial balance, $r$ is the annual interest rate,
  and $t$ is the time in years.

  \begin{subproblem}
  \item Sketch a plot of the balance over time if 1\$ is deposited with a
    rate of 1\%.

    \vfill

  \item What happens to the balance as time increases?

    \vspace{3em}

  \item What would happen to the graph if you make $r$ bigger? What if
    $r$ is smaller?

    \vspace{3em}
  \end{subproblem}
\clearpage

\item As radioactive isotopes decay, the amount of isotope in a sample
  decreases. If the decay rate of an isotope is $r$ then the amount of
  an isotope in a sample is expressed using the function
  \begin{eqnarray*}
    \mathrm{Amount}(t) & = & A e^{-rt},
  \end{eqnarray*}
  where $t$ is measured in years.

  \begin{subproblem}
  \item What is the physical interpretation of the constant $A$?
    \vspace{2em}

  \item If a sample of a radioactive substance initially contains 3
    grams, and the radioactive decay is 0.00004, sketch a plot of the
    amount of the substance in the sample over time.

    \vfill

  \item What happens to the amount of the radioactive substance in the
    sample as time increases?
    
    \vspace{3em}

  \end{subproblem}

\clearpage


\end{problem}

\postClass

\begin{problem}
\item Briefly state two ideas from today's class.
  \begin{itemize}
  \item 
  \item 
  \end{itemize}
\item The average rate of change for a function from $x=a$ to $x=b$ is
  defined to be 
  \begin{eqnarray*}
    \mathrm{Avg.~Rate~of~Change} & = & \frac{f(b)-f(a)}{b-a}.
  \end{eqnarray*}
  \scalebox{0.65}{%% Creator: Matplotlib, PGF backend
%%
%% To include the figure in your LaTeX document, write
%%   \input{<filename>.pgf}
%%
%% Make sure the required packages are loaded in your preamble
%%   \usepackage{pgf}
%%
%% Figures using additional raster images can only be included by \input if
%% they are in the same directory as the main LaTeX file. For loading figures
%% from other directories you can use the `import` package
%%   \usepackage{import}
%% and then include the figures with
%%   \import{<path to file>}{<filename>.pgf}
%%
%% Matplotlib used the following preamble
%%   \usepackage{fontspec}
%%   \setmainfont{Bitstream Vera Serif}
%%   \setsansfont{Bitstream Vera Sans}
%%   \setmonofont{Bitstream Vera Sans Mono}
%%
\begingroup%
\makeatletter%
\begin{pgfpicture}%
\pgfpathrectangle{\pgfpointorigin}{\pgfqpoint{8.000000in}{6.000000in}}%
\pgfusepath{use as bounding box, clip}%
\begin{pgfscope}%
\pgfsetbuttcap%
\pgfsetmiterjoin%
\definecolor{currentfill}{rgb}{1.000000,1.000000,1.000000}%
\pgfsetfillcolor{currentfill}%
\pgfsetlinewidth{0.000000pt}%
\definecolor{currentstroke}{rgb}{1.000000,1.000000,1.000000}%
\pgfsetstrokecolor{currentstroke}%
\pgfsetdash{}{0pt}%
\pgfpathmoveto{\pgfqpoint{0.000000in}{0.000000in}}%
\pgfpathlineto{\pgfqpoint{8.000000in}{0.000000in}}%
\pgfpathlineto{\pgfqpoint{8.000000in}{6.000000in}}%
\pgfpathlineto{\pgfqpoint{0.000000in}{6.000000in}}%
\pgfpathclose%
\pgfusepath{fill}%
\end{pgfscope}%
\begin{pgfscope}%
\pgfsetbuttcap%
\pgfsetmiterjoin%
\definecolor{currentfill}{rgb}{1.000000,1.000000,1.000000}%
\pgfsetfillcolor{currentfill}%
\pgfsetlinewidth{0.000000pt}%
\definecolor{currentstroke}{rgb}{0.000000,0.000000,0.000000}%
\pgfsetstrokecolor{currentstroke}%
\pgfsetstrokeopacity{0.000000}%
\pgfsetdash{}{0pt}%
\pgfpathmoveto{\pgfqpoint{1.000000in}{0.600000in}}%
\pgfpathlineto{\pgfqpoint{7.200000in}{0.600000in}}%
\pgfpathlineto{\pgfqpoint{7.200000in}{5.400000in}}%
\pgfpathlineto{\pgfqpoint{1.000000in}{5.400000in}}%
\pgfpathclose%
\pgfusepath{fill}%
\end{pgfscope}%
\begin{pgfscope}%
\pgfpathrectangle{\pgfqpoint{1.000000in}{0.600000in}}{\pgfqpoint{6.200000in}{4.800000in}} %
\pgfusepath{clip}%
\pgfsetbuttcap%
\pgfsetmiterjoin%
\definecolor{currentfill}{rgb}{0.000000,0.000000,0.000000}%
\pgfsetfillcolor{currentfill}%
\pgfsetlinewidth{1.003750pt}%
\definecolor{currentstroke}{rgb}{0.000000,0.000000,0.000000}%
\pgfsetstrokecolor{currentstroke}%
\pgfsetdash{}{0pt}%
\pgfpathmoveto{\pgfqpoint{4.100000in}{3.526829in}}%
\pgfpathcurveto{\pgfqpoint{4.120052in}{3.526829in}}{\pgfqpoint{4.139285in}{3.532997in}}{\pgfqpoint{4.153464in}{3.543974in}}%
\pgfpathcurveto{\pgfqpoint{4.167643in}{3.554951in}}{\pgfqpoint{4.175610in}{3.569842in}}{\pgfqpoint{4.175610in}{3.585366in}}%
\pgfpathcurveto{\pgfqpoint{4.175610in}{3.600890in}}{\pgfqpoint{4.167643in}{3.615780in}}{\pgfqpoint{4.153464in}{3.626757in}}%
\pgfpathcurveto{\pgfqpoint{4.139285in}{3.637735in}}{\pgfqpoint{4.120052in}{3.643902in}}{\pgfqpoint{4.100000in}{3.643902in}}%
\pgfpathcurveto{\pgfqpoint{4.079948in}{3.643902in}}{\pgfqpoint{4.060715in}{3.637735in}}{\pgfqpoint{4.046536in}{3.626757in}}%
\pgfpathcurveto{\pgfqpoint{4.032357in}{3.615780in}}{\pgfqpoint{4.024390in}{3.600890in}}{\pgfqpoint{4.024390in}{3.585366in}}%
\pgfpathcurveto{\pgfqpoint{4.024390in}{3.569842in}}{\pgfqpoint{4.032357in}{3.554951in}}{\pgfqpoint{4.046536in}{3.543974in}}%
\pgfpathcurveto{\pgfqpoint{4.060715in}{3.532997in}}{\pgfqpoint{4.079948in}{3.526829in}}{\pgfqpoint{4.100000in}{3.526829in}}%
\pgfpathlineto{\pgfqpoint{4.100000in}{3.526829in}}%
\pgfusepath{stroke,fill}%
\end{pgfscope}%
\begin{pgfscope}%
\pgfpathrectangle{\pgfqpoint{1.000000in}{0.600000in}}{\pgfqpoint{6.200000in}{4.800000in}} %
\pgfusepath{clip}%
\pgfsetbuttcap%
\pgfsetmiterjoin%
\definecolor{currentfill}{rgb}{0.000000,0.000000,0.000000}%
\pgfsetfillcolor{currentfill}%
\pgfsetlinewidth{1.003750pt}%
\definecolor{currentstroke}{rgb}{0.000000,0.000000,0.000000}%
\pgfsetstrokecolor{currentstroke}%
\pgfsetdash{}{0pt}%
\pgfpathmoveto{\pgfqpoint{6.368293in}{4.532653in}}%
\pgfpathcurveto{\pgfqpoint{6.388345in}{4.532653in}}{\pgfqpoint{6.407578in}{4.538821in}}{\pgfqpoint{6.421757in}{4.549798in}}%
\pgfpathcurveto{\pgfqpoint{6.435936in}{4.560775in}}{\pgfqpoint{6.443902in}{4.575665in}}{\pgfqpoint{6.443902in}{4.591189in}}%
\pgfpathcurveto{\pgfqpoint{6.443902in}{4.606713in}}{\pgfqpoint{6.435936in}{4.621604in}}{\pgfqpoint{6.421757in}{4.632581in}}%
\pgfpathcurveto{\pgfqpoint{6.407578in}{4.643558in}}{\pgfqpoint{6.388345in}{4.649726in}}{\pgfqpoint{6.368293in}{4.649726in}}%
\pgfpathcurveto{\pgfqpoint{6.348241in}{4.649726in}}{\pgfqpoint{6.329007in}{4.643558in}}{\pgfqpoint{6.314829in}{4.632581in}}%
\pgfpathcurveto{\pgfqpoint{6.300650in}{4.621604in}}{\pgfqpoint{6.292683in}{4.606713in}}{\pgfqpoint{6.292683in}{4.591189in}}%
\pgfpathcurveto{\pgfqpoint{6.292683in}{4.575665in}}{\pgfqpoint{6.300650in}{4.560775in}}{\pgfqpoint{6.314829in}{4.549798in}}%
\pgfpathcurveto{\pgfqpoint{6.329007in}{4.538821in}}{\pgfqpoint{6.348241in}{4.532653in}}{\pgfqpoint{6.368293in}{4.532653in}}%
\pgfpathlineto{\pgfqpoint{6.368293in}{4.532653in}}%
\pgfusepath{stroke,fill}%
\end{pgfscope}%
\begin{pgfscope}%
\pgfpathrectangle{\pgfqpoint{1.000000in}{0.600000in}}{\pgfqpoint{6.200000in}{4.800000in}} %
\pgfusepath{clip}%
\pgfsetrectcap%
\pgfsetroundjoin%
\pgfsetlinewidth{2.007500pt}%
\definecolor{currentstroke}{rgb}{0.000000,0.000000,0.000000}%
\pgfsetstrokecolor{currentstroke}%
\pgfsetdash{}{0pt}%
\pgfpathmoveto{\pgfqpoint{1.075610in}{3.154301in}}%
\pgfpathlineto{\pgfqpoint{1.151220in}{3.159531in}}%
\pgfpathlineto{\pgfqpoint{1.226829in}{3.164938in}}%
\pgfpathlineto{\pgfqpoint{1.302439in}{3.170529in}}%
\pgfpathlineto{\pgfqpoint{1.378049in}{3.176309in}}%
\pgfpathlineto{\pgfqpoint{1.453659in}{3.182285in}}%
\pgfpathlineto{\pgfqpoint{1.529268in}{3.188463in}}%
\pgfpathlineto{\pgfqpoint{1.604878in}{3.194851in}}%
\pgfpathlineto{\pgfqpoint{1.680488in}{3.201456in}}%
\pgfpathlineto{\pgfqpoint{1.756098in}{3.208284in}}%
\pgfpathlineto{\pgfqpoint{1.831707in}{3.215344in}}%
\pgfpathlineto{\pgfqpoint{1.907317in}{3.222643in}}%
\pgfpathlineto{\pgfqpoint{1.982927in}{3.230190in}}%
\pgfpathlineto{\pgfqpoint{2.058537in}{3.237992in}}%
\pgfpathlineto{\pgfqpoint{2.134146in}{3.246059in}}%
\pgfpathlineto{\pgfqpoint{2.209756in}{3.254399in}}%
\pgfpathlineto{\pgfqpoint{2.285366in}{3.263022in}}%
\pgfpathlineto{\pgfqpoint{2.360976in}{3.271937in}}%
\pgfpathlineto{\pgfqpoint{2.436585in}{3.281154in}}%
\pgfpathlineto{\pgfqpoint{2.512195in}{3.290684in}}%
\pgfpathlineto{\pgfqpoint{2.587805in}{3.300537in}}%
\pgfpathlineto{\pgfqpoint{2.663415in}{3.310724in}}%
\pgfpathlineto{\pgfqpoint{2.739024in}{3.321256in}}%
\pgfpathlineto{\pgfqpoint{2.814634in}{3.332145in}}%
\pgfpathlineto{\pgfqpoint{2.890244in}{3.343403in}}%
\pgfpathlineto{\pgfqpoint{2.965854in}{3.355042in}}%
\pgfpathlineto{\pgfqpoint{3.041463in}{3.367077in}}%
\pgfpathlineto{\pgfqpoint{3.117073in}{3.379519in}}%
\pgfpathlineto{\pgfqpoint{3.192683in}{3.392382in}}%
\pgfpathlineto{\pgfqpoint{3.268293in}{3.405682in}}%
\pgfpathlineto{\pgfqpoint{3.343902in}{3.419433in}}%
\pgfpathlineto{\pgfqpoint{3.419512in}{3.433650in}}%
\pgfpathlineto{\pgfqpoint{3.495122in}{3.448348in}}%
\pgfpathlineto{\pgfqpoint{3.570732in}{3.463545in}}%
\pgfpathlineto{\pgfqpoint{3.646341in}{3.479257in}}%
\pgfpathlineto{\pgfqpoint{3.721951in}{3.495501in}}%
\pgfpathlineto{\pgfqpoint{3.797561in}{3.512297in}}%
\pgfpathlineto{\pgfqpoint{3.873171in}{3.529661in}}%
\pgfpathlineto{\pgfqpoint{3.948780in}{3.547614in}}%
\pgfpathlineto{\pgfqpoint{4.024390in}{3.566175in}}%
\pgfpathlineto{\pgfqpoint{4.100000in}{3.585366in}}%
\pgfpathlineto{\pgfqpoint{4.175610in}{3.605207in}}%
\pgfpathlineto{\pgfqpoint{4.251220in}{3.625720in}}%
\pgfpathlineto{\pgfqpoint{4.326829in}{3.646929in}}%
\pgfpathlineto{\pgfqpoint{4.402439in}{3.668857in}}%
\pgfpathlineto{\pgfqpoint{4.478049in}{3.691528in}}%
\pgfpathlineto{\pgfqpoint{4.553659in}{3.714967in}}%
\pgfpathlineto{\pgfqpoint{4.629268in}{3.739201in}}%
\pgfpathlineto{\pgfqpoint{4.704878in}{3.764257in}}%
\pgfpathlineto{\pgfqpoint{4.780488in}{3.790161in}}%
\pgfpathlineto{\pgfqpoint{4.856098in}{3.816944in}}%
\pgfpathlineto{\pgfqpoint{4.931707in}{3.844634in}}%
\pgfpathlineto{\pgfqpoint{5.007317in}{3.873263in}}%
\pgfpathlineto{\pgfqpoint{5.082927in}{3.902863in}}%
\pgfpathlineto{\pgfqpoint{5.158537in}{3.933465in}}%
\pgfpathlineto{\pgfqpoint{5.234146in}{3.965105in}}%
\pgfpathlineto{\pgfqpoint{5.309756in}{3.997817in}}%
\pgfpathlineto{\pgfqpoint{5.385366in}{4.031639in}}%
\pgfpathlineto{\pgfqpoint{5.460976in}{4.066606in}}%
\pgfpathlineto{\pgfqpoint{5.536585in}{4.102759in}}%
\pgfpathlineto{\pgfqpoint{5.612195in}{4.140137in}}%
\pgfpathlineto{\pgfqpoint{5.687805in}{4.178782in}}%
\pgfpathlineto{\pgfqpoint{5.763415in}{4.218737in}}%
\pgfpathlineto{\pgfqpoint{5.839024in}{4.260046in}}%
\pgfpathlineto{\pgfqpoint{5.914634in}{4.302756in}}%
\pgfpathlineto{\pgfqpoint{5.990244in}{4.346913in}}%
\pgfpathlineto{\pgfqpoint{6.065854in}{4.392566in}}%
\pgfpathlineto{\pgfqpoint{6.141463in}{4.439768in}}%
\pgfpathlineto{\pgfqpoint{6.217073in}{4.488569in}}%
\pgfpathlineto{\pgfqpoint{6.292683in}{4.539024in}}%
\pgfpathlineto{\pgfqpoint{6.368293in}{4.591189in}}%
\pgfpathlineto{\pgfqpoint{6.443902in}{4.645123in}}%
\pgfpathlineto{\pgfqpoint{6.519512in}{4.700885in}}%
\pgfpathlineto{\pgfqpoint{6.595122in}{4.758536in}}%
\pgfpathlineto{\pgfqpoint{6.670732in}{4.818142in}}%
\pgfpathlineto{\pgfqpoint{6.746341in}{4.879768in}}%
\pgfpathlineto{\pgfqpoint{6.821951in}{4.943483in}}%
\pgfpathlineto{\pgfqpoint{6.897561in}{5.009358in}}%
\pgfpathlineto{\pgfqpoint{6.973171in}{5.077465in}}%
\pgfpathlineto{\pgfqpoint{7.048780in}{5.147881in}}%
\pgfpathlineto{\pgfqpoint{7.124390in}{5.220684in}}%
\pgfusepath{stroke}%
\end{pgfscope}%
\begin{pgfscope}%
\pgfpathrectangle{\pgfqpoint{1.000000in}{0.600000in}}{\pgfqpoint{6.200000in}{4.800000in}} %
\pgfusepath{clip}%
\pgfsetbuttcap%
\pgfsetroundjoin%
\pgfsetlinewidth{2.007500pt}%
\definecolor{currentstroke}{rgb}{0.000000,0.000000,0.000000}%
\pgfsetstrokecolor{currentstroke}%
\pgfsetdash{{6.000000pt}{6.000000pt}}{0.000000pt}%
\pgfpathmoveto{\pgfqpoint{1.075610in}{2.244268in}}%
\pgfpathlineto{\pgfqpoint{7.124390in}{4.926464in}}%
\pgfusepath{stroke}%
\end{pgfscope}%
\begin{pgfscope}%
\pgfsetrectcap%
\pgfsetmiterjoin%
\pgfsetlinewidth{0.000000pt}%
\definecolor{currentstroke}{rgb}{0.000000,0.000000,0.000000}%
\pgfsetstrokecolor{currentstroke}%
\pgfsetstrokeopacity{0.000000}%
\pgfsetdash{}{0pt}%
\pgfpathmoveto{\pgfqpoint{1.000000in}{5.400000in}}%
\pgfpathlineto{\pgfqpoint{7.200000in}{5.400000in}}%
\pgfusepath{}%
\end{pgfscope}%
\begin{pgfscope}%
\pgfsetrectcap%
\pgfsetmiterjoin%
\pgfsetlinewidth{0.000000pt}%
\definecolor{currentstroke}{rgb}{0.000000,0.000000,0.000000}%
\pgfsetstrokecolor{currentstroke}%
\pgfsetstrokeopacity{0.000000}%
\pgfsetdash{}{0pt}%
\pgfpathmoveto{\pgfqpoint{7.200000in}{0.600000in}}%
\pgfpathlineto{\pgfqpoint{7.200000in}{5.400000in}}%
\pgfusepath{}%
\end{pgfscope}%
\begin{pgfscope}%
\pgfsetrectcap%
\pgfsetmiterjoin%
\pgfsetlinewidth{1.003750pt}%
\definecolor{currentstroke}{rgb}{0.000000,0.000000,0.000000}%
\pgfsetstrokecolor{currentstroke}%
\pgfsetdash{}{0pt}%
\pgfpathmoveto{\pgfqpoint{1.000000in}{3.000000in}}%
\pgfpathlineto{\pgfqpoint{7.200000in}{3.000000in}}%
\pgfusepath{stroke}%
\end{pgfscope}%
\begin{pgfscope}%
\pgfsetrectcap%
\pgfsetmiterjoin%
\pgfsetlinewidth{1.003750pt}%
\definecolor{currentstroke}{rgb}{0.000000,0.000000,0.000000}%
\pgfsetstrokecolor{currentstroke}%
\pgfsetdash{}{0pt}%
\pgfpathmoveto{\pgfqpoint{4.100000in}{0.600000in}}%
\pgfpathlineto{\pgfqpoint{4.100000in}{5.400000in}}%
\pgfusepath{stroke}%
\end{pgfscope}%
\begin{pgfscope}%
\pgfsetbuttcap%
\pgfsetroundjoin%
\pgfsetlinewidth{0.501875pt}%
\definecolor{currentstroke}{rgb}{0.000000,0.000000,0.000000}%
\pgfsetstrokecolor{currentstroke}%
\pgfsetdash{{1.000000pt}{3.000000pt}}{0.000000pt}%
\pgfpathmoveto{\pgfqpoint{1.075610in}{0.600000in}}%
\pgfpathlineto{\pgfqpoint{1.075610in}{5.400000in}}%
\pgfusepath{stroke}%
\end{pgfscope}%
\begin{pgfscope}%
\pgfsetbuttcap%
\pgfsetroundjoin%
\definecolor{currentfill}{rgb}{0.000000,0.000000,0.000000}%
\pgfsetfillcolor{currentfill}%
\pgfsetlinewidth{0.501875pt}%
\definecolor{currentstroke}{rgb}{0.000000,0.000000,0.000000}%
\pgfsetstrokecolor{currentstroke}%
\pgfsetdash{}{0pt}%
\pgfsys@defobject{currentmarker}{\pgfqpoint{0.000000in}{0.000000in}}{\pgfqpoint{0.000000in}{0.055556in}}{%
\pgfpathmoveto{\pgfqpoint{0.000000in}{0.000000in}}%
\pgfpathlineto{\pgfqpoint{0.000000in}{0.055556in}}%
\pgfusepath{stroke,fill}%
}%
\begin{pgfscope}%
\pgfsys@transformshift{1.075610in}{3.000000in}%
\pgfsys@useobject{currentmarker}{}%
\end{pgfscope}%
\end{pgfscope}%
\begin{pgfscope}%
\pgftext[x=1.075610in,y=2.944444in,,top]{\sffamily\fontsize{12.000000}{14.400000}\selectfont \(\displaystyle -4\)}%
\end{pgfscope}%
\begin{pgfscope}%
\pgfsetbuttcap%
\pgfsetroundjoin%
\pgfsetlinewidth{0.501875pt}%
\definecolor{currentstroke}{rgb}{0.000000,0.000000,0.000000}%
\pgfsetstrokecolor{currentstroke}%
\pgfsetdash{{1.000000pt}{3.000000pt}}{0.000000pt}%
\pgfpathmoveto{\pgfqpoint{1.831707in}{0.600000in}}%
\pgfpathlineto{\pgfqpoint{1.831707in}{5.400000in}}%
\pgfusepath{stroke}%
\end{pgfscope}%
\begin{pgfscope}%
\pgfsetbuttcap%
\pgfsetroundjoin%
\definecolor{currentfill}{rgb}{0.000000,0.000000,0.000000}%
\pgfsetfillcolor{currentfill}%
\pgfsetlinewidth{0.501875pt}%
\definecolor{currentstroke}{rgb}{0.000000,0.000000,0.000000}%
\pgfsetstrokecolor{currentstroke}%
\pgfsetdash{}{0pt}%
\pgfsys@defobject{currentmarker}{\pgfqpoint{0.000000in}{0.000000in}}{\pgfqpoint{0.000000in}{0.055556in}}{%
\pgfpathmoveto{\pgfqpoint{0.000000in}{0.000000in}}%
\pgfpathlineto{\pgfqpoint{0.000000in}{0.055556in}}%
\pgfusepath{stroke,fill}%
}%
\begin{pgfscope}%
\pgfsys@transformshift{1.831707in}{3.000000in}%
\pgfsys@useobject{currentmarker}{}%
\end{pgfscope}%
\end{pgfscope}%
\begin{pgfscope}%
\pgftext[x=1.831707in,y=2.944444in,,top]{\sffamily\fontsize{12.000000}{14.400000}\selectfont \(\displaystyle -3\)}%
\end{pgfscope}%
\begin{pgfscope}%
\pgfsetbuttcap%
\pgfsetroundjoin%
\pgfsetlinewidth{0.501875pt}%
\definecolor{currentstroke}{rgb}{0.000000,0.000000,0.000000}%
\pgfsetstrokecolor{currentstroke}%
\pgfsetdash{{1.000000pt}{3.000000pt}}{0.000000pt}%
\pgfpathmoveto{\pgfqpoint{2.587805in}{0.600000in}}%
\pgfpathlineto{\pgfqpoint{2.587805in}{5.400000in}}%
\pgfusepath{stroke}%
\end{pgfscope}%
\begin{pgfscope}%
\pgfsetbuttcap%
\pgfsetroundjoin%
\definecolor{currentfill}{rgb}{0.000000,0.000000,0.000000}%
\pgfsetfillcolor{currentfill}%
\pgfsetlinewidth{0.501875pt}%
\definecolor{currentstroke}{rgb}{0.000000,0.000000,0.000000}%
\pgfsetstrokecolor{currentstroke}%
\pgfsetdash{}{0pt}%
\pgfsys@defobject{currentmarker}{\pgfqpoint{0.000000in}{0.000000in}}{\pgfqpoint{0.000000in}{0.055556in}}{%
\pgfpathmoveto{\pgfqpoint{0.000000in}{0.000000in}}%
\pgfpathlineto{\pgfqpoint{0.000000in}{0.055556in}}%
\pgfusepath{stroke,fill}%
}%
\begin{pgfscope}%
\pgfsys@transformshift{2.587805in}{3.000000in}%
\pgfsys@useobject{currentmarker}{}%
\end{pgfscope}%
\end{pgfscope}%
\begin{pgfscope}%
\pgftext[x=2.587805in,y=2.944444in,,top]{\sffamily\fontsize{12.000000}{14.400000}\selectfont \(\displaystyle -2\)}%
\end{pgfscope}%
\begin{pgfscope}%
\pgfsetbuttcap%
\pgfsetroundjoin%
\pgfsetlinewidth{0.501875pt}%
\definecolor{currentstroke}{rgb}{0.000000,0.000000,0.000000}%
\pgfsetstrokecolor{currentstroke}%
\pgfsetdash{{1.000000pt}{3.000000pt}}{0.000000pt}%
\pgfpathmoveto{\pgfqpoint{3.343902in}{0.600000in}}%
\pgfpathlineto{\pgfqpoint{3.343902in}{5.400000in}}%
\pgfusepath{stroke}%
\end{pgfscope}%
\begin{pgfscope}%
\pgfsetbuttcap%
\pgfsetroundjoin%
\definecolor{currentfill}{rgb}{0.000000,0.000000,0.000000}%
\pgfsetfillcolor{currentfill}%
\pgfsetlinewidth{0.501875pt}%
\definecolor{currentstroke}{rgb}{0.000000,0.000000,0.000000}%
\pgfsetstrokecolor{currentstroke}%
\pgfsetdash{}{0pt}%
\pgfsys@defobject{currentmarker}{\pgfqpoint{0.000000in}{0.000000in}}{\pgfqpoint{0.000000in}{0.055556in}}{%
\pgfpathmoveto{\pgfqpoint{0.000000in}{0.000000in}}%
\pgfpathlineto{\pgfqpoint{0.000000in}{0.055556in}}%
\pgfusepath{stroke,fill}%
}%
\begin{pgfscope}%
\pgfsys@transformshift{3.343902in}{3.000000in}%
\pgfsys@useobject{currentmarker}{}%
\end{pgfscope}%
\end{pgfscope}%
\begin{pgfscope}%
\pgftext[x=3.343902in,y=2.944444in,,top]{\sffamily\fontsize{12.000000}{14.400000}\selectfont \(\displaystyle -1\)}%
\end{pgfscope}%
\begin{pgfscope}%
\pgfsetbuttcap%
\pgfsetroundjoin%
\pgfsetlinewidth{0.501875pt}%
\definecolor{currentstroke}{rgb}{0.000000,0.000000,0.000000}%
\pgfsetstrokecolor{currentstroke}%
\pgfsetdash{{1.000000pt}{3.000000pt}}{0.000000pt}%
\pgfpathmoveto{\pgfqpoint{4.100000in}{0.600000in}}%
\pgfpathlineto{\pgfqpoint{4.100000in}{5.400000in}}%
\pgfusepath{stroke}%
\end{pgfscope}%
\begin{pgfscope}%
\pgfsetbuttcap%
\pgfsetroundjoin%
\definecolor{currentfill}{rgb}{0.000000,0.000000,0.000000}%
\pgfsetfillcolor{currentfill}%
\pgfsetlinewidth{0.501875pt}%
\definecolor{currentstroke}{rgb}{0.000000,0.000000,0.000000}%
\pgfsetstrokecolor{currentstroke}%
\pgfsetdash{}{0pt}%
\pgfsys@defobject{currentmarker}{\pgfqpoint{0.000000in}{0.000000in}}{\pgfqpoint{0.000000in}{0.055556in}}{%
\pgfpathmoveto{\pgfqpoint{0.000000in}{0.000000in}}%
\pgfpathlineto{\pgfqpoint{0.000000in}{0.055556in}}%
\pgfusepath{stroke,fill}%
}%
\begin{pgfscope}%
\pgfsys@transformshift{4.100000in}{3.000000in}%
\pgfsys@useobject{currentmarker}{}%
\end{pgfscope}%
\end{pgfscope}%
\begin{pgfscope}%
\pgftext[x=4.100000in,y=2.944444in,,top]{\sffamily\fontsize{12.000000}{14.400000}\selectfont \(\displaystyle 0\)}%
\end{pgfscope}%
\begin{pgfscope}%
\pgfsetbuttcap%
\pgfsetroundjoin%
\pgfsetlinewidth{0.501875pt}%
\definecolor{currentstroke}{rgb}{0.000000,0.000000,0.000000}%
\pgfsetstrokecolor{currentstroke}%
\pgfsetdash{{1.000000pt}{3.000000pt}}{0.000000pt}%
\pgfpathmoveto{\pgfqpoint{4.856098in}{0.600000in}}%
\pgfpathlineto{\pgfqpoint{4.856098in}{5.400000in}}%
\pgfusepath{stroke}%
\end{pgfscope}%
\begin{pgfscope}%
\pgfsetbuttcap%
\pgfsetroundjoin%
\definecolor{currentfill}{rgb}{0.000000,0.000000,0.000000}%
\pgfsetfillcolor{currentfill}%
\pgfsetlinewidth{0.501875pt}%
\definecolor{currentstroke}{rgb}{0.000000,0.000000,0.000000}%
\pgfsetstrokecolor{currentstroke}%
\pgfsetdash{}{0pt}%
\pgfsys@defobject{currentmarker}{\pgfqpoint{0.000000in}{0.000000in}}{\pgfqpoint{0.000000in}{0.055556in}}{%
\pgfpathmoveto{\pgfqpoint{0.000000in}{0.000000in}}%
\pgfpathlineto{\pgfqpoint{0.000000in}{0.055556in}}%
\pgfusepath{stroke,fill}%
}%
\begin{pgfscope}%
\pgfsys@transformshift{4.856098in}{3.000000in}%
\pgfsys@useobject{currentmarker}{}%
\end{pgfscope}%
\end{pgfscope}%
\begin{pgfscope}%
\pgftext[x=4.856098in,y=2.944444in,,top]{\sffamily\fontsize{12.000000}{14.400000}\selectfont \(\displaystyle 1\)}%
\end{pgfscope}%
\begin{pgfscope}%
\pgfsetbuttcap%
\pgfsetroundjoin%
\pgfsetlinewidth{0.501875pt}%
\definecolor{currentstroke}{rgb}{0.000000,0.000000,0.000000}%
\pgfsetstrokecolor{currentstroke}%
\pgfsetdash{{1.000000pt}{3.000000pt}}{0.000000pt}%
\pgfpathmoveto{\pgfqpoint{5.612195in}{0.600000in}}%
\pgfpathlineto{\pgfqpoint{5.612195in}{5.400000in}}%
\pgfusepath{stroke}%
\end{pgfscope}%
\begin{pgfscope}%
\pgfsetbuttcap%
\pgfsetroundjoin%
\definecolor{currentfill}{rgb}{0.000000,0.000000,0.000000}%
\pgfsetfillcolor{currentfill}%
\pgfsetlinewidth{0.501875pt}%
\definecolor{currentstroke}{rgb}{0.000000,0.000000,0.000000}%
\pgfsetstrokecolor{currentstroke}%
\pgfsetdash{}{0pt}%
\pgfsys@defobject{currentmarker}{\pgfqpoint{0.000000in}{0.000000in}}{\pgfqpoint{0.000000in}{0.055556in}}{%
\pgfpathmoveto{\pgfqpoint{0.000000in}{0.000000in}}%
\pgfpathlineto{\pgfqpoint{0.000000in}{0.055556in}}%
\pgfusepath{stroke,fill}%
}%
\begin{pgfscope}%
\pgfsys@transformshift{5.612195in}{3.000000in}%
\pgfsys@useobject{currentmarker}{}%
\end{pgfscope}%
\end{pgfscope}%
\begin{pgfscope}%
\pgftext[x=5.612195in,y=2.944444in,,top]{\sffamily\fontsize{12.000000}{14.400000}\selectfont \(\displaystyle 2\)}%
\end{pgfscope}%
\begin{pgfscope}%
\pgfsetbuttcap%
\pgfsetroundjoin%
\pgfsetlinewidth{0.501875pt}%
\definecolor{currentstroke}{rgb}{0.000000,0.000000,0.000000}%
\pgfsetstrokecolor{currentstroke}%
\pgfsetdash{{1.000000pt}{3.000000pt}}{0.000000pt}%
\pgfpathmoveto{\pgfqpoint{6.368293in}{0.600000in}}%
\pgfpathlineto{\pgfqpoint{6.368293in}{5.400000in}}%
\pgfusepath{stroke}%
\end{pgfscope}%
\begin{pgfscope}%
\pgfsetbuttcap%
\pgfsetroundjoin%
\definecolor{currentfill}{rgb}{0.000000,0.000000,0.000000}%
\pgfsetfillcolor{currentfill}%
\pgfsetlinewidth{0.501875pt}%
\definecolor{currentstroke}{rgb}{0.000000,0.000000,0.000000}%
\pgfsetstrokecolor{currentstroke}%
\pgfsetdash{}{0pt}%
\pgfsys@defobject{currentmarker}{\pgfqpoint{0.000000in}{0.000000in}}{\pgfqpoint{0.000000in}{0.055556in}}{%
\pgfpathmoveto{\pgfqpoint{0.000000in}{0.000000in}}%
\pgfpathlineto{\pgfqpoint{0.000000in}{0.055556in}}%
\pgfusepath{stroke,fill}%
}%
\begin{pgfscope}%
\pgfsys@transformshift{6.368293in}{3.000000in}%
\pgfsys@useobject{currentmarker}{}%
\end{pgfscope}%
\end{pgfscope}%
\begin{pgfscope}%
\pgftext[x=6.368293in,y=2.944444in,,top]{\sffamily\fontsize{12.000000}{14.400000}\selectfont \(\displaystyle 3\)}%
\end{pgfscope}%
\begin{pgfscope}%
\pgfsetbuttcap%
\pgfsetroundjoin%
\pgfsetlinewidth{0.501875pt}%
\definecolor{currentstroke}{rgb}{0.000000,0.000000,0.000000}%
\pgfsetstrokecolor{currentstroke}%
\pgfsetdash{{1.000000pt}{3.000000pt}}{0.000000pt}%
\pgfpathmoveto{\pgfqpoint{7.124390in}{0.600000in}}%
\pgfpathlineto{\pgfqpoint{7.124390in}{5.400000in}}%
\pgfusepath{stroke}%
\end{pgfscope}%
\begin{pgfscope}%
\pgfsetbuttcap%
\pgfsetroundjoin%
\definecolor{currentfill}{rgb}{0.000000,0.000000,0.000000}%
\pgfsetfillcolor{currentfill}%
\pgfsetlinewidth{0.501875pt}%
\definecolor{currentstroke}{rgb}{0.000000,0.000000,0.000000}%
\pgfsetstrokecolor{currentstroke}%
\pgfsetdash{}{0pt}%
\pgfsys@defobject{currentmarker}{\pgfqpoint{0.000000in}{0.000000in}}{\pgfqpoint{0.000000in}{0.055556in}}{%
\pgfpathmoveto{\pgfqpoint{0.000000in}{0.000000in}}%
\pgfpathlineto{\pgfqpoint{0.000000in}{0.055556in}}%
\pgfusepath{stroke,fill}%
}%
\begin{pgfscope}%
\pgfsys@transformshift{7.124390in}{3.000000in}%
\pgfsys@useobject{currentmarker}{}%
\end{pgfscope}%
\end{pgfscope}%
\begin{pgfscope}%
\pgftext[x=7.124390in,y=2.944444in,,top]{\sffamily\fontsize{12.000000}{14.400000}\selectfont \(\displaystyle 4\)}%
\end{pgfscope}%
\begin{pgfscope}%
\pgftext[x=6.890000in,y=2.760000in,,top]{\sffamily\fontsize{12.000000}{14.400000}\selectfont x}%
\end{pgfscope}%
\begin{pgfscope}%
\pgfsetbuttcap%
\pgfsetroundjoin%
\pgfsetlinewidth{0.501875pt}%
\definecolor{currentstroke}{rgb}{0.000000,0.000000,0.000000}%
\pgfsetstrokecolor{currentstroke}%
\pgfsetdash{{1.000000pt}{3.000000pt}}{0.000000pt}%
\pgfpathmoveto{\pgfqpoint{1.000000in}{0.658537in}}%
\pgfpathlineto{\pgfqpoint{7.200000in}{0.658537in}}%
\pgfusepath{stroke}%
\end{pgfscope}%
\begin{pgfscope}%
\pgfsetbuttcap%
\pgfsetroundjoin%
\definecolor{currentfill}{rgb}{0.000000,0.000000,0.000000}%
\pgfsetfillcolor{currentfill}%
\pgfsetlinewidth{0.501875pt}%
\definecolor{currentstroke}{rgb}{0.000000,0.000000,0.000000}%
\pgfsetstrokecolor{currentstroke}%
\pgfsetdash{}{0pt}%
\pgfsys@defobject{currentmarker}{\pgfqpoint{0.000000in}{0.000000in}}{\pgfqpoint{0.055556in}{0.000000in}}{%
\pgfpathmoveto{\pgfqpoint{0.000000in}{0.000000in}}%
\pgfpathlineto{\pgfqpoint{0.055556in}{0.000000in}}%
\pgfusepath{stroke,fill}%
}%
\begin{pgfscope}%
\pgfsys@transformshift{4.100000in}{0.658537in}%
\pgfsys@useobject{currentmarker}{}%
\end{pgfscope}%
\end{pgfscope}%
\begin{pgfscope}%
\pgftext[x=4.044444in,y=0.658537in,right,]{\sffamily\fontsize{12.000000}{14.400000}\selectfont \(\displaystyle -4\)}%
\end{pgfscope}%
\begin{pgfscope}%
\pgfsetbuttcap%
\pgfsetroundjoin%
\pgfsetlinewidth{0.501875pt}%
\definecolor{currentstroke}{rgb}{0.000000,0.000000,0.000000}%
\pgfsetstrokecolor{currentstroke}%
\pgfsetdash{{1.000000pt}{3.000000pt}}{0.000000pt}%
\pgfpathmoveto{\pgfqpoint{1.000000in}{1.243902in}}%
\pgfpathlineto{\pgfqpoint{7.200000in}{1.243902in}}%
\pgfusepath{stroke}%
\end{pgfscope}%
\begin{pgfscope}%
\pgfsetbuttcap%
\pgfsetroundjoin%
\definecolor{currentfill}{rgb}{0.000000,0.000000,0.000000}%
\pgfsetfillcolor{currentfill}%
\pgfsetlinewidth{0.501875pt}%
\definecolor{currentstroke}{rgb}{0.000000,0.000000,0.000000}%
\pgfsetstrokecolor{currentstroke}%
\pgfsetdash{}{0pt}%
\pgfsys@defobject{currentmarker}{\pgfqpoint{0.000000in}{0.000000in}}{\pgfqpoint{0.055556in}{0.000000in}}{%
\pgfpathmoveto{\pgfqpoint{0.000000in}{0.000000in}}%
\pgfpathlineto{\pgfqpoint{0.055556in}{0.000000in}}%
\pgfusepath{stroke,fill}%
}%
\begin{pgfscope}%
\pgfsys@transformshift{4.100000in}{1.243902in}%
\pgfsys@useobject{currentmarker}{}%
\end{pgfscope}%
\end{pgfscope}%
\begin{pgfscope}%
\pgftext[x=4.044444in,y=1.243902in,right,]{\sffamily\fontsize{12.000000}{14.400000}\selectfont \(\displaystyle -3\)}%
\end{pgfscope}%
\begin{pgfscope}%
\pgfsetbuttcap%
\pgfsetroundjoin%
\pgfsetlinewidth{0.501875pt}%
\definecolor{currentstroke}{rgb}{0.000000,0.000000,0.000000}%
\pgfsetstrokecolor{currentstroke}%
\pgfsetdash{{1.000000pt}{3.000000pt}}{0.000000pt}%
\pgfpathmoveto{\pgfqpoint{1.000000in}{1.829268in}}%
\pgfpathlineto{\pgfqpoint{7.200000in}{1.829268in}}%
\pgfusepath{stroke}%
\end{pgfscope}%
\begin{pgfscope}%
\pgfsetbuttcap%
\pgfsetroundjoin%
\definecolor{currentfill}{rgb}{0.000000,0.000000,0.000000}%
\pgfsetfillcolor{currentfill}%
\pgfsetlinewidth{0.501875pt}%
\definecolor{currentstroke}{rgb}{0.000000,0.000000,0.000000}%
\pgfsetstrokecolor{currentstroke}%
\pgfsetdash{}{0pt}%
\pgfsys@defobject{currentmarker}{\pgfqpoint{0.000000in}{0.000000in}}{\pgfqpoint{0.055556in}{0.000000in}}{%
\pgfpathmoveto{\pgfqpoint{0.000000in}{0.000000in}}%
\pgfpathlineto{\pgfqpoint{0.055556in}{0.000000in}}%
\pgfusepath{stroke,fill}%
}%
\begin{pgfscope}%
\pgfsys@transformshift{4.100000in}{1.829268in}%
\pgfsys@useobject{currentmarker}{}%
\end{pgfscope}%
\end{pgfscope}%
\begin{pgfscope}%
\pgftext[x=4.044444in,y=1.829268in,right,]{\sffamily\fontsize{12.000000}{14.400000}\selectfont \(\displaystyle -2\)}%
\end{pgfscope}%
\begin{pgfscope}%
\pgfsetbuttcap%
\pgfsetroundjoin%
\pgfsetlinewidth{0.501875pt}%
\definecolor{currentstroke}{rgb}{0.000000,0.000000,0.000000}%
\pgfsetstrokecolor{currentstroke}%
\pgfsetdash{{1.000000pt}{3.000000pt}}{0.000000pt}%
\pgfpathmoveto{\pgfqpoint{1.000000in}{2.414634in}}%
\pgfpathlineto{\pgfqpoint{7.200000in}{2.414634in}}%
\pgfusepath{stroke}%
\end{pgfscope}%
\begin{pgfscope}%
\pgfsetbuttcap%
\pgfsetroundjoin%
\definecolor{currentfill}{rgb}{0.000000,0.000000,0.000000}%
\pgfsetfillcolor{currentfill}%
\pgfsetlinewidth{0.501875pt}%
\definecolor{currentstroke}{rgb}{0.000000,0.000000,0.000000}%
\pgfsetstrokecolor{currentstroke}%
\pgfsetdash{}{0pt}%
\pgfsys@defobject{currentmarker}{\pgfqpoint{0.000000in}{0.000000in}}{\pgfqpoint{0.055556in}{0.000000in}}{%
\pgfpathmoveto{\pgfqpoint{0.000000in}{0.000000in}}%
\pgfpathlineto{\pgfqpoint{0.055556in}{0.000000in}}%
\pgfusepath{stroke,fill}%
}%
\begin{pgfscope}%
\pgfsys@transformshift{4.100000in}{2.414634in}%
\pgfsys@useobject{currentmarker}{}%
\end{pgfscope}%
\end{pgfscope}%
\begin{pgfscope}%
\pgftext[x=4.044444in,y=2.414634in,right,]{\sffamily\fontsize{12.000000}{14.400000}\selectfont \(\displaystyle -1\)}%
\end{pgfscope}%
\begin{pgfscope}%
\pgfsetbuttcap%
\pgfsetroundjoin%
\pgfsetlinewidth{0.501875pt}%
\definecolor{currentstroke}{rgb}{0.000000,0.000000,0.000000}%
\pgfsetstrokecolor{currentstroke}%
\pgfsetdash{{1.000000pt}{3.000000pt}}{0.000000pt}%
\pgfpathmoveto{\pgfqpoint{1.000000in}{3.000000in}}%
\pgfpathlineto{\pgfqpoint{7.200000in}{3.000000in}}%
\pgfusepath{stroke}%
\end{pgfscope}%
\begin{pgfscope}%
\pgfsetbuttcap%
\pgfsetroundjoin%
\definecolor{currentfill}{rgb}{0.000000,0.000000,0.000000}%
\pgfsetfillcolor{currentfill}%
\pgfsetlinewidth{0.501875pt}%
\definecolor{currentstroke}{rgb}{0.000000,0.000000,0.000000}%
\pgfsetstrokecolor{currentstroke}%
\pgfsetdash{}{0pt}%
\pgfsys@defobject{currentmarker}{\pgfqpoint{0.000000in}{0.000000in}}{\pgfqpoint{0.055556in}{0.000000in}}{%
\pgfpathmoveto{\pgfqpoint{0.000000in}{0.000000in}}%
\pgfpathlineto{\pgfqpoint{0.055556in}{0.000000in}}%
\pgfusepath{stroke,fill}%
}%
\begin{pgfscope}%
\pgfsys@transformshift{4.100000in}{3.000000in}%
\pgfsys@useobject{currentmarker}{}%
\end{pgfscope}%
\end{pgfscope}%
\begin{pgfscope}%
\pgftext[x=4.044444in,y=3.000000in,right,]{\sffamily\fontsize{12.000000}{14.400000}\selectfont \(\displaystyle 0\)}%
\end{pgfscope}%
\begin{pgfscope}%
\pgfsetbuttcap%
\pgfsetroundjoin%
\pgfsetlinewidth{0.501875pt}%
\definecolor{currentstroke}{rgb}{0.000000,0.000000,0.000000}%
\pgfsetstrokecolor{currentstroke}%
\pgfsetdash{{1.000000pt}{3.000000pt}}{0.000000pt}%
\pgfpathmoveto{\pgfqpoint{1.000000in}{3.585366in}}%
\pgfpathlineto{\pgfqpoint{7.200000in}{3.585366in}}%
\pgfusepath{stroke}%
\end{pgfscope}%
\begin{pgfscope}%
\pgfsetbuttcap%
\pgfsetroundjoin%
\definecolor{currentfill}{rgb}{0.000000,0.000000,0.000000}%
\pgfsetfillcolor{currentfill}%
\pgfsetlinewidth{0.501875pt}%
\definecolor{currentstroke}{rgb}{0.000000,0.000000,0.000000}%
\pgfsetstrokecolor{currentstroke}%
\pgfsetdash{}{0pt}%
\pgfsys@defobject{currentmarker}{\pgfqpoint{0.000000in}{0.000000in}}{\pgfqpoint{0.055556in}{0.000000in}}{%
\pgfpathmoveto{\pgfqpoint{0.000000in}{0.000000in}}%
\pgfpathlineto{\pgfqpoint{0.055556in}{0.000000in}}%
\pgfusepath{stroke,fill}%
}%
\begin{pgfscope}%
\pgfsys@transformshift{4.100000in}{3.585366in}%
\pgfsys@useobject{currentmarker}{}%
\end{pgfscope}%
\end{pgfscope}%
\begin{pgfscope}%
\pgftext[x=4.044444in,y=3.585366in,right,]{\sffamily\fontsize{12.000000}{14.400000}\selectfont \(\displaystyle 1\)}%
\end{pgfscope}%
\begin{pgfscope}%
\pgfsetbuttcap%
\pgfsetroundjoin%
\pgfsetlinewidth{0.501875pt}%
\definecolor{currentstroke}{rgb}{0.000000,0.000000,0.000000}%
\pgfsetstrokecolor{currentstroke}%
\pgfsetdash{{1.000000pt}{3.000000pt}}{0.000000pt}%
\pgfpathmoveto{\pgfqpoint{1.000000in}{4.170732in}}%
\pgfpathlineto{\pgfqpoint{7.200000in}{4.170732in}}%
\pgfusepath{stroke}%
\end{pgfscope}%
\begin{pgfscope}%
\pgfsetbuttcap%
\pgfsetroundjoin%
\definecolor{currentfill}{rgb}{0.000000,0.000000,0.000000}%
\pgfsetfillcolor{currentfill}%
\pgfsetlinewidth{0.501875pt}%
\definecolor{currentstroke}{rgb}{0.000000,0.000000,0.000000}%
\pgfsetstrokecolor{currentstroke}%
\pgfsetdash{}{0pt}%
\pgfsys@defobject{currentmarker}{\pgfqpoint{0.000000in}{0.000000in}}{\pgfqpoint{0.055556in}{0.000000in}}{%
\pgfpathmoveto{\pgfqpoint{0.000000in}{0.000000in}}%
\pgfpathlineto{\pgfqpoint{0.055556in}{0.000000in}}%
\pgfusepath{stroke,fill}%
}%
\begin{pgfscope}%
\pgfsys@transformshift{4.100000in}{4.170732in}%
\pgfsys@useobject{currentmarker}{}%
\end{pgfscope}%
\end{pgfscope}%
\begin{pgfscope}%
\pgftext[x=4.044444in,y=4.170732in,right,]{\sffamily\fontsize{12.000000}{14.400000}\selectfont \(\displaystyle 2\)}%
\end{pgfscope}%
\begin{pgfscope}%
\pgfsetbuttcap%
\pgfsetroundjoin%
\pgfsetlinewidth{0.501875pt}%
\definecolor{currentstroke}{rgb}{0.000000,0.000000,0.000000}%
\pgfsetstrokecolor{currentstroke}%
\pgfsetdash{{1.000000pt}{3.000000pt}}{0.000000pt}%
\pgfpathmoveto{\pgfqpoint{1.000000in}{4.756098in}}%
\pgfpathlineto{\pgfqpoint{7.200000in}{4.756098in}}%
\pgfusepath{stroke}%
\end{pgfscope}%
\begin{pgfscope}%
\pgfsetbuttcap%
\pgfsetroundjoin%
\definecolor{currentfill}{rgb}{0.000000,0.000000,0.000000}%
\pgfsetfillcolor{currentfill}%
\pgfsetlinewidth{0.501875pt}%
\definecolor{currentstroke}{rgb}{0.000000,0.000000,0.000000}%
\pgfsetstrokecolor{currentstroke}%
\pgfsetdash{}{0pt}%
\pgfsys@defobject{currentmarker}{\pgfqpoint{0.000000in}{0.000000in}}{\pgfqpoint{0.055556in}{0.000000in}}{%
\pgfpathmoveto{\pgfqpoint{0.000000in}{0.000000in}}%
\pgfpathlineto{\pgfqpoint{0.055556in}{0.000000in}}%
\pgfusepath{stroke,fill}%
}%
\begin{pgfscope}%
\pgfsys@transformshift{4.100000in}{4.756098in}%
\pgfsys@useobject{currentmarker}{}%
\end{pgfscope}%
\end{pgfscope}%
\begin{pgfscope}%
\pgftext[x=4.044444in,y=4.756098in,right,]{\sffamily\fontsize{12.000000}{14.400000}\selectfont \(\displaystyle 3\)}%
\end{pgfscope}%
\begin{pgfscope}%
\pgfsetbuttcap%
\pgfsetroundjoin%
\pgfsetlinewidth{0.501875pt}%
\definecolor{currentstroke}{rgb}{0.000000,0.000000,0.000000}%
\pgfsetstrokecolor{currentstroke}%
\pgfsetdash{{1.000000pt}{3.000000pt}}{0.000000pt}%
\pgfpathmoveto{\pgfqpoint{1.000000in}{5.341463in}}%
\pgfpathlineto{\pgfqpoint{7.200000in}{5.341463in}}%
\pgfusepath{stroke}%
\end{pgfscope}%
\begin{pgfscope}%
\pgfsetbuttcap%
\pgfsetroundjoin%
\definecolor{currentfill}{rgb}{0.000000,0.000000,0.000000}%
\pgfsetfillcolor{currentfill}%
\pgfsetlinewidth{0.501875pt}%
\definecolor{currentstroke}{rgb}{0.000000,0.000000,0.000000}%
\pgfsetstrokecolor{currentstroke}%
\pgfsetdash{}{0pt}%
\pgfsys@defobject{currentmarker}{\pgfqpoint{0.000000in}{0.000000in}}{\pgfqpoint{0.055556in}{0.000000in}}{%
\pgfpathmoveto{\pgfqpoint{0.000000in}{0.000000in}}%
\pgfpathlineto{\pgfqpoint{0.055556in}{0.000000in}}%
\pgfusepath{stroke,fill}%
}%
\begin{pgfscope}%
\pgfsys@transformshift{4.100000in}{5.341463in}%
\pgfsys@useobject{currentmarker}{}%
\end{pgfscope}%
\end{pgfscope}%
\begin{pgfscope}%
\pgftext[x=4.044444in,y=5.341463in,right,]{\sffamily\fontsize{12.000000}{14.400000}\selectfont \(\displaystyle 4\)}%
\end{pgfscope}%
\begin{pgfscope}%
\pgftext[x=3.790000in,y=5.160000in,,bottom,rotate=90.000000]{\sffamily\fontsize{12.000000}{14.400000}\selectfont f}%
\end{pgfscope}%
\begin{pgfscope}%
\pgftext[x=4.100000in,y=5.469444in,,base]{\sffamily\fontsize{14.400000}{17.280000}\selectfont Graph of f}%
\end{pgfscope}%
\end{pgfpicture}%
\makeatother%
\endgroup%
}

  Visually, it can be thought of as the slope of the line that goes
  through the graph of the function at two points, $(a,f(a))$ and
  $(b,f(b))$.

  \begin{subproblem}
  \item The population of a colony of bacteria is two times its
    previous population every one hour. Determine the average rate of
    change from $a=0$ to the times
    given in the table below. \\
    \begin{tabular}{l|@{\hspace{3em}}l|@{\hspace{3em}}l|@{\hspace{3em}}l|@{\hspace{3em}}l|@{\hspace{3em}}l|@{\hspace{3em}}l}
      $b$ & 1 & $\frac{1}{2}$ & $\frac{1}{2^2}$ & $\frac{1}{2^3}$ &
      $\frac{1}{2^4}$ & $\frac{1}{2^5}$ \\ \hline
      Avg  &&&&& \\ 
      Rate &&&&& \\
      Change &&&&&
    \end{tabular}

    What number is the average rate of change approaching as $b$ gets
    close to zero?

  \item The population of a colony of bacteria is three times its
    previous population every one hour. Determine the average rate of
    change from $a=0$ to the times
    given in the table below. \\
    \begin{tabular}{l|@{\hspace{3em}}l|@{\hspace{3em}}l|@{\hspace{3em}}l|@{\hspace{3em}}l|@{\hspace{3em}}l|@{\hspace{3em}}l}
      $b$ & 1 & $\frac{1}{2}$ & $\frac{1}{2^2}$ & $\frac{1}{2^3}$ &
      $\frac{1}{2^4}$ & $\frac{1}{2^5}$ \\ \hline
      Avg  &&&&& \\ 
      Rate &&&&& \\
      Change &&&&&
    \end{tabular}

    What number is the average rate of change approaching as $b$ gets
    close to zero?

  \item The population of a colony of bacteria is 2.5 its previous
    population every one hour. Determine the average rate of change
    from $a=0$ to
    the times   given in the table below. \\
    \begin{tabular}{l|@{\hspace{3em}}l|@{\hspace{3em}}l|@{\hspace{3em}}l|@{\hspace{3em}}l|@{\hspace{3em}}l|@{\hspace{3em}}l}
      $b$ & 1 & $\frac{1}{2}$ & $\frac{1}{2^2}$ & $\frac{1}{2^3}$ &
      $\frac{1}{2^4}$ & $\frac{1}{2^5}$ \\ \hline
      Avg  &&&&& \\ 
      Rate &&&&& \\
      Change &&&&&
    \end{tabular}

    What number is the average rate of change approaching as $b$ gets
    close to zero?

  \item The population of a colony of bacteria is 2.7 its previous
    population every one hour. Determine the average rate of change
    from $a=0$ to
    the times   given in the table below. \\
    \begin{tabular}{l|@{\hspace{3em}}l|@{\hspace{3em}}l|@{\hspace{3em}}l|@{\hspace{3em}}l|@{\hspace{3em}}l|@{\hspace{3em}}l}
      $b$ & 1 & $\frac{1}{2}$ & $\frac{1}{2^2}$ & $\frac{1}{2^3}$ &
      $\frac{1}{2^4}$ & $\frac{1}{2^5}$ \\ \hline
      Avg  &&&&& \\ 
      Rate &&&&& \\
      Change &&&&&
    \end{tabular}

    What number is the average rate of change approaching as $b$ gets
    close to zero?

  \item Determine a value so that if a population multiplies its
    population by that number every hour the average rate of change
    approaches 1 from $a=0$ and $b$ gets close to zero.

  \end{subproblem}

\end{problem}


%%% Local Variables:
%%% mode: latex
%%% TeX-master: "../labManual"
%%% End:

