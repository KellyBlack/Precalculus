%=========================================================================
% Start of
%=========================================================================
\preClass{Introduction to Functions}

\begin{problem}
\item A biologist grows four different colonies of bacteria. The
  number of bacteria in the colonies is estimated to be 10,000,
  20,000, 30,000, and 40,000. The mass for each colony is measured and
  is estimated to be $2.61\times 10^{-6}$, $5.20\times 10^{-6}$,
  $7.85\times 10^{-6}$ and $1.043\times 10^{-5}$ grams respectively.

  Organize the information above into a table so that the mass can be
  more easily determined given the number of bacteria in the
  colony. Also, graph each point as a coordinate where the number of
  bacteria is on the horizontal axis, and the mass is on the vertical
  axis.
  \sideNote{Label your axes and properly annotate your plot.}

  \vfill

\item What is the change in mass when the number of bacteria increases
  by 10,000?

  \vfill

\item Make rough estimate for a relationship that will provide a
  prediction for the mass of a colony given the number of bacteria
  within it.

  \vfill


  \vspace{5em}


\end{problem}


\actTitle{Functions}
\begin{problem}
\item A balloon has a tether that is attached to the ground, and the
  tether can be extended or retracted as the balloon is raised or
  lowered. One end of the tether is attached to the ground 20m away
  from a point directly below the balloon, and the balloon moves
  straight up and down. If the length of the tether is $x$ meters what
  is the altitude of the balloon?
  \begin{subproblem}
    \item Sketch a diagram of the physical situation. Label the known and
      unknown quantities. Make up names for the quantities that you
      think you \textbf{may} need to use later.
      \sideNote{Assume that the balloon only moves up and down with no
        lateral motion.}
      \vfill
      \vfill
    \item Determine the important relationships between the known and
      unknown quantities.
      \vfill
    \item How can you use this relationship to determine the variable
      of interest?
      \vfill
    \item Determine the height of the balloon given the length of the
      tether.
      \vfill
    \item Determine the domain and range of the function that gives
      the height given the length of the tether.
      \vfill
  \end{subproblem}

  \clearpage

\item A park has two distinct areas separated by a river, and each
  area has its own population of mice.  The population East of the
  river is estimated to have 10,000 individuals at the beginning of
  the year, and each week it grows by a constant 200 individuals. The
  population West of the river is estimated to have 8,000 individuals
  at the beginning of the year, and each week it grows by a constant
  250 individuals.

  \begin{subproblem}
  \item Make a rough sketch of the number of mice in the two
    populations on the same graph. The horizontal axis should be the
    time from the beginning of the year in weeks.
    \sideNote{Label your axes and properly annotate your plot.}
    \vfill

  \item Describe what is happening to the two populations. Is there a
    time when the two populations are equal? If so when is it?
    \vfill

  \item Determine a formula for the total number of mice in the park
    at any week after the beginning of the year.
    \vfill
  \end{subproblem}

\clearpage

\item The mass of a sparrow is proportional to the cube of the length of its wing.
  \begin{subproblem}
  \item \label{sparrowWingArea} A sparrow is measured, and it has a
    wing length of 9cm and a mass of 30 grams.  Determine the mass of
    a sparrow whose wing length is 10cm.  

    \vfill

  \item \label{sparrowMass} It is estimated that the mass of a sparrow
    is 27 grams. Determine an estimate of its wing length.  

    \vfill

  \item Determine the function that returns the mass of a sparrow
    given the length of its wing.

    \vfill
  \end{subproblem}

\clearpage

\item A common task is to convert units. For each statement below
  determine the function that returns a quantity in the second unit
  given a quantity in the first units.  
  \begin{subproblem}
    \item One kilometer is approximately 0.62 miles.
      \vfill
    \item One meter is 100 centimeters.
      \vfill
    \item One US dollar is approximately 1.35 Canadian dollars.
      \vfill
  \end{subproblem}


\end{problem}

\postClass

\begin{problem}
\item Briefly state two ideas from today's class.
  \begin{itemize}
  \item
  \item
  \end{itemize}
\item The surface area of a sphere of radius $r$ is $4\pi r^2$, and
  the volume is $\frac{4}{3}\pi r^3$. Determine the equation for the
  surface area of a sphere given its volume.
\item In the Star Trek television series a ship's velocity is given in
  terms of its warp factor, $w$. According to
  wikipedia\footnote{\href{https://en.wikipedia.org/wiki/Warp_drive}{https://en.wikipedia.org/wiki/Warp\_drive} accessed June 2016},
  the actual speed is the warp factor cubed multiplied by the speed of
  light which is approximately $3.0\times 10^8$ m/s.
  \begin{subproblem}
  \item Determine the speed of a ship that is moving at warp factor 0.2.
  \item Determine the speed of a ship that is moving at warp factor
    2.5.
  \item Determine the speed of a ship that is moving at warp factor
    3.0.
  \item A ship is moving at warp factor 3.1. What warp factor would be
    required to double the ship's speed?
  \item A ship is moving at warp factor 4.1. What warp factor would be
    required to double the ship's speed?
  \item What is the general formula to determine the new warp factor
    required to double the speed given the current warp factor.
  \end{subproblem}

\item You watch a video from your favourite conspiracy theorist.  He
  says that scientists are supressing evidence about prehistoric
  sparrows.  He says that giant sparrows once existed whose wing
  length was 10 meters.  Use the results from exercises
  \ref{sparrowWingArea} and \ref{sparrowMass} to determine if this
  makes sense.  Based on your result write out the comment that you
  will post in the comments section in response to the video.

\end{problem}


%%% Local Variables:
%%% mode: latex
%%% TeX-master: "../labManual"
%%% End:
