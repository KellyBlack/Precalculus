%=========================================================================
% Start of activity on linear models
%=========================================================================
\preClass{Linear Models}

\begin{problem}
\item A chemical reaction has a reactant that is broken down by
  itself, and the resulting reaction produces two different
  products. It is estimated that for each gram of the reactant that
  $\frac{1}{3}$g of the first product is produced and $\frac{1}{6}$g
  of the original reactant remains. Everything else that remains is
  the second product.
  \begin{subproblem}
  \item If you start with four grams of reactant how many grams of the
    products and the reactants will you get?
    \vfill
  \item Determine the number of grams of the products and reactants
    will result when you start with  $x$ grams of reactant.
    \vfill
  \end{subproblem}
\end{problem}


\actTitle{Modeling With Linear Functions}
\begin{problem}
\item 
  \begin{subproblem}
    \item
  \end{subproblem}
\end{problem}

\postClass

\begin{problem}
\item Briefly state two ideas from today's class.
  \begin{itemize}
  \item 
  \item 
  \end{itemize}
\item 
  \begin{subproblem}
    \item
  \end{subproblem}
\end{problem}

