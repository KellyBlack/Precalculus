

\actTitle{Worksheet 3.5A}

\noindent \textbf{Instructions:}  Work together in groups of  3 or 4
to complete the following problems.\\

Student goals:
  \begin{itemize}
  \item Solve equations with multiple exponential terms.
  \item Determine inverses of complicated functions that have either
    exponential or logarithmic terms.
  \item Manipulate equations with exponentials and transform them into
    other forms. For example, transform an expression with exponential
    terms into a quadratic equation.
  \item Define relationships given written descriptions that include
    wither exponential or logarithmic terms.
  \item Convert and write any exponential function using base $e$.
  \item Determine the value of any or all parameters in a compound
    interest problem given a written description of the situation.
  \item Determine the value of any or all parameters in an exponential
    growth/decay problem given a written description of the
    situation. Be able to identify if a situation results in either
    growth or decay.
  \item Determine the value of any or all parameters given a written
    description of a logistic growth function.
  \end{itemize}


  \noindent
\fbox{
  \parbox{\dimexpr\linewidth}%
  {
    
    We have learned 3 techniques to solve exponential and logarithmic
    equations.
    \begin{description}
    \item[(1) Equivalence Property of Exponential Expressions:] If $b, x,$
      and $y$ are real numbers where $b>0, b \neq 1.$ Then
      $$b^x=b^y \text{ implies that } x=y.$$

    \item[(2) Equivalence Property of Logarithmic Expressions:] If $b, x,$
      and $y$ are positive real numbers with
      $$\log_b (x) = \log_b (y) \text{ implies that } x=y.$$

    \item[(3) Solving Exponential Equations by Using Logarithms:] Isolate
      the exponential expression on one side of the equation and take a
      logarithm of the same base on both sides.
    \end{description}

  }
}

\clearpage

\begin{enumerate}

\item  For each equation, determine the best
  technique to use to solve the equation.  \textbf{Do not solve the
    equations.}
\begin{enumerate}
\begin{multicols}{2}
\item $3^x=81$\\[.2in]
\item $\log(x^2+6x)=\log(7)$\\[.2in]
\item $\displaystyle 11^{3x+1}=\Big(\frac{1}{11}\Big)^{x-5}$\\[.2in]
\item $6^x=87$\\[.2in]
\columnbreak
\item $10^{3+4x}-8100=120,000$ \\[.2in]

\item $\log_4(3x+11)=\log_4(3-x)$\\[.2in]

\item $1024=19^x+4$\\[.2in]
\item $5^{2x+2}=625$\\[.2in]
\end{multicols}
\end{enumerate}


\item Solve the following equations using the appropriate techniques.
  Then check your answers.
\begin{enumerate}

\item $2^t=32$
  \vfill

\item $100^{3m-5}=1000^{3-m}$
\vfill

\clearpage
\item $2^x=70$
\vfill
\item $80=320e^{-0.5t}+3$
\vfill

\item $\log_3(12-x)=\log_3(x+6)$
\vfill

\item $\ln(w^2+7w)=\ln(18)+1$
\vfill


\end{enumerate}


\clearpage

\item Sometimes, we need to use other techniques and properties to
  solve exponential equations.  For example, when there are multiple
  exponential functions, it can be helpful to take $\ln$ or $\log$ of
  both sides.  Solve the equations.  Then check your answers.
\begin{enumerate}
\item $3^{6x+5}=5^{2x}$
\vfill

\item $2^{1-6x}=7^{3x+4}$
\vfill

\end{enumerate}
\clearpage

\item You may also notice that an exponential equation has the same
  form as a quadratic equation.  So you may need to substitute to
  solve the problem like a quadratic equation.  Solve the equations.
  Then check your answers.
\begin{enumerate}
\item $e^{2x}-9e^x-22=0$\vfill
\item $e^{2x}-6e^x-16=0$\vfill
\end{enumerate}


\end{enumerate}

\noindent \textbf{Once you have finished this worksheet, go back and
  solve the equations from \#1.}


\hwTitle{Section 3.5A}

\begin{enumerate}
\item Solve the following equations using the appropriate techniques.
  Then check your answers.
  \begin{enumerate}
  \item $\sqrt[3]{5}=5^x$
  \item $\displaystyle 7^{2p-3}=\Big(\frac{1}{49}\Big)^{p+1}$
  \item $801=23^y+6$
  \item $\ln(2x+1) = \ln(x)$
  \item $\log_9(1-2x) = \log_9(x) - 3$
  \end{enumerate}

\end{enumerate}
