%=========================================================================
% Start of 
%=========================================================================
\preClass{Inverse Trigonometric Functions}

\begin{problem}
\item Use the axes below to make a sketch of $\sin(x)$.

  \scalebox{0.65}{\input{trig/img/blankAxesCenteredZero.pgf}}

  \begin{subproblem}
    \item Is the sine function one-to-one? (Justify your answer!)
    \vspace{3em}
    \item What is the domain for the sine function?
    \vspace{1em}
    \item Define a small part of the domain for which the sine
    function is one-to-one. (There is not a unique answer - just find
    one small part.)  Make a sketch of the sine function restricted to
    your domain and explain why it is now one-to-one.

    \vfill

  \end{subproblem}

\end{problem}


\actTitle{Inverse Trigonometric Functions}
\begin{problem}
\item Use your calculator to evaluate the following
  expressions. Explain why it gives your results.
  \begin{subproblem}
    \item $\arcsin\left(\sin\left(\frac{\pi}{4}\right)\right)$
      \vfill
    \item $\arcsin\left(\sin\left(\frac{3\pi}{4}\right)\right)$
      \vfill
  \end{subproblem}

  \clearpage

\item A plane is a straight line distance of 10,000 meters away from a
  radar station. Its altitude is 3,000 meters above the ground. It is
  flying west away from a radar station. What is the distance between
  the radar station and a point on the ground directly below the
  plane?
  \vfill

  \clearpage

\item Determine the value of 
  \begin{eqnarray*}
    \tan\left(\arcsin\left(\frac{1}{3}\right)\right)
  \end{eqnarray*}

  \begin{subproblem}
  \item Make a sketch of a right triangle. Mark the appropriate angle
    and sides.
    \vfill
  \item Determine the value based on your diagram.
    \vfill
    \vfill
  \end{subproblem}

  \clearpage

\item Determine the value of 
  \begin{eqnarray*}
    \tan\left(\mathrm{arcsec}\left(\frac{4}{3}\right)\right)
  \end{eqnarray*}
  without using a calculator.

  \begin{subproblem}
  \item Make a sketch of a right triangle. Mark the appropriate angle
    and sides.
    \vfill
  \item Determine the value based on your diagram.
    \vfill
    \vfill
  \end{subproblem}

\end{problem}

\postClass

\begin{problem}
\item Briefly state two ideas from today's class.
  \begin{itemize}
  \item 
  \item 
  \end{itemize}
\item 
  \begin{subproblem}
    \item
  \end{subproblem}
\end{problem}


%%% Local Variables:
%%% mode: latex
%%% TeX-master: "../labManual"
%%% End:

