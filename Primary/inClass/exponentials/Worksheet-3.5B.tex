

\actTitle{Worksheet 3.5B}


\noindent \textbf{Instructions:}  Work together in groups of  3 or 4 to complete the following problems.\\


\begin{enumerate}
\item Solve the equation $\displaystyle 4^{-x+12}=19$ in three different ways.  Note that the form your answer takes will be different depending on the method, but that all answers should agree.
\begin{enumerate}
\item by converting to logarithmic form \vfill
\item by taking $\displaystyle \log_4$ of both sides\vfill
\item by taking $\ln$ of both sides\vfill
\end{enumerate}

\clearpage
\item Solve the equation $\log_5(2x+19)=8$ in two different ways:
\begin{enumerate}
\item by converting to exponential form\\[1in]
\item by ``exponentiating" both sides with base 5\\[1in]
\end{enumerate}



\item For the equation $5^{2x-7}=4^{9x+12}$, why don't we want to convert to logarithmic form? Solve the equation in two different ways:
\sideNote{\underline{The point}: You can use \emph{any} log to solve an exponential equation. Some are just more convenient than others. A good candidate is the common log, the natural log, or a log using a ``base" in the equation.}


\begin{tabular}{p{0.5\textwidth} | p{0.5\textwidth}} 
(a) by taking the $\ln$ of both sides\hspace{1in} & (b) by taking $\log_4$ of both sides\\
& \rule{0cm}{0.5\textheight}  \\ 
\end{tabular} 

\vfill

\clearpage

\item Solve the logarithmic equation. (Make sure your answers are
  consistent with the original equation.)
\begin{enumerate}
\item $6\log_5(4p-3)-2=16$\vfill
\item $\log(x)+\log(x-7)=\log(x-15)$\vfill
\end{enumerate}


\clearpage

\item If a couple has \$80,000 in a retirement account, how long will it take the money to grow to \$1,000,000 if it grows by 6\% compounded continuously?  Round to the nearest year.\vfill

\item An \$8000 investment grows to \$9289.50 at 3\% interest compounded quarterly.  For how long was the money invested?  Round to the nearest year.\vfill

\item \$20,000 is invested at 3.5\% interest compounded monthly.  How long will it take for the investment to triple?  Round to the nearest tenth of a year.
\vfill
\vfill

\clearpage

\item The body volume versus the brain volume for an order of beetles,
  Coleoptera, was examined in a paper by Polilov and
  Makarova.\footnote{The scaling and allometry of organ size
    associated with miniaturization in insects: A case study for
    Coleoptera and Hymenoptera, Polilov, Alexey A. and Makarova,
    Anastasia A., \textit{Scientific Reports}, Number 7, 22 Feb 2017.}
  The authors found that the volume of an insect's brain was related
  to the volume of the insect's body by the relationship
  \begin{eqnarray*}
    \textrm{Brain Volume} & = & 0.0800 \cdot \left( \textrm{Total Volume} \right)^{0.699}
  \end{eqnarray*}
  (All volumes are measured in nanoliters (nl)).
  \begin{enumerate}
  \item A beetle is captured, and its total volume is estimated to be
    250nl. What is the estimated volume of its brain? (Your answer
    should be to within 0.01nl.)
      
    \vfill
    
  \item A fragment of an ancient beetle is found in a piece of amber,
    and the volume of its brain is estimated to be 50nl. Assuming that
    the relationship above still holds, determine an estimate for the
    volume of the whole beetle.  (Your answer should be to within
    0.01nl.)

    \vfill
    
  \end{enumerate}

  \clearpage

\item The root mass of a tree, $RM$ (kg), \textit{castanopsis eyrei},
  and the diameter,$D$ (cm), of the tree's trunk is related
  by\footnote{The Allometry of Coarse Root Biomass: Log-Transformed
    Linear Regression or Nonlinear Regression?  Jiangshan Lai,Bo
    Yang,Dunmei Lin,Andrew J. Kerkhoff,Keping Ma, October 8, 2013,
    \url{https://doi.org/10.1371/journal.pone.0077007}}
  \begin{eqnarray*}
    RM & = & a \cdot D^{b},
  \end{eqnarray*}
  where $a$ and $b$ are constants.
  \begin{enumerate}
  \item If the root mass of a tree with a trunk diameter of 10cm is 2
    kg, and the root mass of a tree with a trunk diameter of 20 cm is
    25 kg, determine the values of $a$ and $b$.

    \vfill

  \item Using your values from the previous part, if the root mass of
    a tree is 15 kg what is the diameter of the tree?

    \vfill
      
  \end{enumerate}

%\clearpage
%
%\item Prove the following expression, and indicate for which values of $a$, $b$, and $c$ the expression is defined and valid.
%
%$$\log_a(b)\log_b(c) = \log_a(c)$$


\end{enumerate}

\hwTitle{Section 3.5B}

\begin{enumerate}
\item Solve the logarithmic equation.  Then check your answers.
  \begin{enumerate}
  \item $\log(q-6)=3.5$\vfill
  \item $\log_3(y)+\log_3(y+6)=3$\vfill
  \item $\log_3(n-5)+\log_3(n+3)=2$\vfill
  \end{enumerate}
\item A \$2500 bond grows to \$3729.56 in 10 years under continuous
  compounding.  Find the average interest rate.  Round to the nearest
  whole percent.
  \vfill
\item Use the formula pH=$-\log(\text{H}^+)$ to determine the value of $\text{H}^+$ for the following liquids given their pH values.
\begin{enumerate}
\item Seawater pH:  8.5 \vfill
\item Acid rain pH:  2.3\vfill
\end{enumerate}

\item Kleiber's law is used to approximate an animal's metabolic
  rate. For a particular species of feline it is estimated that an
  animal's metabolic rate is approximated by
  \begin{eqnarray*}
    E & = & 0.0012 M^{0.75},
  \end{eqnarray*}
  where $M$ is the animal's mass in kilograms, and $E$ is the animals
  metabolic rate in Watts.

    %https://en.wikipedia.org/wiki/Harris%E2%80%93Benedict_equation
    %https://en.wikipedia.org/wiki/Basal_metabolic_rate
    %https://en.wikipedia.org/wiki/Kleiber%27s_law
  \begin{enumerate}
  \item A feline of the given species is captured, and its mass is
    4.5kg. What is the animal's metabolic rate?
  \item A feline of the given species is captured, and its metabolic
    rate is estimated to be 0.0041W. What is the estimate for its
    mass?
  \item A new species of sea invertebrate is discovered. The mass of a
    speciman is 0.1 kg, and its metabolic rate is 0.000031 Watt. If
    the metabolic rate is approximated by
    \begin{eqnarray*}
      E & = & 0.03 M^\alpha
    \end{eqnarray*}
    determine the value of $\alpha$.
  \end{enumerate}

\end{enumerate}
