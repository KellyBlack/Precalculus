\actTitle{3.4 - Properties of Logarithms}

\noindent \textbf{Topics:}  properties of logarithms, change of base\\

\noindent \textbf{Student Learning Outcomes:}
\begin{enumerate}
\item Students will be able to apply the product, quotient, and power properties of logarithms.
\item Students will be able to write a logarithm in expanded form.
\item Students will be able to write a logarithm as a single logarithm.
\item Students will be able to apply the change of base formula.
\end{enumerate}

\hrule 

\bigskip

\subsection{Properties of Logarithms} ~


\noindent \textbf{Product Property of Logarithms:} Let $b, x,$ and $y$ be positive real numbers where $b \neq 1.$  Then $$\log_b (xy)=\log_b (x) + \log_b(y).$$   



\begin{enumerate}
\item Write the logarithm as a sum and simplify if possible.  Assume $x$ and $y$ represent positive real numbers. \\
\begin{enumerate}
\item $\log_2 (8x)$\\[.2in]
\item $\ln (5xy)$\\[.2in]
\end{enumerate}

\noindent \textbf{Quotient Property of Logarithms:} Let $b, x,$ and $y$ be positive real numbers where $b \neq 1.$  Then $$\log_b \Bigg(\frac{x}{y}\Bigg)=\log_b (x) - \log_b(y).$$   



\item Write the logarithm as a difference of logarithms and simplify if possible.  Assume $x$ and $y$ represent positive real numbers. \\
\begin{enumerate}
\item $\displaystyle \log_3 \Bigg(\frac{c}{d}\Bigg)$\\[.2in]
\item $\displaystyle \log \Bigg(\frac{x}{100}\Bigg)$\\[.2in]
\end{enumerate}


\noindent \textbf{Power Property of Logarithms:} Let $b$ and $x$ be positive real numbers where $b \neq 1.$ Let $p$ be any real number. Then $$\log_b (x^p)=p \log_b (x).$$   



\item Apply the power property of logarithms. \\
\begin{enumerate}
\item $\displaystyle\ln \sqrt[5]{x^2}$\\[.2in]
\item $\displaystyle \log x^2$\\[.2in]
\end{enumerate}








\subsection{Writing a Logarithmic Expression in Expanded Form}

\item Write the expression as the sum or difference of logarithms.
\begin{enumerate}
\item $\displaystyle \log_2 \Bigg(\frac{z^3}{xy^5}\Bigg)$\vfill
\item $\displaystyle \log \sqrt[3]{\frac{(x+y)^2}{10}}$\vfill
\end{enumerate}

\newpage

\subsection{Writing a Logarithmic Expression as a Single Logarithm}

\item Write the expression as a single logarithm and simplify the result if possible.
\begin{enumerate}
\item $\log_2 560 - \log_2 7- \log_2 5$\\[1.5in]
\item $\frac{1}{2} \ln x + \ln (x^2-1)- \ln(x+1))$\\[1.5in]
\end{enumerate}



\subsection{Change of Base Formula} ~

\noindent \textbf{Change-of-Base Formula: } Let $a$ and $b$ be positive real numbers such that $a \neq 1$ and $b \neq 1.$  Then for any positive real number $x$,
$$\log_b x=\frac{\log x}{\log b}=\frac{\ln x}{\ln b}$$


\item Use the change of base formula to approximate $\log_4 153$ by using base $e$.\vfill

\end{enumerate}

\noindent \textbf{Student Learning Outcomes Check}

\begin{enumerate}
\item Can you apply the product, quotient, and power properties of logarithms?
\item Can you write a logarithm in expanded form?
\item Can you write a logarithm as a single logarithm?
\item Are you able to apply the change of base formula?
\end{enumerate}

\noindent \textbf{If any of your answers were no, please ask about these topics in class.}


